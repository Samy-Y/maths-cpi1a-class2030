\documentclass[12pt, a4paper]{report}

\usepackage{elegant-cours}

\begin{document}

\MaPageDeGarde{Chapitre VIII : Intégration d'une fonction sur un segment}{Rédigé par Samy Youssoufine}{./assets/logo.png}{Document WIP. Peut contenir des erreurs/sections incomplètes. Version ALPHA de la nouvelle mise en forme.}

\tableofcontents
\clearpage

\vspace*{\fill}
Ce chapitre est consacré à l'étude de l'intégration d'une fonction sur un segment. Les objectifs principaux sont de :
\begin{itemize}
    \item Rappeler les définitions et propriétés de base des intégrales.
    \item Rappeler les techniques d'intégration usuelles, notamment l'intégration par parties et le changement de variable.
    \item Étendre et généraliser ces techniques d'intégration.
    \item Présenter de nouvelles formules (formule de Taylor-Lagrange avec reste intégral...).
\end{itemize}

Ce chapitre ne traitera pas des intégrales impropres, généralisées, etc. Ces notions seront abordées l'année prochaine.
\vspace*{\fill}

\newpage

\chapter{Généralités}

\section{Primitives, intégrales et propriétés de base}

\begin{definition}[Primitives]
    Soit $f\in\Kset^I$.

    La fonction $F\in\Kset^I$ est une primitive de $f$ sur $I$ si $F$ est dérivable sur $I$ et $F'=f$.

    On dit "une primitive" et non pas "la primitive" car une fonction peut avoir plusieurs primitives différentes (différant d'une constante additive).

    Si $F$ est une primitive de $f$ sur $I$, alors l'ensemble des primitives de $f$ sur $I$ est l'ensemble des fonctions de la forme $F+c$ avec $c\in\Kset$.
\end{definition}

\begin{definition}[Intégrale d'une fonction entre deux points]
    Soit $f\in\Kset^I$.

    L'intégrale de $f$ entre $a$ et $b$ est notée $\int_a^b f$ et est définie par :
    $$\int_a^b f = F(b) - F(a)$$
    où $F$ est une primitive de $f$ sur $[a,b]$.
\end{definition}

\begin{remark}
    Si $f \in \C^I$, alors $\int_a^b f(t)dt = \int_a^b\Re(f)(t)dt + i\int_a^b \Im(f)(t)dt$.

    Par exemple, $\int_0^1 \frac{dt}{t+i} = \dots = \frac12 \ln(2) - i\frac{\pi}{4}$.

    Attention ! Nous n'avons pas défini la fonction $\ln$ sur $\C$. Ici, nous utilisons la définition de $\ln$ sur $\R$. Il existe bel et bien un logarithme complexe, mais il entre dans le domaine des fonctions holomorphes, ce qui est hors de portée de ce cours.
\end{remark}

\begin{theorem}
    Si $f \in \Kset^I$ est continue sur $I$, alors $f$ admet une primitive sur $I$.
\end{theorem}

\begin{proof}
    
\end{proof}
\todo{recop}

\begin{property}[Propriétés de l'intégrale]
    Soient $f,g\in\mathcal{C}(I,\Kset)$, et $a,b\in I$.

    \begin{enumerate}
        \item $\forall \lambda \in \Kset, \int_a^b f+\lambda g = \int_a^b f + \lambda \int_a^b g$.
        \item Relation de Chasles : $\forall c \in I, \int_a^b f = \int_a^c f + \int_c^b f$.
        \item Si $f\le g$ sur I et $a\le b$, alors $\int_a^b f \le \int_a^b g$.
        \item Si $a<b$, $\abs{\int_a^b f} \le \int_a^b \abs{f}$\\, et même $\le \sup_{t\in[a,b]} \abs{f(t)} (b-a)$ (Inégalité triangulaire de l'intégrale, ou encore inégalité de la moyenne).
    \end{enumerate}

\end{property}

\begin{proof}
    
\end{proof}
\todo{recop}

\begin{consequence}
    Soit $f\in\mathcal{C}(I,\R),a<b\in I$.

    Si $f\ge 0$ sur $I$, alors $\int_a^b f(t)dt \ge 0$.

    Avec $\int_a^b f(t)dt = 0 \ssi f=0$ sur $[a,b]$.
\end{consequence}

\begin{proof}
    
\end{proof}
\todo{recop}

\begin{exercise}
    Soit $f:[a,b]\to\C$ une fonction continue.

    Montrer que $\abs{\int_a^b f(t) dt} = \int_a^b \abs{f(t)} dt$ si et seulement si il existe $\theta \in \R$ tel que $\forall t \in [a,b], f(t) = e^{i\theta} \abs{f(t)}$.

    \textbf{Solution :}

    \begin{outline}
        \1[$\leftarrow$] La démonstration dans le sens indirect est triviale.
        \1[$\rightarrow$] Pour démontrer cette propriété dans le sens direct
    \end{outline}
\end{exercise}
\todo{recop}

\begin{remark}
    Si $f:[a,b]\to\R$ est une fonction continue telle que $\abs{\int_a^b f(t) dt} = \int_a^b \abs{f(t)} dt$, alors $f$ est de signe constant sur $[a,b]$. Il s'agit d'un cas particulier du résultat précédent.
\end{remark}

\section{Intégration par parties}

\subsection{Formule d'intégration par parties}

\begin{theorem}
    Soient $f,g\in\mathcal{C}^2([a,b],\Kset)$, et $a,b\in I$.

    $$\int_a^b f\cdot g' = \bracks{f\cdot g}_a^b - \int_a^b f'\cdot g$$

    avec $\bracks{f}_a^b = f(b) - f(a)$.
\end{theorem}

\begin{proof}
    On a $(f \cdot g)' = f' \cdot g + f \cdot g'$.

    Donc, en intégrant entre $a$ et $b$, on obtient :

    $$\int_a^b (f \cdot g)' = \int_a^b f' \cdot g + \int_a^b f \cdot g'$$

    Donc $\int_a^b f\cdot g' = \bracks{fg}_a^b - \int_a^b f'\cdot g$.
\end{proof}

\begin{example}
    $$\int_a^x \ln(t) dt \text{ avec } a,x > 0$$

    On pose $\begin{cases} u(t) = \ln(t) &\implies u'(t) = \frac{1}{t} \\ v'(t) = 1 &\implies v(t) = t \end{cases}$.

    On a donc : $\int_a^x \ln(t) dt = \bracks{t\ln(t)}_a^x - \int_a^x 1 dt = x\ln(x) - a\ln(a) - (x-a)$.

    Donc : $\int_a^x \ln(t) dt = x\ln(x) - x - (a\ln(a) - a)$.
\end{example}

\subsection{Formule d'IPP. généralisée}

\begin{exercise}[Formule d'intégration par parties généralisée]
    Soient $f,g\in\mathcal{C}^n(I,\Kset)$, et $a,b\in I$.

    On souhaite démontrer que :

    $$\int_a^b f\cdot g^{(n)}=\bracks{\sum_{k=1}^n (-1)^{k-1} f^{(k-1)}\cdot g^{(n-k)}}_a^b + (-1)^n \int_a^b f^{(n)}\cdot g$$

    \textbf{Solution :}

    On peut procéder par récurrence sur $n$, mais il est plus simple de faire une démonstration directe par intégration par parties.

    \begin{align*}
        \parens{\sum_{k=1}^n (-1)^{k-1} f^{(k-1)}\cdot g^{(n-k)}}' &= \sum_{k=1}^n (-1)^{k-1} \parens{f^{(k)}\cdot g^{(n-k)} + f^{(k-1)}\cdot g^{(n-k+1)}} \\
        &= \sum_{k=1}^n (-1)^{k-1} f^{(k)}\cdot g^{(n-k+1)} - (-1)^k f^{(k)}\cdot g^{(n-k)} \\
        &= f\cdot g^{(n)} - (-1)^n f^{(n)}\cdot g
    \end{align*}

    Donc, en intégrant entre $a$ et $b$, on obtient :
    $$\int_a^b \parens{\sum_{k=1}^n (-1)^{k-1} f^{(k-1)}\cdot g^{(n-k)}}' = \int_a^b f\cdot g^{(n)} - (-1)^n \int_a^b f^{(n)}\cdot g$$

    Ce qui donne bien la formule voulue.
\end{exercise}

\begin{application}
    On souhaite calculer $J=\int_a^b P(x)e^{cx}dx$ avec $P\in\Kset_n[X]$ et $c\in\R_+^*$.

    On pose $f:x\mapsto e^{cx}$, qui est $\mathcal{C}^\infty$ sur $\R$.

    On a, pour tout $n \in \N, f^{(n+1)}:x\mapsto c^{n+1} e^{cx}$.

    \begin{align*}
        J&=\int_a^b P(x)e^{cx}dx\\
        &=\frac{1}{c^{n+1}}\int_a^b P(x) f^{(n+1)}(x) dx\\
        &=\frac{1}{c^{n+1}}\bracks{\sum_{k=1}^{n+1} (-1)^{k-1} P^{(k-1)}(x) f^{(n+1-k)}(x)}_a^b + \frac{(-1)^{n+1}}{c^{n+1}}\int_a^b P^{(n+1)}(x) f(x) dx
    \end{align*}

    Or, on sait que $P$ est de degré $n$, donc $P^{(n+1)}\equiv 0$.

    Donc, on a finalement :

    $$J=\frac{1}{c^{n+1}}\bracks{\sum_{k=1}^{n+1} (-1)^{k-1} P^{(k-1)}(x) f^{(n+1-k)}(x)}_a^b$$

    $$J=\bracks{\sum_{k=0}^n(-1)^k \cdot P^{(k)}(x)\frac{e^{cx}}{c^{k+1}}}_a^b$$
\end{application}

\section{Changement de variable}

\begin{definition}[Intégration par changement de variable]
   Soient $g\in\mathcal{C}^1([a,b],\R)$ et $f\in\mathcal{C}(I,\Kset)$ telles que $g([a,b])\subset I$.

    $$\int_a^b f(g(x))\cdot g'(x) dx = \int_{g(a)}^{g(b)} f(t) dt$$

    On dit qu'on a fait le changement de variable suivant :
    
    $$t=g(x)\implies \begin{cases}x=a&\rightarrow t=g(a)\\x=b&\rightarrow t=g(b)\end{cases}$$
    
    avec $dt=g'(x)dx$.

    Donc $\int_a^b \underbrace{f(g(x))}_{=t}\cdot \underbrace{g'(x)dx}_{=dt}=\int_{g(a)}^{g(b)}f(t)dt$
\end{definition}

\begin{proof}
    Soit $F$ une primitive de $f$ sur $I$.

    On a donc :

    \begin{align*}
        \int_a^b f(g(x))g'(x)dx&=\int_a^b F'(g(x))\cdot g'(x)dx\\
        &=\int_a^b (F\circ g)'(x)dx=F(g(b))-F(g(a))\\
        &=\int_{g(a)}^{g(b)}F'(t)dt\\
        &=\int_{g(a)}^{g(b)}f(t)dt
    \end{align*}
\end{proof}

\begin{example}
    Calculer $I_{n,p}=\int_a^b=(b-x)^n\cdot (x-a)^p dx$, avec $n,p\in\N$.

    \textit{(Ou bien, montrer que $I_{n,p}=(b-a)^{n+p+1}\cdot \sum_{k=0}^n\legbinom{n}{k}\frac{(-1)^k}{p+k+1}$)}

    \textbf{Solution :}

    On va utiliser le changement de variable qui permet de transformer n'importe quel $x\in[a,b]$ en un $t\in[0,1]$.

    On sait que $\forall x \in [a,b], \exists t \in [0,1], x=(1-t)a + tb$.

    On pose alors $x=(1-t)a+tb$. En dérivant, on obtient :

    $$dx = (b-a)dt$$

    Donc, on a :

    \begin{align*}
        I_{n,p} &= \int_a^b (b-x)^n (x-a)^p dx \\
        &= \int_0^1 (b-(1-t)a-tb)^n ((1-t)a+tb-a)^p (b-a)dt \\
        &= (b-a)^{n+p+1} \int_0^1 (1-t)^n t^p dt
    \end{align*}

    Or, on sait que :

    $$\int_0^1 (1-t)^n t^p dt = \int_0^1 t^p \sum_{k=0}^n \legbinom{n}{k} (-1)^k t^k dt = \sum_{k=0}^n \legbinom{n}{k} (-1)^k \int_0^1 t^{p+k} dt$$

    Donc, on a finalement :

    $$\int_0^1 (1-t)^n t^p dt = \sum_{k=0}^n \legbinom{n}{k} \frac{(-1)^k}{p+k+1}$$

    Donc, on en déduit que :

    $$I_{n,p} = (b-a)^{n+p+1} \cdot \sum_{k=0}^n \legbinom{n}{k} \frac{(-1)^k}{p+k+1}$$
\end{example}

\section{Formule de Taylor-Lagrange avec reste intégral}

\begin{theorem}[Formule de Taylor-Lagrange avec reste intégral]
    Soient $f\in\mathcal{C}^{n+1}(I,\Kset)$, $a,x\in I$.

    Alors on a :

    $$f(x) = \sum_{k=0}^n \frac{f^{(k)}(a)}{k!} (x-a)^k + \int_a^x \frac{(x-t)^n}{n!} f^{(n+1)}(t) dt$$
\end{theorem}

\begin{proof}
    On pose, pour $k\in\N$, $I_k=\int_a^x \frac{(x-t)^k}{k!} f^{(k+1)}(t) dt$.

    On effectue une intégration par parties sur $I_k$ :
    $$\begin{cases}
        u(t) = (x-t)^k &\implies u'(t) = -k(x-t)^{k-1} \\
        v'(t) = f^{(k+1)}(t) &\implies v(t) = f^{(k)}(t)
    \end{cases}$$

    Donc, on a :
    \begin{align*}
        I_k &= \bracks{\frac{(x-t)^k}{k!} f^{(k)}(t)}_a^x + \underbrace{\int_a^x \frac{k(x-t)^{k-1}}{k!} f^{(k)}(t) dt}_{=I_{k-1}}
    \end{align*}

    Donc $I_{k-1}-I_k = \frac{(x-a)^k}{k!} f^{(k)}(a)$.

    Donc $\sum_{k=1}^n (I_{k-1}-I_k) = \sum_{k=1}^n \frac{(x-a)^k}{k!} f^{(k)}(a)$.

    Donc $I_0 - I_n = \sum_{k=1}^n \frac{(x-a)^k}{k!} f^{(k)}(a)$.

    Or $I_0 = \int_a^x f^{(1)}(t) dt = f(x) - f(a)$.

    Donc, on a bien :

    $$f(x) = \sum_{k=0}^n \frac{f^{(k)}(a)}{k!} (x-a)^k + \int_a^x \frac{(x-t)^n}{n!} f^{(n+1)}(t) dt$$
\end{proof}

\begin{consequence}[Inégalité de Taylor-Lagrange]
    Même énoncé que précédemment écrit dans le chapitre 7.

    Soit $f \in \mathcal{C}^{n+1}(I,\R)$. 

    On a, pour tout $x\ne a \in I$ :

    $$\abs{f(x)-\sum_{k=0}^n\frac{f^{(k)}}{k!}(x-a)^k}\le\sup_{t\in[a,x] \text{ ou }t \in [x,a]}\abs{f^{(n+1)}(t)}\frac{\abs{x-a}^{n+1}}{(n+1)!}$$
\end{consequence}

\begin{proof}
    Soient $f\in\mathcal{C}^{n+1}(I,\Kset)$ et $x\not=a \in I$.

    Nous allons procéder à une démonstration différente à celle vue dans le chapitre 7 (nous allons utiliser la formule de Taylor-Lagrange avec reste intégral).

    On a, par la formule de Taylor-Lagrange avec reste intégral :

    \begin{align*}
        \abs{f(x)-\sum_{k=0}^n \frac{f^{(k)}(a)}{k!}(x-a)^k}&\le\abs{\int_a^x \frac{(x-t)^n}{n!}f^{(n+1)}(t)dt}\\
        &\le \int_a^x \abs{f^{(n+1)}(t)}\frac{(x-t)^n}{n!}dt\\
    \end{align*}

    Or $f^{(n+1)}$ est continue sur le segment $[a,x]$ ou $[x,a]$, donc elle est bornée.

    Donc, on a :

\end{proof}
\todo{recop}

\begin{application}
    On pose $f:x\mapsto e^{ix}$. Cette fonction est de classe $\mathcal{C}^\infty$ sur $\R$.

    Soit $x\in\R^*$. D'après l'inégalité de Taylor-Lagrange, on a :

    $$\abs{f(x)-\sum_{k=0}^n \frac{f^{(k)}(0)}{k!} x^k} \le \sup_{t\in[0,x] \text{ ou } t\in[x,0]} \abs{f^{(n+1)}(t)} \frac{\abs{x}^{n+1}}{(n+1)!}$$

    Or on sait que $\forall k \in \N, f^{(k)}:x\mapsto i^k e^{ix}$.

    Donc $f^{(k)}(0) = i^k$.

    Donc $\sup_{t\in[0,x] \text{ ou } t\in[x,0]} \abs{f^{(n+1)}(t)} \le 1$.

    Donc, on a finalement :

    $$\forall n \in N, \abs{e^{ix}-\sum_{k=0}^n \frac{i^k}{k!} x^k} \le \frac{\abs{x}^{n+1}}{(n+1)!}$$

    On a $\frac{\abs{x}^{n+1}}{(n+1)!} \to 0$ lorsque $n \to +\infty$ d'après le critère de d'Alembert.

    Donc, on a bien :

    $$\lim_{n \to +\infty} \abs{e^{ix}-\sum_{k=0}^n \frac{i^k}{k!} x^k} = 0$$

    Donc, on en déduit la formule d'Euler :

    $$\boxed{e^{ix} = \sum_{k=0}^{+\infty} \frac{(ix)^k}{k!}}$$

    En général, pour tout $z\in\C$, on a :
    $$\boxed{e^{z} = \sum_{k=0}^{+\infty} \frac{z^k}{k!}}$$
\end{application}

\chapter{Calcul des primitives}

On note $\int f(x)dx$ une primitive de $f$.
Le calcul de $\int f$ se fait après décomposition de $f$ en éléments plus simples (généralement les fonctions usuelles qui seront présentées dans la partie ci-dessous).

\section{Primitives des fonctions usuelles}

Les primitives des fonctions usuelles sont les suivantes :

\begin{itemize}
    \item $\forall k,\lambda \in \N\times\Kset, \int \lambda x^k dx = \lambda \frac{x^{k+1}}{k+1} + \gamma$ avec $\gamma \in \Kset$.
    \item $\forall n \in \Z\setminus\{-1\}, \forall a \in \Kset, \int (x-a)^n dx = \frac{(x-a)^{n+1}}{n+1} + \gamma$ avec $\gamma \in \Kset$.
    \item $\forall a \in \R$, $\int \frac{1}{x-a} dx = \ln\abs{x-a} + \gamma$ avec $\gamma \in \R$.
    \item $\forall a \in \C, \int \frac{dx}{x-a} = \frac12 \ln\parens{(x-\alpha)^2+\beta^2} + i\arctg\parens{\frac{x-\alpha}{\beta}}+\gamma$ avec $\gamma \in \C$.
    \item $\forall a,b\in\R \tqp a^2-4b<0 \et n\in \N^*, \int \frac{dx}{\parens{x^2+ax+b}^n}=\int \frac{dx}{\parens{\parens{x+\frac{a}{2}}^2+\parens{b-\frac{a^2}{4}}}^n}$.\\Par changement de variable, on peut aboutir à une nouvelle intégrale de la forme $\int \frac{dt}{(t^2+1)^n}$, qu'on peut calculer par IPP.
\end{itemize}

\begin{exercise}
    On pose $f:x\mapsto \frac{1}{(x^3+1)(x+1)}$.

    \begin{enumerate}
        \item Déterminer $a,b,c,d\in\R$ tels que $\forall x\in\R\setminus\{-1\}, f(x)=\frac{a}{x+1}+\frac{b}{(x+1)^2}+\frac{cx+d}{x^2 - x + 1}$.
        \item Calculer $\int f(x)dx$.
    \end{enumerate}

    \textbf{Solution :}

    \begin{outline}
        \1 On a $x^3+1=(x+1)(x^2 - x + 1)$.

        \2 \textbf{Méthode 1 :} Donc, on cherche $a,b,c,d\in\R$ tels que $\forall x\in\R\setminus\{-1\}, 1 = a(x+1)(x^2 - x + 1) + b(x^2 - x + 1) + (cx + d)(x+1)^2$.

        En développant et en regroupant les termes, on obtient :

        $$1 = (a+c)x^3 + (b - a + 2d)x^2 + (a - b + 2c + d)x + (b + d)$$

        En identifiant les coefficients, on obtient le système suivant :

        $$\begin{cases}
            a + c = 0 \\
            b - a + 2d = 0 \\
            b - d = 0 \\
        \end{cases}$$

        En résolvant ce système, on trouve $a = \frac{1}{3}, b = \frac{1}{3}, c = -\frac{1}{3}, d = \frac{1}{3}$.

        \2 \textbf{Méthode 2 :} On peut faire tendre $x$ vers $-1$ pour trouver $b$, après avoir multiplié par $(x+1)^2$. On trouve $b=\frac{1}{3}$. Ensuite, en faisant tendre $x$ vers $+\infty$ et avoir multiplié l'expression par $(x+1)$, on trouve $c=-a$. On peut ensuite poser $x=0$ et $x=1$ pour obtenir un système de deux équations à deux inconnues en $a$ et $d$. On trouve les mêmes valeurs que précédemment.
    \1 Pour la deuxième question, on a donc :
        \begin{align*}
            \int f(x) dx &= \int \parens{\frac{1/3}{x+1} + \frac{1/3}{(x+1)^2} + \frac{-\frac{1}{3}x + \frac{1}{3}}{x^2 - x + 1}} dx \\
            &= \frac{1}{3} \ln\abs{x+1} - \frac{1}{3(x+1)} -\frac{1}{3}\int \frac{x-1}{x^2 - x + 1} dx
        \end{align*}
        \2 Après quelques calculs, on trouve finalement :
        $$\int f(x) dx = \frac{1}{3} \ln\abs{x+1} - \frac{1}{3(x+1)} - \frac{1}{6} \ln\parens{x^2 - x + 1} + \frac{1}{\sqrt{3}} \arctg\parens{\frac{2x - 1}{\sqrt{3}}} + C$$
    \end{outline}
\end{exercise}

\section{$\int f(\cos(x), \sin(x)) dx$ où $f$ est une fraction rationnelle}

Pour calculer cette intégrale, il faut utiliser le changement de variable de Weierstrass : $t = \tan\parens{\frac{x}{2}}$. Cela veut dire que $x=2\arctg(t)$, et donc $dx = \frac{2}{1+t^2} dt$, sous réserve que $x \ne \pi + 2k\pi$ avec $k\in\Z$ (car $\tan$ n'est pas défini en ces points).

Ce changement de variable est efficace car il permet d'exprimer $\cos(x)$, $\sin(x)$ et $dx$ en fonction de $t$ de la manière suivante :
\begin{itemize}
    \item $\cos(x) = \frac{1 - t^2}{1 + t^2}$, car $\cos(x) = 2\cos^2\parens{\frac{x}{2}} - 1=\frac{2}{1+\tan^2(\frac x{2})}-1=\frac{2}{1+t^2}-1$
    \item $\sin(x) = \frac{2t}{1 + t^2}$, car $\sin(x) = 2\sin\parens{\frac{x}{2}}\cos\parens{\frac{x}{2}}=\frac{2\tan(\frac x{2})}{1+\tan^2(\frac x{2})}=\frac{2t}{1+t^2}$
    \item $dx = \frac{2}{1 + t^2} dt$
\end{itemize}

\begin{remark}
    On peut également appliquer la règle de Bioche, qui consiste à effectuer le changement de variable suivant :

    $$\begin{cases}
    \int \underbrace{f(\cos x, \sin x) dx}_{=\omega(x)}\\
    \text{Si } \omega(-x) = \omega(x) &\implies t = \cos x,~~\omega \text{ invariante par } -x\\
    \text{Si } \omega(\pi - x) = \omega(x) &\implies t = \sin x,~~\omega \text{ invariante par } \pi-x\\
    \text{Si } \omega(\pi + x) = \omega(x) &\implies t = \tan x,~~\omega \text{ invariante par } \pi+x\\
    \text{Sinon} &\implies t = \tan\parens{\frac{x}{2}}
    \end{cases}$$
\end{remark}

\begin{example}
    \begin{outline}[enumerate]
        \1 $I_1=\int \frac{dx}{\sin(x)}$.
        \2 On pose $t=\cos(x)\implies dt=-\sin(x)dx$.
        \2 Donc, on a $I_1 = -\int \frac{dt}{1-t^2} = -\frac12 \int \parens{\frac{1}{1-t} + \frac{1}{1+t}} dt$.
        \2 Donc, on trouve finalement $I_1 = \frac12 \ln\parens{\abs{\frac{1+\cos(x)}{1-\cos(x)}}} + C$, soit $I_1 = -\argth\parens{\cos(x)} + C$.
        \2 On peut aussi directement remarquer à partir de la deuxième ligne qu'on peut utiliser la formule de $\argth$.
        \1 $I_2=\int \frac{\sin(x)-\cos(x)}{1+\cos(x)}dx$.
        \2 On pose $t=\tan\parens{\frac{x}{2}}$.
        \2 Donc $I_2 = \int \frac{\frac{2t}{1+t^2} - \frac{1-t^2}{1+t^2}}{1 + \frac{1-t^2}{1+t^2}} \cdot \frac{2}{1+t^2} dt$.
        \2 Après simplification, on trouve $I_2 = \int \frac{t^2+2t-1}{1+t^2} dt$.
        \2 On sépare en deux intégrales : $I_2 = \int (1 + \frac{2t}{1+t^2}) dt - 2\int \frac{1}{1+t^2} dt$.
        \2 On trouve finalement $I_2 = \tan(\frac x2)+\ln(1+\tan^2(\frac x2)) - x + C$.
    \end{outline}
\end{example}

Fin du chapitre VIII.

\end{document}