\documentclass[12pt, a4paper]{report}

% --- PACKAGES ESSENTIELS ---
\usepackage[french]{babel}
\usepackage[utf8]{inputenc}
\usepackage[T1]{fontenc}
\usepackage{lmodern} % Police plus nette et moderne
\usepackage{amsmath, amssymb, amsthm, amsfonts} % Importe amsthm pour les environnements
\usepackage{geometry}
\usepackage{graphicx}
\usepackage{todonotes}
\usepackage{outlines}
\usepackage{pgfplots}
\pgfplotsset{compat=1.18}
\usepackage{float}
\usepackage{afterpage}
\usepackage{placeins}
\usepackage{titlesec}
\usepackage{tcolorbox}
\tcbuselibrary{skins}
\usepackage{ragged2e}

% Clear any pending floats before document end
\AtEndDocument{\clearpage}

% --- PACKAGES POUR LE STYLE ---
\usepackage[x11names, dvipsnames]{xcolor}
\usepackage{framed}
\usepackage{tikz}
\usepackage{tikz}
\usepackage[
    unicode=true,
    pdfusetitle,
    colorlinks=true,
    linkcolor=MidnightBlue,
    urlcolor=Firebrick4,
    citecolor=Green4
]{hyperref}
\usetikzlibrary{calc,arrows.meta}

% --- CONFIGURATION DE LA MISE EN PAGE ---
\geometry{
    a4paper,
    top=25mm,
    bottom=25mm,
    left=20mm,
    right=20mm,
}
\setlength{\parindent}{0pt}
\setlength{\parskip}{0.8em}

% --- DÉFINITION D'UNE PALETTE DE COULEURS ---
\definecolor{mainBlue}{HTML}{003366}
\definecolor{accentBlue}{HTML}{005b96}
\definecolor{boxBg}{HTML}{E8F0F7}
\definecolor{boxFrame}{HTML}{005b96}
\definecolor{remarkBg}{HTML}{F0F8FF}
\definecolor{proofBg}{HTML}{F6F6F6}

% Colors for different environments
\definecolor{definitionColor}{HTML}{2E86C1}    % Blue for definitions
\definecolor{theoremColor}{HTML}{28B463}       % Green for theorems
\definecolor{propertyColor}{HTML}{E74C3C}      % Red for properties
\definecolor{propositionColor}{HTML}{8E44AD}   % Purple for propositions
\definecolor{remarkColor}{HTML}{F39C12}        % Orange for remarks
\definecolor{exampleColor}{HTML}{17A2B8}       % Teal for examples
\definecolor{exerciseColor}{HTML}{6C757D}      % Gray for exercises
\definecolor{consequenceColor}{HTML}{DC3545}   % Dark red for consequences
\definecolor{applicationColor}{HTML}{FF6B35}   % Orange-red for applications
\definecolor{proofColor}{HTML}{495057}         % Dark gray for proofs

% Background colors (lighter versions of heading colors)
\definecolor{definitionBg}{HTML}{EBF3FD}       % Light blue background
\definecolor{theoremBg}{HTML}{E8F6F0}          % Light green background
\definecolor{propertyBg}{HTML}{FADBD8}         % Light red background
\definecolor{propositionBg}{HTML}{F4ECF7}      % Light purple background
\definecolor{remarkBg}{HTML}{FEF9E7}           % Light orange background
\definecolor{exampleBg}{HTML}{E1F5F8}          % Light teal background
\definecolor{exerciseBg}{HTML}{F8F9FA}         % Light gray background
\definecolor{consequenceBg}{HTML}{FADBD8}      % Light dark red background
\definecolor{applicationBg}{HTML}{FFF4F1}      % Light orange background
\definecolor{proofBg}{HTML}{F8F9FA}            % Light gray background

% --- PERSONNALISATION DES TITRES ---
\titleformat{\chapter}[display]
  {\normalfont\Huge\bfseries\color{mainBlue}\raggedright}
  {Partie\ \thechapter}
  {5pt}
  {\Huge\color{mainBlue}\raggedright}
\titlespacing*{\chapter}{0pt}{-30pt}{40pt}
\titleformat{\section}{\normalfont\Large\bfseries\color{accentBlue}\raggedright}{\thesection}{1em}{}
\titleformat{\subsection}{\normalfont\large\bfseries\color{accentBlue}\raggedright}{\thesubsection}{1em}{}

% --- DÉFINITION DES STYLES D'ENVIRONNEMENTS ---
% Définition des styles de boîtes tcolorbox
\tcbset{
  coursebox/.style={
    colback=boxBg,
    colframe=boxFrame,
    sharp corners,
    boxrule=0.5pt,
    arc=0mm,
    boxsep=5pt,
    before=\par\bigskip,
    after=\par\bigskip,
  },
  remarkbox/.style={
    colback=remarkBg,
    colframe=MidnightBlue!50,
    sharp corners,
    boxrule=0.5pt,
    arc=0mm,
    boxsep=5pt,
    before=\par\bigskip,
    after=\par\bigskip,
  },
  proofbox/.style={
    colback=proofBg,
    colframe=Gray,
    sharp corners,
    boxrule=0.5pt,
    arc=0mm,
    boxsep=5pt,
    before=\par\bigskip,
    after=\par\bigskip,
  }
}

% Déclaration des environnements théorème avec amsthm
\newtheorem{definition}{Définition}
\newtheorem{proposition}{Proposition}
\newtheorem{theorem}{Théorème}
\newtheorem{property}{Propriété}
\newtheorem{remark}{Remarque}
\newtheorem{example}{Exemple}
\newtheorem{exercise}{Exercice}
\newtheorem{consequence}{Conséquence}
\newtheorem{application}{Application}
% Définition de l'environnement importantbox amélioré
\newenvironment{importantbox}[1][]{%
  \begin{tcolorbox}[
    colback=yellow!10,
    colframe=orange!70!red,
    colbacktitle=orange!70!red,
    coltitle=white,
    title=\textbf{Formule importante},
    fonttitle=\bfseries,
    arc=3pt,
    boxrule=1.5pt,
    enhanced,
    drop fuzzy shadow,
    center title,
    #1
  ]%
  \centering\large\bfseries
}{\end{tcolorbox}}
% Nouvel environnement pour les formules clés
\newenvironment{keyformula}[1][]{%
  \begin{tcolorbox}[
    colback=blue!5,
    colframe=blue!70,
    colbacktitle=blue!70,
    coltitle=white,
    title=\textbf{Formule clé},
    fonttitle=\bfseries,
    arc=2pt,
    boxrule=1pt,
    center title,
    enhanced,
    drop fuzzy shadow,
    #1
  ]%
  \centering\large
}{\end{tcolorbox}}
  \centering\large
% Environnement pour les méthodes importantes
\newenvironment{methodbox}[1][]{%
  \begin{tcolorbox}[
    colback=green!8,
    colframe=green!60!black,
    colbacktitle=green!60!black,
    coltitle=white,
    title=\textbf{Méthode},
    fonttitle=\bfseries,
    arc=2pt,
    boxrule=1pt,
    enhanced,
    #1
  ]%
  \normalfont
}{\end{tcolorbox}}
  \normalfont
% Environnement pour les points d'attention
\newenvironment{attention}[1][]{%
  \begin{tcolorbox}[
    colback=red!8,
    colframe=red!70,
    colbacktitle=red!70,
    coltitle=white,
    title=\textbf{$\triangle$ Attention},
    fonttitle=\bfseries,
    arc=2pt,
    boxrule=1pt,
    enhanced,
    #1
  ]%
  \normalfont
}{\end{tcolorbox}}

% Style pour les listes élégantes
\usepackage{enumitem}
\newlist{elegantlist}{itemize}{3}
\setlist[elegantlist,1]{
  label=\textcolor{accentBlue}{$\blacktriangleright$},
  leftmargin=1.5em,
  itemsep=0.3em
}
\setlist[elegantlist,2]{
  label=\textcolor{accentBlue}{$\circ$},
  leftmargin=2em,
  itemsep=0.2em
}
\setlist[elegantlist,3]{
  label=\textcolor{accentBlue}{$-$},
  leftmargin=2.5em,
  itemsep=0.1em
}

% Redéfinition des environnements existants avec tcolorbox
\renewenvironment{definition}[1][]{%
  \refstepcounter{definition}%
  \begin{tcolorbox}[colback=definitionBg, colframe=definitionColor, 
    sharp corners, boxrule=0.5pt, arc=0mm, boxsep=5pt,
    before=\par\bigskip, after=\par\bigskip,
    title=\textbf{Définition \thechapter.\thesection.\thedefinition%
    \ifx&#1&\else\ \textit{(#1)}\fi%
    }, fonttitle=\bfseries\color{white}, colbacktitle=definitionColor]%
  \normalfont
}{\end{tcolorbox}}

\renewenvironment{proposition}[1][]{%
  \refstepcounter{proposition}%
  \begin{tcolorbox}[colback=propositionBg, colframe=propositionColor,
    sharp corners, boxrule=0.5pt, arc=0mm, boxsep=5pt,
    before=\par\bigskip, after=\par\bigskip,
    title=\textbf{Proposition \thechapter.\thesection.\theproposition%
    \ifx&#1&\else\ \textit{(#1)}\fi%
    }, fonttitle=\bfseries\color{white}, colbacktitle=propositionColor]%
  \normalfont
}{\end{tcolorbox}}

\renewenvironment{theorem}[1][]{%
  \refstepcounter{theorem}%
  \begin{tcolorbox}[colback=theoremBg, colframe=theoremColor,
    sharp corners, boxrule=0.5pt, arc=0mm, boxsep=5pt,
    before=\par\bigskip, after=\par\bigskip,
    title=\textbf{Théorème \thechapter.\thesection.\thetheorem%
    \ifx&#1&\else\ \textit{(#1)}\fi%
    }, fonttitle=\bfseries\color{white}, colbacktitle=theoremColor]%
  \normalfont
}{\end{tcolorbox}}

\renewenvironment{property}[1][]{%
  \refstepcounter{property}%
  \begin{tcolorbox}[colback=propertyBg, colframe=propertyColor,
    sharp corners, boxrule=0.5pt, arc=0mm, boxsep=5pt,
    before=\par\bigskip, after=\par\bigskip,
    title=\textbf{Propriété \thechapter.\thesection.\theproperty%
    \ifx&#1&\else\ \textit{(#1)}\fi%
    }, fonttitle=\bfseries\color{white}, colbacktitle=propertyColor]%
  \normalfont
}{\end{tcolorbox}}

\renewenvironment{remark}[1][]{%
  \refstepcounter{remark}%
  \begin{tcolorbox}[colback=remarkBg, colframe=remarkColor,
    sharp corners, boxrule=0.5pt, arc=0mm, boxsep=5pt,
    before=\par\bigskip, after=\par\bigskip,
    title=\textbf{Remarque \thechapter.\thesection.\theremark%
    \ifx&#1&\else\ \textit{(#1)}\fi%
    }, fonttitle=\bfseries\color{white}, colbacktitle=remarkColor]%
  \normalfont
}{\end{tcolorbox}}

\renewenvironment{example}[1][]{%
  \refstepcounter{example}%
  \begin{tcolorbox}[colback=exampleBg, colframe=exampleColor,
    sharp corners, boxrule=0.5pt, arc=0mm, boxsep=5pt,
    before=\par\bigskip, after=\par\bigskip,
    title=\textbf{Exemple \thechapter.\thesection.\theexample%
    \ifx&#1&\else\ \textit{(#1)}\fi%
    }, fonttitle=\bfseries\color{white}, colbacktitle=exampleColor]%
  \normalfont
}{\end{tcolorbox}}

\renewenvironment{exercise}[1][]{%
  \refstepcounter{exercise}%
  \begin{tcolorbox}[colback=exerciseBg, colframe=exerciseColor,
    sharp corners, boxrule=0.5pt, arc=0mm, boxsep=5pt,
    before=\par\bigskip, after=\par\bigskip,
    title=\textbf{Exercice \thechapter.\thesection.\theexercise%
    \ifx&#1&\else\ \textit{(#1)}\fi%
    }, fonttitle=\bfseries\color{white}, colbacktitle=exerciseColor]%
  \normalfont
}{\end{tcolorbox}}

\renewenvironment{consequence}[1][]{%
  \refstepcounter{consequence}%
  \begin{tcolorbox}[colback=consequenceBg, colframe=consequenceColor,
    sharp corners, boxrule=0.5pt, arc=0mm, boxsep=5pt,
    before=\par\bigskip, after=\par\bigskip,
    title=\textbf{Conséquence \thechapter.\thesection.\theconsequence%
    \ifx&#1&\else\ \textit{(#1)}\fi%
    }, fonttitle=\bfseries\color{white}, colbacktitle=consequenceColor]%
  \itshape
}{\end{tcolorbox}}

\renewenvironment{application}[1][]{%
  \refstepcounter{application}%
  \begin{tcolorbox}[colback=applicationBg, colframe=applicationColor,
    sharp corners, boxrule=0.5pt, arc=0mm, boxsep=5pt,
    before=\par\bigskip, after=\par\bigskip,
    title=\textbf{Application \thechapter.\thesection.\theapplication%
    \ifx&#1&\else\ \textit{(#1)}\fi%
    }, fonttitle=\bfseries\color{white}, colbacktitle=applicationColor]%
  \normalfont
}{\end{tcolorbox}}

\renewenvironment{proof}[1][]{%
  \begin{tcolorbox}[colback=proofBg, colframe=proofColor,
    sharp corners, boxrule=0.5pt, arc=0mm, boxsep=5pt,
    before=\par\bigskip, after=\par\bigskip,
    title=\textbf{Preuve%
    \ifx&#1&\else\ \textit{(#1)}\fi%
    }, fonttitle=\bfseries\color{white}, colbacktitle=proofColor]%
  \normalfont
}{\end{tcolorbox}}

\renewenvironment{importantbox}[1][]{%
  \begin{tcolorbox}[coursebox, title=, #1]%
  \centering\large
}{\end{tcolorbox}}

\newcommand{\doubleheadrightarrow}{%
    \tikz[overlay,remember picture,baseline]{%
        \draw[->,very thick,mainBlue] (0,0) -- (0.5em,0.5em);
        \draw[->,very thick,accentBlue] (0.25em,0.25em) -- (0.75em,0.75em);
    }%
}

% FIN DU PREAMBULE

\begin{document}

\begin{titlepage}
    \thispagestyle{empty}
    \centering
    \vspace*{5cm}

    {\Huge\bfseries Chapitre II : Nombres Complexes\par}
    {\Large \it Généralités et Géométrie\par}
    \vspace{1.5cm}

    {\Large Rédigé par Samy Youssoufine\par}
    \vspace{1cm}

    {\large \today\par}

    % L'image n'est pas une figure flottante, elle restera sur la même page
    \includegraphics[width=0.6\textwidth]{./assets/logo.png}\par

    \vspace{1cm}
    \vfill
\end{titlepage}

\cleardoublepage
\tableofcontents
\clearpage

% REPLACE \mathbb{R} WITH \mathbb{R}
% SAME WITH \mathbb{C}

\chapter{Généralités}

\begin{itemize}
    \item On munit $\mathbb{R}^2$ par les lois + et $\times$ définies par :\\
    $$\forall (a,b),(c,d) \in \mathbb{R}^2: \begin{cases} (a,b) + (c,d) = (a+c,b+d)\\ (a,b)\times(c,d)=(ac-bd,ad+bc) \end{cases}$$
    \item $(\mathbb{R}^2,+,\times)$ est un corps commutatif (voir Chap. 10) appelé le cors des nombres complexes, et noté $\mathbb{C}$.
    \item L'application $\Phi : \begin{cases} \mathbb{R} \rightarrow \mathbb{R}^2 \\ x \rightarrowtail (x,0)\end{cases}$ est injective (démonstration triviale).\\ On peut donc identifier $(x,0)$ par $x$, et on écrit $(x,0)$ ``$=$'' $x$.\\ \textit{N.B. que cette expression n'est pas mathématiquement correcte.}
    \item On pose $i=(0,1)$, alors : $i^2=(0,1)\times(0,1)=(-1,0)=-1$.
    \item On a : $\forall (x,y) \in \mathbb{R}^2: (x,y)=(x,0)+(0,y)=x+y(0,1)=x+iy$. 
    
    Donc $\boxed{\forall z\in \mathbb{C}, \exists!(a,b)\in \mathbb{R}^2 \text{ tel que } z=a+ib}$ (l'écriture algébrique de $z$).
    
    $$\boxed{
        \begin{cases}
    a \text{ est appelé la partie réelle de }z\text{, notée }\Re(z)\\
    b \text{ est appelé la partie imaginaire de }z\text{, notée }\Im(z)
    \end{cases}
    }$$
\end{itemize}

\begin{definition}[Conjugué d'un nombre complexe]
    Soit $z=a+ib \in \mathbb{C}$, on appelle le conjugué de $z$ le complexe noté $\boxed{\bar{z}=a-ib}$. On peut en déduire les résultats suivants :
    $$
    \boxed{
        \begin{cases}
            z+\bar{z}=2.\cdot\Re(z)\\
            z-\bar{z}=2\times i\times\Im{z}
        \end{cases}
    }
    $$ 
\end{definition}
\begin{definition}[Module d'un nombre complexe]
    Soit $z=a+ib \in \mathbb{C}$, le module de $z$ est le réel positif : $|z|=\sqrt{a^2+b^2} \geq 0$. On a donc : $z\cdot\bar{z}=|z|^2$
\end{definition}

\begin{property}\quad
\begin{enumerate}
    \item Soit $P$ le plan orthonormé direct $(O,\vec{e_1},\vec{e_2})$.\\
L'application suivante est bijective :
$$
\Psi : \begin{cases}
    \mathbb{C} \rightarrow P\\
    z=a+ib \rightarrowtail a\cdot\vec{e_1}+b\cdot\vec{e_2}
\end{cases}
$$

% INSERT GRAPH HERE

\item $\forall z \in \mathbb{C}: z=0 \iff |z|=0$
\item $\forall z_1,z_2 \in \mathbb{C}: |z_1\cdot z_2|=|z_1|\cdot |z_2|$
\item $\forall z,z' \in \mathbb{C}, |z+z'| \leq |z|+|z'|$\\Avec égalité si et seulement si : $(z=0 \text{ ou }\exists \alpha \geq 0:z'=\alpha z)$.
\item $\forall z \in \mathbb{C}^*:\frac{1}{z}=\frac{\bar{z}}{|z|^2}$
\end{enumerate}
\end{property}

\begin{proof}\quad
    \begin{enumerate}
        \item (À faire en tant que travail personnel).
        \item (Trivial).
        \item (Trivial).
        \item \textbf{Si $z=0$: }On a égalité.\\ \textbf{Si $z\not=0$:} On pose $\frac{z'}{z}=\alpha+i\beta$ avec $(\alpha,\beta) \in \mathbb{R}^2$.\\ $|1+\frac{z'}{z}|=|(1+\alpha)+i\beta|=\sqrt{(1+\alpha)^2+\beta^2}$\\ $\iff 1+|\frac{z'}{z}| = 1+\sqrt{\alpha^2+\beta^2}$\\$\iff\dots$\\$\iff \alpha \leq \sqrt{\alpha^2+\beta^2}$ (toujours vrai)\\$\iff|1+\frac{z'}{z}|\leq1+|\frac{z'}{z}|$\\$\iff|z+z'|\leq|z|+|z'|$\\Avec égalité si et seulement si : $\alpha=\sqrt{\alpha^2+\beta^2}\\ \iff\dots\\ \iff z'=\alpha z / \alpha \geq 0$
    \end{enumerate}
\end{proof}

\begin{remark}
    Soit $z \in \mathbb{C}$.\\
    \begin{enumerate}
        \item $z\in \mathbb{R} \iff \Im(z)=0 \iff \bar{z}=z$.
        \item $z\in i \mathbb{R} \iff \Re(z)=0 \iff \bar{z}=-z$ ($z$ est dit \textit{imaginaire pur}).
        \item $\begin{cases}
            |\Re(z)|\leq|z|\\
            |\Im(z)|\leq|z|\\
        \end{cases}$
    \end{enumerate}
\end{remark}

\begin{property}
    Soient $z,z'\in \mathbb{C}$.
    \begin{enumerate}
        \item $\bar{z+z'}=\bar{z'}+\bar{z}$
        \item $\bar{z.z'}|=\bar{z'}.\bar{z}$
        \item $|z+z'|^2=|z|^2+2.\Re(z.\bar{z'})+|z'|^2$
        \item $|z-z'|^2=|z|^2-2.\Re(z.\bar{z'})+|z'|^2$
        \item $|z+z'|^2+|z-z'|^2=2(|z|^2+|z'|^2)$ \textit{(Identité du parallélogramme)}.
        \item $|z+z'|^2-|z-z'|^2=4\Re({z.\bar{z'}})$
    \end{enumerate}
\end{property}

\begin{exercise}
Soit $z_1,z_n \in \mathbb{C}$. Montrer que : $|\sum_{k=1}^nz_k|\leq\sum_{k=1}^n|z_k|$ avec égalité si et seulement si : $(\exists u \in \mathbb{C}, \exists a_1,\dots,a_n \in \mathbb{R} : \forall k\in {1,\dots,n}: z_k=a_k.u)$

\textbf{Solution : }\\
\textit{Démonstration de l'inégalité (par récurrence)}\\

\textit{Démonstration de l'égalité :} (sens direct et indirect)\\
À refaire en tant qu'exercice.
\end{exercise}

\begin{definition}[Argument d'un complexe (non-nul)]
    Soit $z \in \mathbb{C}^*$, on a : $|\frac{z}{|z|}|=1$.
    On pose $\frac{z}{|z|}=x+iy$ pour avoir $\sqrt{x^2+y^2}=1$.\\
    Donc $\exists\theta\in\mathbb{R}\begin{cases}x=\cos\theta\\y=\sin\theta\end{cases}$\\
    $\theta$ est appelé un argument de $z$, et on le note par $\arg(z)$ (Les autres arguments de $z$ sont $\theta +2k\pi/k\in\mathbb{Z}$).\\
    Si $\theta \in [-\pi,\pi]$, alors $\theta$ est dit argument principal de $z$.\\
    Donc on peut écrire :
    $$\forall z\in\mathbb{C}:\exists\theta\in\mathbb{R}:z=|z|(\cos\theta+i\sin\theta)=|z|e^{i\theta}$$ (ce qui est aussi vrai pour $z=0$ (pour tout $\theta \in \mathbb{R}$))
\end{definition}

\begin{property}\quad
    \begin{outline}[enumerate]
        \1 $\forall z\in\mathbb{C}^*$:
            \2 $\arg(z)\equiv0[2\pi]\iff z\in\mathbb{R}^*_+$
            \2 $\arg(z)\equiv0[\pi]\iff z\in\mathbb{R}^*_+$
            \2 $\arg(z)\equiv\frac{\pi}{2}[\pi]\iff z\in i\mathbb{R}^*$
        \1 Soient $z,z'\in\mathbb{C}^*$:
            \2 $\arg(z.z')\equiv\arg(z)+\arg(z')[2\pi]$
            \2 $\forall n\in\mathbb{Z}:\arg(z^n)\equiv n.\arg(z)[2\pi]$
            \2 $\arg(\frac{z}{z'})\equiv\arg(z)-\arg(z')[2\pi]$
    \end{outline}
\end{property}

\begin{definition}[L'ensemble $\mathbb{U}$]
    On note $\mathbb{U}=\{z\in\mathbb{C}/|z|=1\}={e^{i\theta}/\theta\in\mathbb{R}}$ l'ensemble des nombres complexes de module 1. Il s'agit du cercle unité de centre $O$ et de rayon 1.
\end{definition}

\begin{property}[Formule d'Euler]
    $\forall \theta \in \mathbb{R}: \begin{cases}\cos\theta=\frac{e^{i\theta}+e^{-i\theta}}{2} \\ \sin\theta=\frac{e^{i\theta}-e^{-i\theta}}{2i}\end{cases}$
    $\forall \theta \in \mathbb{R}, \forall n\in\mathbb{Z}: e^{i n \theta}=(\cos\theta+i\sin\theta)^n=\cos(n\theta)+i\sin(n\theta)$
\end{property}

\begin{remark}
    $$\begin{cases}\forall a,b \in \mathbb{R}: e^{ia}+e^{ib}=2\cos(\frac{a-b}{2})e^{i\frac{a+b}{2}}\\
    \forall a,b \in \mathbb{R}: e^{ia}-e^{ib}=-2i\sin(\frac{a-b}{2})e^{i\frac{a+b}{2}}\end{cases}$$

    $$b=0 \implies \begin{cases}\forall a \in \mathbb{R}: e^{ia}+1=2\cos(\frac{a}{2})e^{i\frac{a}{2}}\\
    \forall a \in \mathbb{R}: e^{ia}-1=-2i\sin(\frac{a}{2})e^{i\frac{a}{2}}\end{cases}$$
\end{remark}

\begin{example}
    Calculer le module de $z=(\sqrt{2}+\sqrt{3})+i(1+\sqrt{2})$.
    \textbf{Solution : }\\
    $z=(\sqrt{2}+i\sqrt{2})+(\sqrt{3}+i)$\\
    $\implies z=2(\frac{\sqrt{2}}{2}+i\frac{\sqrt{2}}{2})+2(\frac{\sqrt{3}}{2}+i\frac{1}{2})$\\
    $\implies z=2e^{i\frac{\pi}{4}}+2e^{i\frac{\pi}{6}}$\\
    $\implies z=2\cos(\frac{\pi}{12})e^{i\frac{5\pi}{12}}$\\
    $\implies |z|=2\cos(\frac{\pi}{12})$ (sachant que $\cos(\frac{\pi}{12})\geq0$)
\end{example}

\begin{property}\quad
    \begin{outline}[enumerate]
    \1 $\forall\theta\in\mathbb{R}:|e^{i\theta}|=1$
    \1 $\forall\theta\in\mathbb{R}:e^{i\theta}=1\iff \theta \in 2\pi\mathbb{Z}$
    \1 $\forall \theta_1,\theta_2\in\mathbb{R}:e^{i\theta_1}.e^{i\theta_2}=e^{i(\theta_1+\theta_2)}$
    \1 L'application $\begin{cases}\mathbb{R}\rightarrow\mathbb{U}\\\theta\rightarrowtail e^{i\theta}\end{cases}$ est surjective.
    \end{outline}
\end{property}

\begin{consequence}
    \begin{outline}[enumerate]
        \1 $\forall\theta_1,\dots,\theta_n\in\mathbb{R}:\prod_{k=1}^n e^{i\theta_k}=e^{i\left(\sum_{k=1}^n\theta_k\right)}$
        \1 $\forall\theta\in\mathbb{R},\forall n\in\mathbb{Z}: (e^{i\theta})^n=e^{in\theta}$
        \1 $\forall \theta\in\mathbb{R},\forall n\in\mathbb{Z}: \cos(n\theta)+i\sin(n\theta)=(\cos\theta+i\sin\theta)^n$
    \end{outline}
\end{consequence}

\begin{definition}[Racine n-ième d'un complèxe]
    Soit $z\in\mathbb{C}^*$ et $n\in\mathbb{N}^*$. On appelle racine n-ième de $z$ tout complexe $\omega$ tel que $\omega^n=z$.\\
    Si $z=|z|e^{i\theta}$, alors les racines n-ièmes de $z$ sont les complexes : $$\omega_k=|z|^{\frac{1}{n}}e^{i\frac{\theta+2k\pi}{n}} \text{ pour } k=0,1,\dots,n-1$$
    Ces racines sont les sommets d'un polygone régulier à n côtés inscrit dans le cercle de centre $O$ et de rayon $|z|^{\frac{1}{n}}$.
\end{definition}
\begin{theorem}
    Soient $a=re^{i\theta}\in\mathbb{C}^*$ et $n\in\mathbb{N}^*$. Alors l'équation $z^n=a$ admet $n$ solutions distinctes dans $\mathbb{C}$ données par : $$z_k=r^{\frac{1}{n}}e^{i\frac{\theta+2k\pi}{n}} \text{ pour } k\in\{0,1,\dots,n-1\}$$
\end{theorem}
\begin{proof}
    Soit $z\in\mathbb{C}^*$ tel que $z^n=a$ et $z=\rho e^{i\alpha}$.\\
    $\implies z^n=\rho^n e^{in\alpha}=re^{i\theta}$\\
    $\implies \begin{cases}\rho^n=r\\n\alpha\equiv\theta[2\pi]\end{cases}$\\
    $\implies \begin{cases}\rho=r^{\frac{1}{n}}\\\alpha\equiv\frac{\theta}{n}+2\frac{k\pi}{n}\end{cases}$ avec $k\in\mathbb{Z}$\\
    Par la division euclidienne de $s$ par $n$, on a :
    $$\exists (q,r)\in\mathbb{Z}\times\{0,1,\dots,n-1\}:k=qn+r$$
    Donc les solutions sont : $$z_k=r^{\frac{1}{n}}e^{i\left(\frac{\theta}{n}+2\frac{k\pi}{n}\right)} \text{ pour } k\in\{0,1,\dots,n-1\}$$
    On a montré qu'il y a au plus $n$ solutions.\\
    Montrons qu'elles sont toutes distinctes :\\
    Soient $k,k'\in\{0,1,\dots,n-1\}$ tels que $z_k=z_{k'}$. (avec $k> k'$)\\
    $\implies r^{\frac{1}{n}}e^{i\left(\frac{\theta}{n}+2\frac{k\pi}{n}\right)}=r^{\frac{1}{n}}e^{i\left(\frac{\theta}{n}+2\frac{k'\pi}{n}\right)}$\\
    $\implies e^{i2\frac{k\pi}{n}}=e^{i2\frac{k'\pi}{n}}$\\
    $\implies 2\frac{k\pi}{n}\equiv2\frac{k'\pi}{n}[2\pi]$\\
    $\implies k\equiv k'[n]$\\
    $\implies \exists p\in\mathbb{N}^* : k= k'+pn$\\
    $\implies k-k'\geq n$ (contradiction) CQFD.

\end{proof}

\begin{example}\quad
    \begin{outline}[enumerate]
        \1 Les racines carrées de $a=re^{i\theta}$ sont $z_1=\sqrt{r}e^{i\frac{\theta}{2}}$ et $z_2=\sqrt{r}e^{i\left(\frac{\theta}{2}+\pi\right)}$.
        \1 Les racines cubiques de $a=re^{i\theta}$ sont $z_k=r^{\frac{1}{3}}e^{i\left(\frac{\theta}{3}+2\frac{k\pi}{3}\right)}$ pour $k\in\{0,1,2\}$, soit $\begin{cases}z_0=r^{\frac{1}{3}}e^{i\frac{\theta}{3}}\\z_1=r^{\frac{1}{3}}e^{i\left(\frac{\theta}{3}+\frac{2\pi}{3}\right)}\\z_2=r^{\frac{1}{3}}e^{i\left(\frac{\theta}{3}+\frac{4\pi}{3}\right)}\end{cases}$
        \\On peut remplacer $e^{i\frac{2\pi}{3}}$ par $j$ pour obtenir $\begin{cases}z_0=r^{\frac{1}{3}}e^{i\frac{\theta}{3}}\\z_1=j.r^{\frac{1}{3}}e^{i\frac{\theta}{3}}\\z_2=j^2.r^{\frac{1}{3}}e^{i\frac{\theta}{3}}\end{cases}$ avec $j=e^{i\frac{2\pi}{3}}=-\frac{1}{2}+i\frac{\sqrt{3}}{2}$ et $j^2=e^{i\frac{4\pi}{3}}=-\frac{1}{2}-i\frac{\sqrt{3}}{2}$. On remarque que $1+j+j^2=0$.
    \end{outline}
\end{example}

\begin{consequence}
    Soit $n\in\mathbb{N}^*$. L'ensemble des racines n-ièmes de l'unité est : $\mathbb{U}_n=\{e^{i\frac{2k\pi}{n}}/k=0,1,\dots,n-1\}$.
    Il forme l'ensemble $\mathbb{U}_n=\{z\in\mathbb{C}/z^n=1\}$ des solutions de l'équation $z^n=1$, qu'on peut aussi noter $\mathbb{U}_n=\{\omega^k\in\mathbb{C}/k\in\{0,1,\dots,n-1\}\}$ et $\omega = e^{i\frac{2\pi}{n}}$.
\end{consequence}

\begin{remark}\quad
    \begin{outline}[enumerate]
        \1 $X^n-1=\prod_{k=0}^{n-1}(X-\omega^k)$
        \1 $\sum_{k=0}^{n-1}\omega^k=0$ (somme des racines n-ièmes de l'unité) (Voir relation entre coefficients et racines d'un polynôme scindé dans le chapitre 12).
    \end{outline}
\end{remark}

\begin{definition}[Exponentielle complexe]
    Soit $z=a+ib\in\mathbb{C}$. On définit l'exponentielle complexe par : $e^z=e^a(\cos b+i\sin b)$.
    On peut aussi l'écrire sous la forme : $e^z=e^{a+ib}=e^a e^{ib}$.
\end{definition}

\begin{property}\quad
\begin{outline}[enumerate]
    \1 $\forall z\in\mathbb{C}:e^{z}\neq0$
    \1 $\forall z \in\mathbb{C}: |e^z|=e^{\Re(z)}.\cos(\Im(z))$
    \1 $\forall z \in \mathbb{C}:\Im(z)=e^{\Re(z)}.\sin(\Im(z))$
    \1 $\forall z,z' \in \mathbb{C}: e^{z+z'}=e^z.e^{z'}$
    \1 $\forall z \in \mathbb{C}: e^{-z}=\frac{1}{e^z}$
    \1 $\forall z \in \mathbb{C}:\overline{e^z}=e^{\bar{z}}$
    \1 $\forall z \in \mathbb{C}:e^z=1\iff (\Re(z)=0 \text{ et } b \equiv 0[2\pi]) \iff z\in2i\pi\mathbb{Z}$
    \1 L'application $\begin{cases}\mathbb{C}\rightarrow\mathbb{C}^*\\z\rightarrowtail e^z\end{cases}$ est surjective.
\end{outline}
\end{property}

\begin{figure}[htbp]
    \centering
    \begin{minipage}{0.45\textwidth}
        \centering
        \begin{tikzpicture}
        \begin{axis}[
            title={Partie réelle de $e^z$},
            xlabel=$x$,
            ylabel=$y$,
            zlabel={Re($e^z$)},
            view={40}{30},
            domain=-2:2,
            y domain=-pi:pi,
            samples=15,
            colormap/viridis,
            width=\linewidth,
            height=5cm,
        ]
            \addplot3[surf] {exp(x)*cos(deg(y))};
        \end{axis}
        \end{tikzpicture}
        \caption{Surface représentant la partie réelle de $e^z$ où $z=x+iy$}
    \end{minipage}\hfill
    \begin{minipage}{0.45\textwidth}
        \centering
        \begin{tikzpicture}
        \begin{axis}[
            title={Partie imaginaire de $e^z$},
            xlabel=$x$,
            ylabel=$y$,
            zlabel={Im($e^z$)},
            view={40}{30},
            domain=-2:2,
            y domain=-pi:pi,
            samples=15,
            colormap/viridis,
            width=\linewidth,
            height=5cm,
        ]
            \addplot3[surf] {exp(x)*sin(deg(y))};
        \end{axis}
        \end{tikzpicture}
        \caption{Surface représentant la partie imaginaire de $e^z$ où $z=x+iy$}
    \end{minipage}
\end{figure}

\chapter{Nombres complexes et géométrie}
\section{Généralités}
Soit $P$ le plan muni d'un repère orthonormé direct $(O,\vec{e_1},\vec{e_2})$.\\
Soient $A(a)$,$B(b)$,$C(c),D(d)\in P$ et $z_a,z_b,z_c,z_d\in\mathbb{C}$ leurs affixes respectives, avec $z_a\neq z_b$ et $z_c\neq z_d$.

\begin{property}\quad
    \begin{outline}[enumerate]
        \1 $AB=|z_b-z_a|$
        \1 $\vec{AB}=\overrightarrow{z_a z_b}=z_b-z_a$
        \1 $I$ milieu de $[AB]\iff z_i=\frac{z_a+z_b}{2}$
        \1 $G$ barycentre de $(A,a),(B,b),(C,c)\iff z_g=\frac{az_a+bz_b+cz_c}{a+b+c}$
        \1 $A,B,C$ alignés $\iff \exists\lambda\in\mathbb{R}: z_c-z_a=\lambda(z_b-z_a)$
        \1 $A,B,C$ alignés $\iff \frac{z_c-z_a}{z_b-z_a}\in\mathbb{R}$
        \1 $\widehat{(\vec{u},\vec{v})}\equiv\arg(\frac{z_v}{z_u})[\pi]$ (angle orienté entre deux vecteurs non nuls)
        \1 $\widehat{(AB,AC)}\equiv\arg(\frac{z_c-z_a}{z_b-z_a})[\pi]$
        \1 $(AB) \parallel (CD) \iff \frac{z_b-z_a}{z_d-z_c}\in\mathbb{R}$
        \1 $(AB)\perp(CD)\iff \frac{z_b-z_a}{z_d-z_c}\in i\mathbb{R}^*$
        \1 $ABC$ est un triangle rectangle \textbf{direct} en A $(\widehat{\vec{AB},\vec{AC}})\equiv\frac{\pi}{2}[2\pi]\iff \arg(\frac{z_c-z_a}{z_b-z_a})\equiv\frac{\pi}{2}[2\pi]\iff \frac{z_c-z_a}{z_b-z_a}\in i\mathbb{R}^*_+$
    \end{outline}
\end{property}

\begin{example}
    Soient $A(2+i)$,$B(4+2i)$ deux points du plan $P$. Déterminer $C(c)$ tel que $ABC$ soit un triangle et isocèle rectangle en $A$.
    \textbf{Solution : }\\
    $ABC$ isocèle rectangle en $A \iff \begin{cases}AB=AC\\\widehat{(AB,AC)}\equiv\frac{\pi}{2}[2\pi]\end{cases}$\\
    $\iff \begin{cases}|b-a|=|c-a|\\\arg(\frac{c-a}{b-a})\equiv\frac{\pi}{2}[2\pi]\end{cases}$\\
    $\iff \begin{cases}|\frac{c-a}{b-a}|=1\\\frac{c-a}{b-a}=i\end{cases}$\\
    $\iff c-a=i(b-a)$\\
    $\iff c=ia+(1-i)b$\\
    $\iff \dots$\\
    $\iff c=1+3i$
\end{example}

\begin{remark}
    Soit $D(A(a),\vec{u}(\alpha))$ avec $\begin{cases}a\in\mathbb{C}\\ \alpha \in\mathbb{C}^*\end{cases}$.

    $M(z)\in D \iff \vec{AM} \text{ et } \vec{u} \text{ sont liés.}\iff \exists\lambda\in\mathbb{R}:\vec{AM}=\lambda\vec{u}\iff \frac{z-a}{\alpha}\in\mathbb{R}$\\
    $\iff \overline{\frac{z-a}{\alpha}}=\frac{z-a}{\alpha}\iff (z-a)\bar{\alpha}=\overline{(z-a)}\alpha\iff \bar{\alpha}z-\bar{\alpha}a=\alpha\bar{z}-\alpha\bar{a}\iff \bar{\alpha}z-\alpha\bar{z}=\bar{\alpha}a-\alpha\bar{a}$\\

    Il s'agit d'une équation complexe de la droite $(D)$.
\end{remark}

% représentation graphique

\begin{figure}[htbp]
    \centering
    \begin{tikzpicture}
        \begin{axis}[
            axis lines=middle,
            xlabel=$\Re$,
            ylabel=$\Im$,
            xmin=-3, xmax=3,
            ymin=-3, ymax=3,
            xtick={-2,-1,0,1,2},
            ytick={-2,-1,0,1,2},
            grid=both,
            width=10cm,
            height=10cm,
            domain=-3:3,
            samples=100,
        ]
            % Draw the line (D)
            \addplot[blue, thick] {x + 1}; % Example line equation: y = x + 1
            \node at (axis cs:2,3) [anchor=north west] {$(D)$};
        \end{axis}
    \end{tikzpicture}
    \caption{Représentation graphique de la droite $(D)$ dans le plan complexe.}
\end{figure}

\begin{example}
    $(D):(1+i)z - (1-i)\bar{z}-2i=0$ est une équation complexe de la droite passant par $A(1)$ et de vecteur directeur $\vec{u}(1-i)$.
\end{example}

\section{Similitudes directes}

\begin{definition}
    On appelle une similitude directe toute application de la forme $S:P\rightarrow P$ telle que :
    $$\exists a,b\in\mathbb{C}:\forall z\in\mathbb{C}:S(a,b):\begin{cases}\mathbb{C}\rightarrow \mathbb{C}\\z\rightarrowtail az+b \text{ avec }\end{cases} a\in\mathbb{C}^*,b\in\mathbb{C}$$
\end{definition}

\subsection{Translation}
\begin{definition}
    On appelle translation toute application de la forme $T_{\vec{u}(b)}:\mathbb{C}\rightarrow \mathbb{C}$ telle que :
    $$\forall z\in\mathbb{C}:T_{\vec{u}(b)}:\begin{cases}\mathbb{C}\rightarrow \mathbb{C}\\z\rightarrowtail z+b \text{ avec } b\in\mathbb{C}\end{cases}$$
    \centering(Il s'agit du cas où $a=1$ et $b\in\mathbb{C}^*$)
\end{definition}

% représentation graphique

\begin{figure}[htbp]
    \centering
    \begin{tikzpicture}[scale=0.75]
        \begin{axis}[
            axis lines=middle,
            xlabel=$\Re$,
            ylabel=$\Im$,
            xmin=-1, xmax=4,
            ymin=-0.5, ymax=4,
            xtick={-3,-2,-1,0,1,2,3},
            ytick={-3,-2,-1,0,1,2,3},
            grid=both,
            width=12cm,
            height=8cm,
            domain=-4:4,
            samples=100,
        ]
            % Draw the original point A
            \addplot[red, only marks] coordinates {(1,1)};
            \node at (axis cs:1,1) [anchor=north west] {$A(1+i)$};

            % Draw the translated point A'
            \addplot[blue, only marks] coordinates {(2,2)};
            \node at (axis cs:2,2) [anchor=north west] {$A'(2+2i)$};

            % Draw the translation vector
            \draw[->, thick] (axis cs:1,1) -- (axis cs:2,2) node[midway, above] {$\vec{u}(1+i)$};
        \end{axis}
    \end{tikzpicture}
    \caption{Représentation graphique de la translation $T_{\vec{u}(1+i)}$.}
\end{figure}

\subsection{Homothétie}
\begin{definition}
    On appelle homothétie toute application de la forme $H_{(\Omega(\omega),a)}:\mathbb{C}\rightarrow \mathbb{C}$ telle que :
    $$\forall z\in\mathbb{C}:H_{(\omega)}:\begin{cases}\mathbb{C}\rightarrow \mathbb{C}\\z\rightarrowtail az+b \iff z'-\omega=a(z-\omega) \text{ avec } \omega=\frac{b}{1-a}\end{cases}$$
    On dit que c'est une homothétie de centre $\Omega(\omega)$ et de rapport $a$, avec $\omega = \frac{b}{1-a}$.\\
    \centering(Il s'agit du cas où $b\in\mathbb{C}^*$ et $a\in\mathbb{R}^*-\{1\}$)
\end{definition}

\begin{figure}[htbp]
    \centering
    \begin{tikzpicture}[scale=0.7]
        \begin{axis}[
            axis lines=middle,
            xlabel=$\Re$,
            ylabel=$\Im$,
            xmin=-1, xmax=4,
            ymin=-1, ymax=4.5,
            xtick={-0,1,2,3,4},
            ytick={-0,1,2,3,4},
            grid=both,
            width=12cm,
            height=10cm,
            enlargelimits=false,
            clip=false,
        ]
            % Parameters
            \def\omegaX{0.5}
            \def\omegaY{-0.5}
            \def\a{1.8}

            % Original triangle vertices
            \coordinate (A) at (axis cs:1,1);
            \coordinate (B) at (axis cs:2,0.5);
            \coordinate (C) at (axis cs:0.2,1.7);
            % Center of homothety
            \coordinate (O) at (axis cs:\omegaX,\omegaY);

            % Image vertices: A' = O + a*(A-O)
            \coordinate (Ap) at (axis cs:{\omegaX + \a*(1-\omegaX)},{\omegaY + \a*(1-\omegaY)});
            \coordinate (Bp) at (axis cs:{\omegaX + \a*(2-\omegaX)},{\omegaY + \a*(0.5-\omegaY)});
            \coordinate (Cp) at (axis cs:{\omegaX + \a*(0.2-\omegaX)},{\omegaY + \a*(1.7-\omegaY)});

            % Filled shapes
            \draw[fill=red!20, draw=red!70] (A) -- (B) -- (C) -- cycle;
            \draw[fill=blue!20, draw=blue!70] (Ap) -- (Bp) -- (Cp) -- cycle;

            % Center marker
            \draw[green!60!black,fill=green!60!black] (O) circle (1.8pt);
            \node[green!30!black, anchor=north west] at (axis cs:\omegaX,\omegaY) {$\Omega(\omega)$};

            % Points markers and labels
            \draw[red, fill=red] (A) circle (1.6pt) node[anchor=south west] {$A(1+i)$};
            \draw[red, fill=red] (B) circle (1.6pt) node[anchor=north west] {$B(2+0.5i)$};
            \draw[red, fill=red] (C) circle (1.6pt) node[anchor=south east] {$C(0.2+1.7i)$};

            \draw[blue, fill=blue] (Ap) circle (1.6pt) node[anchor=south west] {$A'$};
            \draw[blue, fill=blue] (Bp) circle (1.6pt) node[anchor=south west] {$B'$};
            \draw[blue, fill=blue] (Cp) circle (1.6pt) node[anchor=north east] {$C'$};

            % Rays from center and dashed alignment lines
            \draw[->, thick, gray!70] (O) -- (A) node[midway, left] {};
            \draw[->, thick, gray!70] (O) -- (Ap) node[midway, right] {};
            \draw[dashed, gray] (A) -- (Ap);
            \draw[dashed, gray] (B) -- (Bp);
            \draw[dashed, gray] (C) -- (Cp);

            % Arrows showing scaling direction
            \draw[->, very thick, green!50!black] (O) -- (Ap) node[pos=0.65, above right] {rapport $a=\ \a$};

            % Legend box
            \node[draw, fill=white, opacity=0.9] at (axis cs:3.6,4.2) {
                \begin{minipage}{5.2cm}
                    \small
                    \textcolor{red!70!black}{\textbullet} triangle initial \\
                    \textcolor{blue!70!black}{\textbullet} triangle image par $H_{\Omega,a}$ \\
                    $\Omega$ : centre, segments en pointillés : correspondances colinéaires \\
                    flèche verte : direction et sens de l'homothétie (facteur $a$)
                \end{minipage}
            };
        \end{axis}
    \end{tikzpicture}
    \caption{Homothétie de centre $\Omega(\omega)$ et rapport $a$ — image d'un triangle.}
\end{figure}



\begin{property}[Composée de deux homothéties]
    Soient $H_{(\Omega_1(\omega_1),a_1)}$ et $H_{(\Omega_2(\omega_2),a_2)}$ deux homothéties. On pose $H_1:z'=a_1z+b_1$ et $H_2:z'=a_2z+b_2$. Alors :
    \begin{enumerate}
        \item Si $a_1.a_2=1$, alors $H_{(\Omega_1,a_1)}\circ H_{(\Omega_2,a_2)}$ est une translation de vecteur $\vec{u}(a_1b_2+b_1)$.
        \item Si $a_1.a_2\neq1$, alors $H_{(\Omega_1,a_1)}\circ H_{(\Omega_2,a_2)}$ est une homothétie de centre $\Omega(\frac{a_1b_2+b_1}{1-a_1.a_2})$ et de rapport $a=a_1.a_2$.
    \end{enumerate}

\end{property}

\begin{figure}[htbp]
    \centering
     \begin{tikzpicture}[
     scale=1.5,
     point/.style={circle,fill=black,inner sep=1.5pt},
     vector/.style={-{Stealth[length=3mm]},thick},
     annotation/.style={font=\small}
]

% drawing a background grid and axes (repère)
\begin{scope}
     % light grid
     \draw[step=0.5,gray!25,very thin] (-1.2,-1.2) grid (4.4,3.6);

     % axes
     \draw[->,thick] (-1.2,0) -- (4.4,0) node[right] {$\Re$};
     \draw[->,thick] (0,-1.2) -- (0,3.6) node[above] {$\Im$};

     % x ticks and labels
     \foreach \x in {-1,0,1,2,3,4}
          \draw (\x,0.06) -- (\x,-0.06) node[below,yshift=-1pt] {\small $\x$};

     % y ticks and labels
     \foreach \y in {-1,0,1,2,3}
          \draw (0.06,\y) -- (-0.06,\y) node[left,xshift=-2pt] {\small $\y$};
\end{scope}

% Centres des homothéties
\coordinate (O1) at (0,0);
\coordinate (O2) at (3,0);

% Points initiaux
\coordinate (A) at (1,1);
\coordinate (B) at (0.5,2);
\coordinate (C) at (2,1.5);

% Première homothétie : h1 de centre O1 et rapport k1=1.5
\def\kone{1.5}
\coordinate (A1) at ($(O1) + \kone*(A) - \kone*(O1)$);
\coordinate (B1) at ($(O1) + \kone*(B) - \kone*(O1)$);
\coordinate (C1) at ($(O1) + \kone*(C) - \kone*(O1)$);

% Deuxième homothétie : h2 de centre O2 et rapport k2=0.7
\def\ktwo{0.7}
\coordinate (A2) at ($(O2) + \ktwo*(A1) - \ktwo*(O2)$);
\coordinate (B2) at ($(O2) + \ktwo*(B1) - \ktwo*(O2)$);
\coordinate (C2) at ($(O2) + \ktwo*(C1) - \ktwo*(O2)$);

% Triangle initial (bleu)
\draw[blue,thick] (A) -- (B) -- (C) -- cycle;
\node[point,blue] at (A) {};
\node[point,blue] at (B) {};
\node[point,blue] at (C) {};

% Triangle après première homothétie (rouge)
\draw[red,thick,dashed] (A1) -- (B1) -- (C1) -- cycle;
\node[point,red] at (A1) {};
\node[point,red] at (B1) {};
\node[point,red] at (C1) {};

% Triangle après deuxième homothétie (vert)
\draw[green!70!black,thick,dotted] (A2) -- (B2) -- (C2) -- cycle;
\node[point,green!70!black] at (A2) {};
\node[point,green!70!black] at (B2) {};
\node[point,green!70!black] at (C2) {};

% Centres des homothéties
\node[point,label=below left:$O_1$] at (O1) {};
\node[point,label=below right:$O_2$] at (O2) {};

% Flèches illustrant les homothéties
\draw[vector,red] (A) -- node[above left] {$h_1$} (A1);
\draw[vector,green!70!black] (A1) -- node[above right] {$h_2$} (A2);
\draw[vector,blue,dashed] (A) -- node[below] {$h_2 \circ h_1$} (A2);

% Légende
\begin{scope}[shift={(1.5,-2)}]
     \draw[blue,thick] (0,0) -- (0.5,0) node[right,annotation] {Triangle initial $ABC$};
     \draw[red,thick,dashed] (0,-0.3) -- (0.5,-0.3) node[right,annotation] {Image par $h_1$ ($k_1=1,5$)};
     \draw[green!70!black,thick,dotted] (0,-0.6) -- (0.5,-0.6) node[right,annotation] {Image par $h_2$ ($k_2=0,7$)};
     \draw[blue,dashed,vector] (0,-0.9) -- (0.5,-0.9) node[right,annotation] {Composée $h_2 \circ h_1$ ($k=1,05$)};
\end{scope}

% Annotations mathématiques
\node[annotation,blue] at (1.5,2.2) {$h_2 \circ h_1$ est une homothétie de rapport $k_1 \times k_2 = 1,05$};

\end{tikzpicture}

    \caption{Composition de deux homothéties : \(H_{\Omega_2,a_2}\circ H_{\Omega_1,a_1}\). Les lignes pointillées montrent les droites de correspondance (colinéarité) depuis chaque centre.}
\end{figure}



\subsection{Rotation}
\begin{definition}
    On appelle rotation toute application de la forme $R_{(\Omega(\omega),\theta)}:\mathbb{C}\rightarrow \mathbb{C}$ telle que :
    $$\forall z\in\mathbb{C}:R_{(\omega,\theta)}:\begin{cases}\mathbb{C}\rightarrow \mathbb{C}\\z\rightarrowtail e^{i\theta}(z-\omega)+\omega \text{ avec } \omega\in\mathbb{C} \text{ et } \theta\in\mathbb{R}\end{cases}$$
    On dit que c'est une rotation de centre $\Omega(\omega)$ et d'angle $\theta$.\\
    \centering(Il s'agit du cas où $b\in\mathbb{C}$ et $a\in\mathbb{C}-\mathbb{R}$)
\end{definition}

\begin{figure}[htbp]
    \centering
    \begin{tikzpicture}
        \begin{axis}[
            axis lines=middle,
            xlabel=$\Re$,
            ylabel=$\Im$,
            xmin=-1, xmax=5,
            ymin=-1, ymax=5,
            xtick={-0,1,2,3,4,5},
            ytick={-0,1,2,3,4,5},
            grid=both,
            width=12cm,
            height=10cm,
            enlargelimits=false,
            clip=false,
        ]
            % Parameters
            \def\omegaX{2}
            \def\omegaY{2}
            \def\theta{60} % Angle in degrees

            % Original triangle vertices
            \coordinate (A) at (axis cs:3,2);
            \coordinate (B) at (axis cs:4,3);
            \coordinate (C) at (axis cs:2,4);
            % Center of rotation
            \coordinate (O) at (axis cs:\omegaX,\omegaY);

            % Image vertices: A' = O + R*(A-O)
            \pgfmathsetmacro{\cosTheta}{cos(\theta)}
            \pgfmathsetmacro{\sinTheta}{sin(\theta)}
            \coordinate (Ap) at (axis cs:{\omegaX + (\cosTheta*(3-\omegaX) - \sinTheta*(2-\omegaY))},{\omegaY + (\sinTheta*(3-\omegaX) + \cosTheta*(2-\omegaY))});
            \coordinate (Bp) at (axis cs:{\omegaX + (\cosTheta*(4-\omegaX) - \sinTheta*(3-\omegaY))},{\omegaY + (\sinTheta*(4-\omegaX) + \cosTheta*(3-\omegaY))});
            \coordinate (Cp) at (axis cs:{\omegaX + (\cosTheta*(2-\omegaX) - \sinTheta*(4-\omegaY))},{\omegaY + (\sinTheta*(2-\omegaX) + \cosTheta*(4-\omegaY))});

            % Filled shapes
            \draw[fill=red!20, draw=red!70] (A) -- (B) -- (C) -- cycle;
            \draw[fill=blue!20, draw=blue!70] (Ap) -- (Bp) -- (Cp) -- cycle;
            % Center marker
            \draw[green!60!black,fill=green!60!black] (O) circle (1.8pt);
            \node[green!30!black, anchor=north west] at (axis cs:\omegaX,\omegaY) {$\Omega(\omega)$};
            % Points markers and labels
            \draw[red, fill=red] (A) circle (1.6pt) node[anchor=south west] {$A(3+2i)$};
            \draw[red, fill=red] (B) circle (1.6pt) node[anchor=south west] {$B(4+3i)$};
            \draw[red, fill=red] (C) circle (1.6pt) node[anchor=north east] {$C(2+4i)$};
            \draw[blue, fill=blue] (Ap) circle (1.6pt) node[anchor=south west] {$A'$};
            \draw[blue, fill=blue] (Bp) circle (1.6pt) node[anchor=south west] {$B'$};
            \draw[blue, fill=blue] (Cp) circle (1.6pt) node[anchor=north east] {$C'$};
            % Rays from center and dashed alignment lines
            \draw[dashed] (O) -- (A);
            \draw[dashed] (O) -- (B);
            \draw[dashed] (O) -- (C);
            \draw[dashed] (O) -- (Ap);
            \draw[dashed] (O) -- (Bp);
            \draw[dashed] (O) -- (Cp);
            % Angle arc
            \draw[->, thick, orange!70!black] (axis cs:2.5,2) arc[start angle=0,end angle=\theta,radius=0.5];
            \node[orange!70!black] at (axis cs:2.8,2.3) {$\theta$};
            % Legend box
            \begin{scope}[shift={(5,0)}]
                \draw[fill=red!20, draw=red!70] (0,0) rectangle (0.5,0.5);
                \node[anchor=north west] at (0.5,0.5) {Avant rotation};
                \draw[fill=blue!20, draw=blue!70] (0,1) rectangle (0.5,1.5);
                \node[anchor=north west] at (0.5,1.5) {Après rotation};
            \end{scope}
        \end{axis}
    \end{tikzpicture}
    \caption{Rotation de centre $\Omega(\omega)$ et d'angle $\theta$ — image d'un triangle.}
\end{figure}



\section{Similitudes indirectes}

\subsection{Symétrie axiale}
\begin{definition}
    Soit $(D)$ la droite passant par $\Omega(\omega)$ et dirigée par le vecteur unitaire $\vec{u}(e^{i\theta})$.

    Soit $M'(z')$ l'image de $M(z)$ par la symétrie axiale d'axe $(D)$.

    % représentation graphique
    La relation complexe de la symétrie axiale d'axe $(D)$ est :
    $$\boxed{z'=\bar{z}e^{2i\theta}-\omega e^{2i\theta}+\omega}$$
    où $\omega$ est l'affixe de $\Omega$ et $\theta$ l'argument de $\vec{u}$.

\end{definition}

\begin{figure}[htbp]
    \centering
    \begin{tikzpicture}
        \begin{axis}[
            axis lines=middle,
            xlabel=$\Re$,
            ylabel=$\Im$,
            xmin=-3, xmax=3,
            ymin=-3, ymax=3,
            xtick={-2,-1,0,1,2},
            ytick={-2,-1,0,1,2},
            grid=both,
            width=10cm,
            height=10cm,
            domain=-3:3,
            samples=100,
        ]
            % Draw the line (D)
            \addplot[blue, thick] {tan(45)*(x)}; % Example line equation: y = tan(45°)*x
            \node at (axis cs:2,2) [anchor=north west] {$(D)$};

            % Draw the original point M
            \addplot[red, only marks] coordinates {(1,2)};
            \node at (axis cs:1,2) [anchor=south west] {$M(z)$};

            % Draw the reflected point M'
            \addplot[green!70!black, only marks] coordinates {(2,1)};
            \node at (axis cs:2,1) [anchor=north west] {$M'(z')$};

            % Draw the perpendicular from M to (D)
            \draw[dashed, gray] (axis cs:1,2) -- (axis cs:1.5,1.5);
            \draw[dashed, gray] (axis cs:2,1) -- (axis cs:1.5,1.5);

            % Mark the foot of the perpendicular
            \draw[gray, fill=gray] (axis cs:1.5,1.5) circle (1.6pt);
            \node at (axis cs:1.5,1.5) [anchor=north east] {$H$};
        \end{axis}
    \end{tikzpicture}
    \caption{Représentation graphique de la symétrie axiale d'axe $(D)$.}
\end{figure}


\begin{proof}
    On a $M$ et $M'$ symétriques par rapport à $(D)\iff \begin{cases} \vec{OM'}=\vec{OM}+2\vec{HM}\\ \vec{HM}\perp\vec{u}\end{cases}$.\\
    % en utilisant les modules et arguments
    % utilise uniquement omega, m', m et u, PAS H
    $$\iff \begin{cases} \Omega M' = \Omega M \\ (\widehat{\vec{\Omega M},\vec{u}})\equiv(\widehat{\vec{u},\Omega M'})\end{cases}$$
    \\
    $$
    \iff \begin{cases} |z'-\omega|=|z-\omega| \\ \theta-\arg(z-\omega)\equiv\arg(z'-\omega)-\theta[2\pi]\end{cases}$$
    \\
    $$
    \iff \begin{cases} |z'-\omega|=|z-\omega| \\ \arg(z'-\omega)\equiv2\theta-\arg(z-\omega)[2\pi]\end{cases}$$\\$$
    \iff z'-\omega=|z-\omega|e^{i(2\theta-\arg(z-\omega))}=\overline{|z-\omega|}e^{2i\theta}e^{i\arg(\overline{(z-\omega)})}=e^{2i\theta}(\bar{z}-\bar{\omega})\\
    $$\\

    Conclusion : $z'=\bar{z}e^{2i\theta}-\omega e^{2i\theta}+\omega$.
    
\end{proof}

\begin{example}
    Soit $(D)$ la droite passant par $\Omega(1+i)$ et dirigée par le vecteur unitaire $\vec{u}(e^{i\frac{\pi}{3}})$. Déterminer l'image de $M(2+2i)$ par la symétrie axiale d'axe $(D)$.
    \textbf{Solution : }\\
    $z'= \bar{z}e^{2i\theta}-\omega e^{2i\theta}+\omega$\\
    $= (2-2i)e^{2i\frac{\pi}{3}}-(1-i)e^{2i\frac{\pi}{3}}+(1+i)$\\
    $= (2-2i)(-\frac{1}{2}+i\frac{\sqrt{3}}{2})-(1-i)(-\frac{1}{2}+i\frac{\sqrt{3}}{2})+(1+i)$\\
    $= (-1-\sqrt{3}-i(1-\sqrt{3}))-(\frac{-1-\sqrt{3}}{2}+i\frac{1-\sqrt{3}}{2})+(1+i)$\\
    $= \frac{-1+\sqrt{3}}{2}+i\frac{1+\sqrt{3}}{2}$\\
    $\boxed{\implies M'(\frac{-1+\sqrt{3}}{2}+i\frac{1+\sqrt{3}}{2})}$
\end{example}

\begin{exercise}
    Soit $(D)$ la droite d'équation cartésienne $y=x+1$. Déterminer l'image de $M(2+i)$ par la symétrie axiale d'axe $(D)$.

    \textbf{Solution : }\\
    $(D)$ passe par $A(i)$ et $B(-1)$. Donc $\vec{AB}$ ou $\vec{BA}$ est un vecteur directeur de $(D)$.\\
    Ce qui veut dire que $\vec{BA}(1+i)$ est un vecteur directeur de $(D)$.\\
    On prend $\vec{u}(e^{i\frac{\pi}{4}})$ vecteur unitaire directeur de $(D)$, passant par $B(-1)$.\\
    On aura donc : $M'=S_{(D)}(M) \iff z'=e^{i\frac{\pi}{2}}(\bar{z}-(-1))+(-1)$\\
    $\iff \boxed{z' = 3i} \iff \boxed{M'(3i)}$.
\end{exercise}

\begin{definition}[Similitude indirecte]
    On appelle une similitude indirect d'axe $(D)$ de centre $\Omega(\omega)$ et de rapport $r$ toute application de la forme $f=h\circ S_{(D)}$, avec $$\begin{cases}h \text{ homothétie de centre } \Omega(\omega) \text{ et de rapport } r\\ S_{(D)} \text{ symétrie axiale d'axe } (D) \text{ passant par }\Omega\end{cases}$$
\end{definition}

\begin{property}
    Soient $h=H(\Omega(\omega),r)$ et $\vec{u}(e^{i\theta})$ un vecteur unitaire directeur de $(D)$.

    On a : $$f(z)=a\bar{z}+b \text{ avec } \begin{cases} a=re^{2i\theta}\\ b=\omega-re^{2i\theta}\bar{\omega}\end{cases}$$
\end{property}

\begin{proof}
    On a : $f(z)=h(S_{(D)}(z))=h(\bar{z}e^{2i\theta}-\omega e^{2i\theta}+\omega)$\\
    $=\dots$\\
    $=a\bar{z}+b$ avec $\begin{cases} a=re^{2i\theta}\\ b=\omega-re^{2i\theta}\bar{\omega}\end{cases}$
\end{proof}

\begin{property}
Soit $f:z\rightarrowtail a\bar{z}+b$, avec $a\in\mathbb{C}^*$ et $b\in\mathbb{C}$.

\begin{outline}
    \1 \textbf{Si $|a|\neq1$, alors $f$ est une similitude indirecte. Elle admet un unique point fixe $\Omega(\omega)$, et dans ce cas, $f(z)-\omega=r.e^{i2\theta}(\bar{z}-\bar{\omega})$}, avec r=$|a|$ et $\theta\equiv\frac{\arg(a)}{2}[\pi]$.
    \1 \textbf{Si $|a|=1$, alors $f$ est une isométrie indirecte.}
        \2 Si $b\neq-a\bar{b}$, alors $f$ n'admet pas de point fixe.
        \2 Si $b=-a\bar{b}$, alors $f$ admet une infinité de points fixes appartenant à la droite d'équation complexe $(D)$
            \3 Si $b=0$, alors $(D)$ est dirigée par $\vec{u}(e^{i\theta})$ avec $\theta\equiv\frac{\arg(a)}{2}[2\pi]$.
            \3 Si $b\neq0$, alors $(D)$ est la médiatrice du segment $OB$ ($B(b)$).

\end{outline}


% \begin{figure}[htbp]
%     \centering
%     \begin{tikzpicture}[
%         node distance=1.5cm,
%         every node/.style={draw, rectangle, rounded corners, align=center, minimum width=2cm, minimum height=0.7cm, font=\footnotesize},
%         arrow/.style={->, thick},
%     ]

%     % Nodes
%     \node (start) {$f(z)=a\overline{z}+b$\\ ($a\in\mathbb{C}^*,\; b\in\mathbb{C}$)};
%     \node[below of=start] (modulus) {$|a|=1$?};
%     \node[below left of=modulus, xshift=-1.5cm] (similitude) {Similitude indirecte\\($|a|\neq 1$)\\Unique point fixe $\Omega(\omega)$\\$f(z)-\omega = r e^{i2\theta}(\overline{z}-\overline{\omega})$};
%     \node[below right of=modulus, xshift=1.5cm] (isometry) {Isométrie indirecte\\($|a|=1$)\\$b = -a\overline{b}$?};
%     \node[below left of=isometry, xshift=-1.5cm] (no_fixed) {Pas de point fixe\\($b \neq -a\overline{b}$)};
%     \node[below right of=isometry, xshift=1.5cm] (infinite_fixed) {Infinité de points fixes\\($b = -a\overline{b}$)\\Droite (D)};
%     \node[below left of=infinite_fixed, xshift=-1.5cm] (D_direction) {(D) dirigée par $\vec{u}(e^{i\theta})$\\($b = 0$)};
%     \node[below right of=infinite_fixed, xshift=1.5cm] (D_mediator) {(D) médiatrice du segment $OB$\\($b \neq 0$)};

%     % Arrows
%     \draw[arrow] (start) -- (modulus);
%     \draw[arrow] (modulus) -- node[left] {Non} (similitude);
%     \draw[arrow] (modulus) -- node[right] {Oui} (isometry);
%     \draw[arrow] (isometry) -- node[left] {Non} (no_fixed);
%     \draw[arrow] (isometry) -- node[right] {Oui} (infinite_fixed);
%     \draw[arrow] (infinite_fixed) -- node[left] {$b=0$} (D_direction);
%     \draw[arrow] (infinite_fixed) -- node[right] {$b\neq0$} (D_mediator);  
%     \end{tikzpicture}
%     \caption{Diagramme des cas pour la fonction $f(z)=a\overline{z}+b$.}
% \end{figure}

\end{property}

% démonstration à faire plus tard

% \begin{proof}
%     \quad
%     \begin{enumerate}
%         \item Supposons que $|a|\neq1$. On cherche les points fixes de $f$, c'est-à-dire les solutions de l'équation $z=a\bar{z}+b$.\\
%         $z=a\bar{z}+b$\\
%         $\iff \bar{z}=\bar{a}z+\bar{b}$\\
%         $\iff \bar{z}-\bar{a}z=-\bar{b}$\\
%         $\iff (1-\bar{a})\bar{z}=-\bar{b}$\\
%         $\iff \bar{z}=\frac{-\bar{b}}{1-\bar{a}}$\\
%         Donc, le point fixe est donné par $z=\frac{-b}{1-a}$.
%     \end{enumerate}
% \end{proof}

\begin{remark}
    On remarque que dans le cas où $|a|=1$ et $b\neq -a\bar{b}$, alors la composée de $f\circ f$ et une translation de vecteur $\vec{v}(a\bar{b}+b)$.
\end{remark}

\end{document}