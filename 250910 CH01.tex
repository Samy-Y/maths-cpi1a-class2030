\documentclass[12pt, a4paper]{report}

% --- PACKAGES ESSENTIELS ---
\usepackage[french]{babel}
\usepackage[utf8]{inputenc}
\usepackage[T1]{fontenc}
\usepackage{lmodern} % Police plus nette et moderne
\usepackage{amsmath, amssymb, amsthm, amsfonts} % Importe amsthm pour les environnements
\usepackage{geometry}
\usepackage{graphicx}
\usepackage{todonotes}
\usepackage{outlines}
\usepackage{pgfplots}
\pgfplotsset{compat=1.18}
\usepackage{float}
\usepackage{afterpage}
\usepackage{placeins}
\usepackage{titlesec}
\usepackage{tcolorbox}
\tcbuselibrary{skins}
\usepackage{ragged2e}

% Clear any pending floats before document end
\AtEndDocument{\clearpage}

% --- PACKAGES POUR LE STYLE ---
\usepackage[x11names, dvipsnames]{xcolor}
\usepackage{framed}
\usepackage{tikz}
\usepackage{tikz}
\usepackage[
    unicode=true,
    pdfusetitle,
    colorlinks=true,
    linkcolor=MidnightBlue,
    urlcolor=Firebrick4,
    citecolor=Green4
]{hyperref}
\usetikzlibrary{calc,arrows.meta}

% --- CONFIGURATION DE LA MISE EN PAGE ---
\geometry{
    a4paper,
    top=25mm,
    bottom=25mm,
    left=20mm,
    right=20mm,
}
\setlength{\parindent}{0pt}
\setlength{\parskip}{0.8em}

% --- DÉFINITION D'UNE PALETTE DE COULEURS ---
\definecolor{mainBlue}{HTML}{003366}
\definecolor{accentBlue}{HTML}{005b96}
\definecolor{boxBg}{HTML}{E8F0F7}
\definecolor{boxFrame}{HTML}{005b96}
\definecolor{remarkBg}{HTML}{F0F8FF}
\definecolor{proofBg}{HTML}{F6F6F6}

% Colors for different environments
\definecolor{definitionColor}{HTML}{2E86C1}    % Blue for definitions
\definecolor{theoremColor}{HTML}{28B463}       % Green for theorems
\definecolor{propertyColor}{HTML}{E74C3C}      % Red for properties
\definecolor{propositionColor}{HTML}{8E44AD}   % Purple for propositions
\definecolor{remarkColor}{HTML}{F39C12}        % Orange for remarks
\definecolor{exampleColor}{HTML}{17A2B8}       % Teal for examples
\definecolor{exerciseColor}{HTML}{6C757D}      % Gray for exercises
\definecolor{consequenceColor}{HTML}{DC3545}   % Dark red for consequences
\definecolor{applicationColor}{HTML}{FF6B35}   % Orange-red for applications
\definecolor{proofColor}{HTML}{495057}         % Dark gray for proofs

% Background colors (lighter versions of heading colors)
\definecolor{definitionBg}{HTML}{EBF3FD}       % Light blue background
\definecolor{theoremBg}{HTML}{E8F6F0}          % Light green background
\definecolor{propertyBg}{HTML}{FADBD8}         % Light red background
\definecolor{propositionBg}{HTML}{F4ECF7}      % Light purple background
\definecolor{remarkBg}{HTML}{FEF9E7}           % Light orange background
\definecolor{exampleBg}{HTML}{E1F5F8}          % Light teal background
\definecolor{exerciseBg}{HTML}{F8F9FA}         % Light gray background
\definecolor{consequenceBg}{HTML}{FADBD8}      % Light dark red background
\definecolor{applicationBg}{HTML}{FFF4F1}      % Light orange background
\definecolor{proofBg}{HTML}{F8F9FA}            % Light gray background

% --- PERSONNALISATION DES TITRES ---
\titleformat{\chapter}[display]
  {\normalfont\Huge\bfseries\color{mainBlue}\raggedright}
  {Partie\ \thechapter}
  {5pt}
  {\Huge\color{mainBlue}\raggedright}
\titlespacing*{\chapter}{0pt}{-30pt}{40pt}
\titleformat{\section}{\normalfont\Large\bfseries\color{accentBlue}\raggedright}{\thesection}{1em}{}
\titleformat{\subsection}{\normalfont\large\bfseries\color{accentBlue}\raggedright}{\thesubsection}{1em}{}

% --- DÉFINITION DES STYLES D'ENVIRONNEMENTS ---
% Définition des styles de boîtes tcolorbox
\tcbset{
  coursebox/.style={
    colback=boxBg,
    colframe=boxFrame,
    sharp corners,
    boxrule=0.5pt,
    arc=0mm,
    boxsep=5pt,
    before=\par\bigskip,
    after=\par\bigskip,
  },
  remarkbox/.style={
    colback=remarkBg,
    colframe=MidnightBlue!50,
    sharp corners,
    boxrule=0.5pt,
    arc=0mm,
    boxsep=5pt,
    before=\par\bigskip,
    after=\par\bigskip,
  },
  proofbox/.style={
    colback=proofBg,
    colframe=Gray,
    sharp corners,
    boxrule=0.5pt,
    arc=0mm,
    boxsep=5pt,
    before=\par\bigskip,
    after=\par\bigskip,
  }
}

% Déclaration des environnements théorème avec amsthm
\newtheorem{definition}{Définition}
\newtheorem{proposition}{Proposition}
\newtheorem{theorem}{Théorème}
\newtheorem{property}{Propriété}
\newtheorem{remark}{Remarque}
\newtheorem{example}{Exemple}
\newtheorem{exercise}{Exercice}
\newtheorem{consequence}{Conséquence}
\newtheorem{application}{Application}
% Définition de l'environnement importantbox amélioré
\newenvironment{importantbox}[1][]{%
  \begin{tcolorbox}[
    colback=yellow!10,
    colframe=orange!70!red,
    colbacktitle=orange!70!red,
    coltitle=white,
    title=\textbf{Formule importante},
    fonttitle=\bfseries,
    arc=3pt,
    boxrule=1.5pt,
    enhanced,
    drop fuzzy shadow,
    center title,
    #1
  ]%
  \centering\large\bfseries
}{\end{tcolorbox}}
% Nouvel environnement pour les formules clés
\newenvironment{keyformula}[1][]{%
  \begin{tcolorbox}[
    colback=blue!5,
    colframe=blue!70,
    colbacktitle=blue!70,
    coltitle=white,
    title=\textbf{Formule clé},
    fonttitle=\bfseries,
    arc=2pt,
    boxrule=1pt,
    center title,
    enhanced,
    drop fuzzy shadow,
    #1
  ]%
  \centering\large
}{\end{tcolorbox}}
  \centering\large
% Environnement pour les méthodes importantes
\newenvironment{methodbox}[1][]{%
  \begin{tcolorbox}[
    colback=green!8,
    colframe=green!60!black,
    colbacktitle=green!60!black,
    coltitle=white,
    title=\textbf{Méthode},
    fonttitle=\bfseries,
    arc=2pt,
    boxrule=1pt,
    enhanced,
    #1
  ]%
  \normalfont
}{\end{tcolorbox}}
  \normalfont
% Environnement pour les points d'attention
\newenvironment{attention}[1][]{%
  \begin{tcolorbox}[
    colback=red!8,
    colframe=red!70,
    colbacktitle=red!70,
    coltitle=white,
    title=\textbf{$\triangle$ Attention},
    fonttitle=\bfseries,
    arc=2pt,
    boxrule=1pt,
    enhanced,
    #1
  ]%
  \normalfont
}{\end{tcolorbox}}

% Style pour les listes élégantes
\usepackage{enumitem}
\newlist{elegantlist}{itemize}{3}
\setlist[elegantlist,1]{
  label=\textcolor{accentBlue}{$\blacktriangleright$},
  leftmargin=1.5em,
  itemsep=0.3em
}
\setlist[elegantlist,2]{
  label=\textcolor{accentBlue}{$\circ$},
  leftmargin=2em,
  itemsep=0.2em
}
\setlist[elegantlist,3]{
  label=\textcolor{accentBlue}{$-$},
  leftmargin=2.5em,
  itemsep=0.1em
}

% Redéfinition des environnements existants avec tcolorbox
\renewenvironment{definition}[1][]{%
  \refstepcounter{definition}%
  \begin{tcolorbox}[colback=definitionBg, colframe=definitionColor, 
    sharp corners, boxrule=0.5pt, arc=0mm, boxsep=5pt,
    before=\par\bigskip, after=\par\bigskip,
    title=\textbf{Définition \thechapter.\thesection.\thedefinition%
    \ifx&#1&\else\ \textit{(#1)}\fi%
    }, fonttitle=\bfseries\color{white}, colbacktitle=definitionColor]%
  \normalfont
}{\end{tcolorbox}}

\renewenvironment{proposition}[1][]{%
  \refstepcounter{proposition}%
  \begin{tcolorbox}[colback=propositionBg, colframe=propositionColor,
    sharp corners, boxrule=0.5pt, arc=0mm, boxsep=5pt,
    before=\par\bigskip, after=\par\bigskip,
    title=\textbf{Proposition \thechapter.\thesection.\theproposition%
    \ifx&#1&\else\ \textit{(#1)}\fi%
    }, fonttitle=\bfseries\color{white}, colbacktitle=propositionColor]%
  \normalfont
}{\end{tcolorbox}}

\renewenvironment{theorem}[1][]{%
  \refstepcounter{theorem}%
  \begin{tcolorbox}[colback=theoremBg, colframe=theoremColor,
    sharp corners, boxrule=0.5pt, arc=0mm, boxsep=5pt,
    before=\par\bigskip, after=\par\bigskip,
    title=\textbf{Théorème \thechapter.\thesection.\thetheorem%
    \ifx&#1&\else\ \textit{(#1)}\fi%
    }, fonttitle=\bfseries\color{white}, colbacktitle=theoremColor]%
  \normalfont
}{\end{tcolorbox}}

\renewenvironment{property}[1][]{%
  \refstepcounter{property}%
  \begin{tcolorbox}[colback=propertyBg, colframe=propertyColor,
    sharp corners, boxrule=0.5pt, arc=0mm, boxsep=5pt,
    before=\par\bigskip, after=\par\bigskip,
    title=\textbf{Propriété \thechapter.\thesection.\theproperty%
    \ifx&#1&\else\ \textit{(#1)}\fi%
    }, fonttitle=\bfseries\color{white}, colbacktitle=propertyColor]%
  \normalfont
}{\end{tcolorbox}}

\renewenvironment{remark}[1][]{%
  \refstepcounter{remark}%
  \begin{tcolorbox}[colback=remarkBg, colframe=remarkColor,
    sharp corners, boxrule=0.5pt, arc=0mm, boxsep=5pt,
    before=\par\bigskip, after=\par\bigskip,
    title=\textbf{Remarque \thechapter.\thesection.\theremark%
    \ifx&#1&\else\ \textit{(#1)}\fi%
    }, fonttitle=\bfseries\color{white}, colbacktitle=remarkColor]%
  \normalfont
}{\end{tcolorbox}}

\renewenvironment{example}[1][]{%
  \refstepcounter{example}%
  \begin{tcolorbox}[colback=exampleBg, colframe=exampleColor,
    sharp corners, boxrule=0.5pt, arc=0mm, boxsep=5pt,
    before=\par\bigskip, after=\par\bigskip,
    title=\textbf{Exemple \thechapter.\thesection.\theexample%
    \ifx&#1&\else\ \textit{(#1)}\fi%
    }, fonttitle=\bfseries\color{white}, colbacktitle=exampleColor]%
  \normalfont
}{\end{tcolorbox}}

\renewenvironment{exercise}[1][]{%
  \refstepcounter{exercise}%
  \begin{tcolorbox}[colback=exerciseBg, colframe=exerciseColor,
    sharp corners, boxrule=0.5pt, arc=0mm, boxsep=5pt,
    before=\par\bigskip, after=\par\bigskip,
    title=\textbf{Exercice \thechapter.\thesection.\theexercise%
    \ifx&#1&\else\ \textit{(#1)}\fi%
    }, fonttitle=\bfseries\color{white}, colbacktitle=exerciseColor]%
  \normalfont
}{\end{tcolorbox}}

\renewenvironment{consequence}[1][]{%
  \refstepcounter{consequence}%
  \begin{tcolorbox}[colback=consequenceBg, colframe=consequenceColor,
    sharp corners, boxrule=0.5pt, arc=0mm, boxsep=5pt,
    before=\par\bigskip, after=\par\bigskip,
    title=\textbf{Conséquence \thechapter.\thesection.\theconsequence%
    \ifx&#1&\else\ \textit{(#1)}\fi%
    }, fonttitle=\bfseries\color{white}, colbacktitle=consequenceColor]%
  \itshape
}{\end{tcolorbox}}

\renewenvironment{application}[1][]{%
  \refstepcounter{application}%
  \begin{tcolorbox}[colback=applicationBg, colframe=applicationColor,
    sharp corners, boxrule=0.5pt, arc=0mm, boxsep=5pt,
    before=\par\bigskip, after=\par\bigskip,
    title=\textbf{Application \thechapter.\thesection.\theapplication%
    \ifx&#1&\else\ \textit{(#1)}\fi%
    }, fonttitle=\bfseries\color{white}, colbacktitle=applicationColor]%
  \normalfont
}{\end{tcolorbox}}

\renewenvironment{proof}[1][]{%
  \begin{tcolorbox}[colback=proofBg, colframe=proofColor,
    sharp corners, boxrule=0.5pt, arc=0mm, boxsep=5pt,
    before=\par\bigskip, after=\par\bigskip,
    title=\textbf{Preuve%
    \ifx&#1&\else\ \textit{(#1)}\fi%
    }, fonttitle=\bfseries\color{white}, colbacktitle=proofColor]%
  \normalfont
}{\end{tcolorbox}}

\renewenvironment{importantbox}[1][]{%
  \begin{tcolorbox}[coursebox, title=, #1]%
  \centering\large
}{\end{tcolorbox}}

\newcommand{\doubleheadrightarrow}{%
    \tikz[overlay,remember picture,baseline]{%
        \draw[->,very thick,mainBlue] (0,0) -- (0.5em,0.5em);
        \draw[->,very thick,accentBlue] (0.25em,0.25em) -- (0.75em,0.75em);
    }%
}

% FIN DU PREAMBULE

% --- DOCUMENT ---
\begin{document}

\title{
    % \begin{center}
    % \includegraphics[width=0.3\textwidth]{./assets/logo.png} % <-- REMETTEZ LE CHEMIN VERS VOTRE LOGO ICI
    % \end{center}
    \vspace{1cm}
    \huge{\textbf{Chapitre I : Calcul Algébrique}}
}
\author{Rédigé par Samy Youssoufine}
\date{\today}

\maketitle

\tableofcontents
\newpage

\chapter{Sommes ($\Sigma$)}

\section{Généralités}

\begin{definition}
Soit $I$ un ensemble fini non vide et $(a_i)_{i \in I}$ une famille d'éléments de $\mathbb{C}$. La \textbf{somme} des éléments de cette famille est notée :
$$ \sum_{i \in I} a_i $$
Si $I = \{0, 1, \dots, n\}$ (où $n \in \mathbb{N}$), alors on écrit :
$$ \sum_{i=0}^{n} a_i = \sum_{0 \le i \le n} a_i = a_0 + a_1 + \dots + a_n $$
\end{definition}

\begin{property}[Linéarité de la somme]
\quad
\begin{enumerate}
    \item \textbf{Additivité sur les ensembles disjoints :} Si $I$ et $J$ sont deux ensembles finis tels que $I \cap J = \emptyset$, alors :
    $$ \sum_{k \in I \cup J} a_k = \sum_{i \in I} a_i + \sum_{j \in J} a_j $$
    
    \item \textbf{Homogénéité :} Si $\lambda \in \mathbb{C}$ est une constante (indépendante de l'indice de sommation $i$), alors :
    $$ \sum_{i \in I} \lambda a_i = \lambda \sum_{i \in I} a_i $$
    
    \item \textbf{Additivité :} Pour deux familles $(a_i)_{i \in I}$ et $(b_i)_{i \in I}$ :
    $$ \sum_{i \in I} (a_i + b_i) = \sum_{i \in I} a_i + \sum_{i \in I} b_i $$
\end{enumerate}
\end{property}

\begin{proof}
\quad
La preuve repose sur les propriétés de commutativité et d'associativité de l'addition.
\begin{enumerate}
    \item On pose $I = \{i_1, \dots, i_p\}$ et $J = \{j_1, \dots, j_q\}$. Puisque $I \cap J = \emptyset$, l'union $I \cup J$ contient $p+q$ éléments distincts. La somme sur $I \cup J$ est donc $(a_{i_1} + \dots + a_{i_p}) + (a_{j_1} + \dots + a_{j_q})$, ce qui correspond à la somme des deux sommes.
    \item C'est une simple factorisation : $\lambda a_1 + \dots + \lambda a_p = \lambda(a_1 + \dots + a_p)$.
\end{enumerate}
\end{proof}

\begin{consequence}
\quad
\itshape
\begin{enumerate}
    \item \textbf{Relation de Chasles pour les sommes :} Si $1 \le n < m$, alors :
    $$ \sum_{i=1}^{m} a_i = \sum_{i=1}^{n} a_i + \sum_{i=n+1}^{m} a_i $$
    
    \item \textbf{Somme d'une constante :}
    $$ \sum_{i \in I} \lambda = \lambda \cdot \text{card}(I) $$
    En particulier, $\sum_{k=p}^{n} \lambda = \lambda \cdot (n-p+1)$.
\end{enumerate}
\end{consequence}

\begin{example}[Sommes usuelles]
\begin{enumerate}
    \item $\sum_{k=0}^{n} k = \frac{n(n+1)}{2}$
    \item $\sum_{k=0}^{n} (2k+1) = 2 \sum_{k=0}^{n} k + \sum_{k=0}^{n} 1 = 2 \frac{n(n+1)}{2} + (n+1) = n(n+1) + (n+1) = (n+1)^2$.
\end{enumerate}
\end{example}

\section{Changement d'indice et sommes télescopiques}

\subsection{Changement d'indice}
\begin{proposition}
Soit $(a_k)_{k \in \mathbb{N}}$ une suite d'éléments de $\mathbb{C}$. Pour tout $p \in \mathbb{Z}$ :
$$ \sum_{k=m}^{n} a_{k+p} = \sum_{j=m+p}^{n+p} a_j $$
Cette opération est appelée un \textbf{changement d'indice} (ou translation d'indice).
\end{proposition}

\begin{remark}
\begin{enumerate}
    \item Un changement d'indice doit être une transformation affine ($k \mapsto ak+b$) pour préserver la nature "consécutive" des indices. Un changement non linéaire comme $k \mapsto k^2$ n'est pas valide.
    \item La permutation des bornes est un cas particulier : $\sum_{k=1}^{n} a_{n-k} = a_{n-1} + a_{n-2} + \dots + a_0 = \sum_{j=0}^{n-1} a_j$.
\end{enumerate}
\end{remark}

\subsection{Sommes télescopiques}
\begin{proposition}
Pour une suite $(a_k)$, on a :
$$ \sum_{k=p}^{n} (a_{k+1} - a_k) = a_{n+1} - a_p $$
\end{proposition}
\begin{proof}
\quad
Le développement de la somme donne :
$$ (a_{p+1} - a_p) + (a_{p+2} - a_{p+1}) + \dots + (a_n - a_{n-1}) + (a_{n+1} - a_n) $$
Tous les termes intermédiaires s'annulent deux à deux, ne laissant que $-a_p$ et $a_{n+1}$.
\end{proof}

\begin{example}
\begin{enumerate}
    \item Calculer $S_n = \sum_{k=1}^{n} \frac{1}{k(k+1)}$.\\
    On décompose en éléments simples : $\frac{1}{k(k+1)} = \frac{1}{k} - \frac{1}{k+1}$. La somme devient télescopique :
    $$ S_n = \sum_{k=1}^{n} \left( \frac{1}{k} - \frac{1}{k+1} \right) = \frac{1}{1} - \frac{1}{n+1} = \frac{n}{n+1} $$
    
    \item Calculer $T_n = \sum_{k=1}^{n} \frac{1}{\sqrt{k+1} + \sqrt{k}}$.\\
    On utilise l'expression conjuguée : $\frac{1}{\sqrt{k+1} + \sqrt{k}} = \frac{\sqrt{k+1} - \sqrt{k}}{k+1-k} = \sqrt{k+1} - \sqrt{k}$.
    $$ T_n = \sum_{k=1}^{n} (\sqrt{k+1} - \sqrt{k}) = \sqrt{n+1} - \sqrt{1} = \sqrt{n+1} - 1 $$
\end{enumerate}
\end{example}

\begin{remark}[Télescopage à plusieurs pas]
Pour une somme comme $\sum_k (a_{k+2} - a_k)$, on peut la décomposer pour faire apparaître des sommes télescopiques simples :
$$ \sum_k (a_{k+2} - a_k) = \sum_k (a_{k+2} - a_{k+1} + a_{k+1} - a_k) = \sum_k (a_{k+2} - a_{k+1}) + \sum_k (a_{k+1} - a_k) $$
\end{remark}

\section{Sommes classiques}

\subsection{Identité remarquable et suite géométrique}

\begin{theorem}[Identité de Bernoulli]
\quad
\itshape
Pour tous $a, b \in \mathbb{C}$ et tout $n \in \mathbb{N}^*$:
\begin{importantbox}{Formule}
$ a^n - b^n = (a-b) \sum_{k=0}^{n-1} a^k b^{n-1-k} $
\end{importantbox}
\end{theorem}

\begin{proof}
\quad
On développe le membre de droite :
$$ (a-b) \sum_{k=0}^{n-1} a^k b^{n-1-k} = \sum_{k=0}^{n-1} a^{k+1} b^{n-1-k} - \sum_{k=0}^{n-1} a^k b^{n-k} $$
$$ = \sum_{j=1}^{n} a^j b^{n-j} - \sum_{k=0}^{n-1} a^k b^{n-k} = (a^n + \sum_{j=1}^{n-1} a^j b^{n-j}) - (b^n + \sum_{k=1}^{n-1} a^k b^{n-k}) = a^n - b^n $$
\end{proof}

\begin{remark}
En posant $b=1$ et $a \neq 1$, on retrouve la somme des termes d'une suite géométrique.
\end{remark}

\begin{theorem}[Somme géométrique]
\quad
\itshape
Pour tout $x \in \mathbb{C}$ et $n \in \mathbb{N}$ :
\begin{enumerate}
    \item $$ \sum_{k=0}^{n} x^k = 
        \begin{cases} 
            n+1 & \text{si } x = 1 \\
            \frac{1-x^{n+1}}{1-x} & \text{si } x \neq 1 
        \end{cases}
    $$
    \item Pour $p \le n$ :
    $$ \sum_{k=p}^{n} x^k = 
        \begin{cases} 
            n-p+1 & \text{si } x = 1 \\
            x^p \frac{1-x^{n-p+1}}{1-x} & \text{si } x \neq 1 
        \end{cases}
    $$
\end{enumerate}
\end{theorem}

\subsection{Binôme de Newton}

\begin{theorem}[Formule du binôme de Newton]
\quad
\itshape
Pour tous $a, b \in \mathbb{C}$ et tout $n \in \mathbb{N}$ :
\begin{importantbox}{Binôme de Newton}
$ (a+b)^n = \sum_{k=0}^{n} \binom{n}{k} a^k b^{n-k} $
\end{importantbox}
où $\binom{n}{k} = \frac{n!}{k!(n-k)!}$ est le coefficient binomial.
\end{theorem}

\begin{property}[Propriétés des coefficients binomiaux]
\quad
\itshape
\begin{enumerate}
    \item $\binom{n}{0} = \binom{n}{n} = 1$.
    \item \textbf{Symétrie :} Pour $k \in \{0, \dots, n\}$, $\binom{n}{k} = \binom{n}{n-k}$.
    \item \textbf{Formule de Pascal :} Pour $k \in \{0, \dots, n-1\}$, $\binom{n}{k} + \binom{n}{k+1} = \binom{n+1}{k+1}$.
\end{enumerate}
\end{property}

\begin{proof}[Preuve du binôme par récurrence]
\quad
\begin{itemize}
    \item \textbf{Initialisation ($n=0$) :} $(a+b)^0 = 1$ et $\sum_{k=0}^{0} \binom{0}{k} a^k b^{0-k} = \binom{0}{0} a^0 b^0 = 1$. La formule est vraie.
    \item \textbf{Hérédité :} Supposons la formule vraie pour un rang $n \in \mathbb{N}$.
    \begin{align*}
        (a+b)^{n+1} &= (a+b)(a+b)^n \\
        &= (a+b) \sum_{k=0}^{n} \binom{n}{k} a^k b^{n-k} \quad \text{(par H.R.)} \\
        &= \sum_{k=0}^{n} \binom{n}{k} a^{k+1} b^{n-k} + \sum_{k=0}^{n} \binom{n}{k} a^k b^{n+1-k} \\
        &= \sum_{j=1}^{n+1} \binom{n}{j-1} a^j b^{n+1-j} + \sum_{k=0}^{n} \binom{n}{k} a^k b^{n+1-k} \\
        &= \binom{n}{n} a^{n+1} + \sum_{k=1}^{n} \binom{n}{k-1} a^k b^{n+1-k} + \binom{n}{0} b^{n+1} + \sum_{k=1}^{n} \binom{n}{k} a^k b^{n+1-k} \\
        &= a^{n+1} + b^{n+1} + \sum_{k=1}^{n} \left( \binom{n}{k-1} + \binom{n}{k} \right) a^k b^{n+1-k} \\
        &= a^{n+1} + b^{n+1} + \sum_{k=1}^{n} \binom{n+1}{k} a^k b^{n+1-k} \quad \text{(Formule de Pascal)} \\
        &= \binom{n+1}{n+1} a^{n+1} b^0 + \binom{n+1}{0} a^0 b^{n+1} + \sum_{k=1}^{n} \binom{n+1}{k} a^k b^{n+1-k} \\
        &= \sum_{k=0}^{n+1} \binom{n+1}{k} a^k b^{n+1-k}
    \end{align*}
    La formule est vraie au rang $n+1$.
\end{itemize}
\end{proof}

\begin{exercise}
\quad
\begin{enumerate}
    \item Calculer $\sum_{k=0}^{n} \binom{n}{k}$.
    \item Calculer $\sum_{k=0}^{n} (-1)^k \binom{n}{k}$.
    \item Calculer $\sum_{k=0}^{n} k \binom{n}{k}$.
\end{enumerate}
\end{exercise}

\begin{proof}[Solutions]
\quad
\begin{enumerate}
    \item $\sum_{k=0}^{n} \binom{n}{k} = \sum_{k=0}^{n} \binom{n}{k} 1^k 1^{n-k} = (1+1)^n = 2^n$.
    \item $\sum_{k=0}^{n} (-1)^k \binom{n}{k} = \sum_{k=0}^{n} \binom{n}{k} (-1)^k 1^{n-k} = (-1+1)^n = 0$ pour $n \ge 1$.
    \item On utilise la technique de la dérivation. Soit $f(x) = (1+x)^n = \sum_{k=0}^n \binom{n}{k} x^k$.
    Alors $f'(x) = n(1+x)^{n-1} = \sum_{k=1}^n k \binom{n}{k} x^{k-1}$.
    Pour $x=1$, on obtient $\sum_{k=1}^n k \binom{n}{k} = n(1+1)^{n-1} = n 2^{n-1}$.
\end{enumerate}
\end{proof}

\section{Sommes doubles}

\begin{definition}
Soit une famille d'éléments $(a_{i,j})$ où $i \in I$ et $j \in J$ sont des ensembles finis. La somme double est :
$$ \sum_{(i,j) \in I \times J} a_{i,j} = \sum_{i \in I} \left( \sum_{j \in J} a_{i,j} \right) = \sum_{j \in J} \left( \sum_{i \in I} a_{i,j} \right) \quad \text{(Théorème de Fubini)} $$
\end{definition}

\begin{example}[Somme triangulaire]
Calculer $S = \sum_{1 \le i \le j \le n} \frac{i}{j}$.
On peut écrire cette somme comme une somme double et intervertir l'ordre de sommation :
$$ S = \sum_{j=1}^{n} \sum_{i=1}^{j} \frac{i}{j} = \sum_{j=1}^{n} \frac{1}{j} \left( \sum_{i=1}^{j} i \right) = \sum_{j=1}^{n} \frac{1}{j} \frac{j(j+1)}{2} = \frac{1}{2} \sum_{j=1}^{n} (j+1) = \frac{1}{2} \left( \frac{n(n+1)}{2} + n \right) = \frac{n(n+3)}{4} $$
\end{example}

\chapter{Produits ($\Pi$)}

\section{Définition et propriétés}

\begin{definition}
Soit $(a_i)_{i \in I}$ une famille d'éléments de $\mathbb{C}$ avec $I=\{i_1, \dots, i_n\}$ un ensemble fini.
On note le \textbf{produit} des éléments de cette famille par :
$$ \prod_{i \in I} a_i = a_{i_1} \times a_{i_2} \times \dots \times a_{i_n} $$
Si $I = \{1, \dots, n\}$, alors :
$$ \prod_{i \in I} a_i = \prod_{i=1}^{n} a_i = \prod_{1 \le i \le n} a_i $$
\end{definition}

\begin{example}
\begin{enumerate}
    \item \textbf{Factorielle :} $\prod_{k=1}^{n} k = 1 \times 2 \times \dots \times n = n!$
    \item \textbf{Produit et somme :} $\prod_{k=1}^{n} 2^k = 2^{\sum_{k=1}^{n} k} = 2^{\frac{n(n+1)}{2}}$
\end{enumerate}
\end{example}

\begin{property}
\quad
\itshape
\itshape
\begin{enumerate}
    \item $\prod_{i=1}^{n} (\lambda a_i) = \lambda^n \prod_{i=1}^{n} a_i$. Plus généralement, $\prod_{i \in I} (\lambda a_i) = \lambda^{\text{card}(I)} \prod_{i \in I} a_i$.
    \item Si $I \cap J = \emptyset$, alors $\prod_{k \in I \cup J} a_k = \left( \prod_{i \in I} a_i \right) \left( \prod_{j \in J} a_j \right)$.
    En particulier, pour $m > n$, $\prod_{i=1}^{m} a_i = \left( \prod_{i=1}^{n} a_i \right) \left( \prod_{i=n+1}^{m} a_i \right)$.
    \item \textbf{Changement d'indice :} $\prod_{i=1}^{n} a_{i+p} = \prod_{k=p+1}^{n+p} a_k$.
    \item \textbf{Produit télescopique :} Si $\forall i \in \{p, \dots, n\}, a_i \neq 0$, alors :
    $$ \prod_{i=p}^{n} \frac{a_{i+1}}{a_i} = \frac{a_{n+1}}{a_p} $$
    \item $\prod_{i \in I} (a_i b_i) = \left( \prod_{i \in I} a_i \right) \left( \prod_{i \in I} b_i \right)$.
\end{enumerate}
\end{property}

\begin{remark}
Attention à ne pas confondre les propriétés des sommes et des produits.
En général :
$$ \sum_{i \in I} a_i b_i \neq \left( \sum_{i \in I} a_i \right) \left( \sum_{i \in I} b_i \right) $$
En effet, le membre de droite correspond à une somme double : $\left( \sum_{i \in I} a_i \right) \left( \sum_{j \in I} b_j \right) = \sum_{(i,j) \in I^2} a_i b_j$.
\end{remark}

\section{Exercices sur les produits}

\begin{exercise}
\quad
\begin{enumerate}
    \item Calculer $\prod_{k=2}^{n} \left(1 - \frac{1}{k^2}\right)$.
    \item Calculer $\prod_{k=1}^{n} \left(1 - \frac{1}{4k^2}\right)$ à l'aide des factorielles.
    \item (Intégrales de Wallis) Soit $n \ge 0$, on pose $U_n = \int_{0}^{\pi/2} \sin^n(t) \, dt$.
    \begin{enumerate}
        \item Calculer $U_0$ et $U_1$.
        \item Montrer que $\forall n \in \mathbb{N}, U_{n+2} = \frac{n+1}{n+2} U_n$.
        \item En déduire $U_{2n}$ et $U_{2n+1}$ en fonction de $n$.
    \end{enumerate}
\end{enumerate}
\end{exercise}

\begin{proof}[Solutions (Partielles)]
\quad
\begin{enumerate}
    \item \textbf{Produit télescopique :}
    $$ \prod_{k=2}^{n} \left(1 - \frac{1}{k^2}\right) = \prod_{k=2}^{n} \frac{k^2-1}{k^2} = \prod_{k=2}^{n} \frac{(k-1)(k+1)}{k \cdot k} = \left(\prod_{k=2}^{n} \frac{k-1}{k}\right) \left(\prod_{k=2}^{n} \frac{k+1}{k}\right) $$
    $$ = \left(\frac{1}{2} \cdot \frac{2}{3} \cdots \frac{n-1}{n}\right) \cdot \left(\frac{3}{2} \cdot \frac{4}{3} \cdots \frac{n+1}{n}\right) = \left(\frac{1}{n}\right) \cdot \left(\frac{n+1}{2}\right) = \frac{n+1}{2n} $$
    
    \item \textbf{Utilisation des factorielles :}
    $$ \prod_{k=1}^{n} \frac{(2k-1)(2k+1)}{(2k)^2} = \frac{\prod_{k=1}^{n} (2k-1) \cdot \prod_{k=1}^{n} (2k+1)}{\left(\prod_{k=1}^{n} 2k\right)^2} $$
    On a $\prod_{k=1}^{n} (2k-1) = \frac{(2n)!}{\prod_{k=1}^{n}(2k)} = \frac{(2n)!}{2^n n!}$ et $\prod_{k=1}^{n} (2k) = 2^n n!$.
    De même, $\prod_{k=1}^{n} (2k+1) = \frac{(2n+1)!}{2^n n!}$. Le produit vaut donc :
    $$ \frac{\frac{(2n)!}{2^n n!} \cdot \frac{(2n+1)!}{2^n n!}}{(2^n n!)^2} = \frac{(2n)!(2n+1)!}{2^{4n}(n!)^4} = \frac{2n+1}{4^{2n}} \binom{2n}{n}^2 $$
    
    \item \textbf{Intégrales de Wallis :}
    \begin{enumerate}
        \item $U_0 = \int_0^{\pi/2} 1 \, dt = \frac{\pi}{2}$. \\
        $U_1 = \int_0^{\pi/2} \sin(t) \, dt = [-\cos(t)]_0^{\pi/2} = 0 - (-1) = 1$.
        \item On effectue une intégration par parties sur $U_{n+2} = \int_0^{\pi/2} \sin^{n+1}(t)\sin(t) \, dt$.
        \begin{align*}
        \text{Posons } u(t) = \sin^{n+1}(t) &\implies u'(t) = (n+1)\cos(t)\sin^n(t) \\
        \text{et } v'(t) = \sin(t) &\implies v(t) = -\cos(t)
        \end{align*}
        \begin{align*}
            U_{n+2} &= [-\sin^{n+1}(t)\cos(t)]_0^{\pi/2} + (n+1)\int_0^{\pi/2} \cos^2(t)\sin^n(t) \, dt \\
            &= 0 + (n+1)\int_0^{\pi/2} (1-\sin^2(t))\sin^n(t) \, dt \\
            &= (n+1) (U_n - U_{n+2})
        \end{align*}
        D'où $(n+2)U_{n+2} = (n+1)U_n$, et donc $U_{n+2} = \frac{n+1}{n+2}U_n$.
        
        \item Par récurrence, on obtient :
        $$ U_{2n} = \frac{2n-1}{2n} \cdot \frac{2n-3}{2n-2} \cdots \frac{1}{2} U_0 = \frac{(2n)!}{4^n(n!)^2} \frac{\pi}{2} = \frac{\pi}{2 \cdot 4^n} \binom{2n}{n} $$
        $$ U_{2n+1} = \frac{2n}{2n+1} \cdot \frac{2n-2}{2n-1} \cdots \frac{2}{3} U_1 = \frac{4^n(n!)^2}{(2n+1)!} = \frac{4^n}{(2n+1)\binom{2n}{n}} $$
    \end{enumerate}
\end{enumerate}
\end{proof}

\chapter{Systèmes linéaires}

\section{Généralités}

\subsection{Définition d'un système}

\begin{definition}
Un système de $n$ équations linéaires à $p$ inconnues $(x_1, \dots, x_p)$ est de la forme :
$$
\begin{cases}
a_{11}x_1+a_{12}x_2+\dots+a_{1p}x_p = b_1 \\
a_{21}x_1+a_{22}x_2+\dots+a_{2p}x_p = b_2 \\
\vdots \\
a_{n1}x_1+a_{n2}x_2+\dots+a_{np}x_p = b_n
\end{cases}
$$
Les coefficients $(a_{ij})$ pour $1 \le i \le n, 1 \le j \le p$ et les seconds membres $(b_i)$ pour $1 \le i \le n$ sont des constantes connues.
\end{definition}

\begin{example}
Pour $n=p=2$, le système s'écrit :
$$
\begin{cases}
a_{11}x_1 + a_{12}x_2 = b_1\\
a_{21}x_1 + a_{22}x_2 = b_2
\end{cases}
$$
\end{example}

\begin{application}[Position de deux droites dans le plan]
Un système de deux équations à deux inconnues peut être interprété comme l'intersection de deux droites $(D_1)$ et $(D_2)$ :
$$
\begin{cases}
(D_1) : \alpha x + \beta y = \gamma\\
(D_2) : ax + by = c
\end{cases}
$$
\begin{itemize}
    \item Si $(D_1)$ rencontre $(D_2)$ en un point $(x_0, y_0)$, le système $(L)$ admet une \textbf{solution unique} : $S_{(L)} = \{(x_0, y_0)\}$.
    \item Si $(D_1)$ et $(D_2)$ sont strictement parallèles, le système $(L)$ n'admet \textbf{aucune solution} : $S_{(L)} = \emptyset$.
    \item Si $(D_1) = (D_2)$, le système $(L)$ admet une \textbf{infinité de solutions} (les points de la droite) : $S_{(L)}=\{(x,y)\in \mathbb{R}^2 \mid \alpha x + \beta y = \gamma\}$.
\end{itemize}
\end{application}

\begin{example}[Exemples triviaux]
\begin{align*}
(L_1) : \begin{cases}
2x+y=1\\
x-y=2
\end{cases}
&&
(L_2) : \begin{cases}
x-2y=1\\
-2x+4y=3
\end{cases}
&&
(L_3) : \begin{cases}
x-y=1\\
-2x+2y=-2
\end{cases}
\end{align*}
\end{example}

\subsection{Résolution d'un système linéaire}

\begin{proposition}
Pour résoudre un système linéaire, on cherche à le transformer en un système \textbf{équivalent} (ayant les mêmes solutions) et "plus simple", c'est-à-dire un \textbf{système triangulaire} ou, dans le cas général, un \textbf{système échelonné}.
\end{proposition}

\begin{definition}[Système triangulaire]
Un système est dit \textbf{triangulaire} si tous les coefficients sous la diagonale principale sont nuls. Par exemple :
$$
\begin{cases}
a_{11}x_1+a_{12}x_2+\dots+a_{1n}x_n &= b_1\\
\qquad a_{22}x_2+\dots+a_{2n}x_n &= b_2\\
&\ddots\\
\qquad\qquad\qquad a_{nn}x_n &= b_n
\end{cases}
$$
Un tel système se résout en utilisant la "méthode de la remontée".
\end{definition}

\begin{definition}[Système échelonné]
Un \textbf{système échelonné} est un système linéaire dont chaque équation (ou ligne) commence par strictement plus de zéros que la précédente. Par exemple :
$$
\begin{cases}
\alpha_{11}x_1 + \alpha_{12}x_2 + \alpha_{13}x_3 + \dots + \alpha_{1n}x_n &= \gamma_1\\
\qquad 0 \qquad + \qquad 0 \qquad + \alpha_{23}x_3 + \dots + \alpha_{2n}x_n &= \gamma_2\\
\qquad 0 \qquad + \qquad 0 \qquad + \qquad 0 \qquad + \alpha_{34}x_4 + \dots &= \gamma_3\\
\ddots
\end{cases}
$$
\end{definition}

\begin{theorem}[Opérations élémentaires (Pivot de Gauss)]
\quad
\itshape
\itshape
Les opérations suivantes permettent de passer d'un système à un autre qui lui est équivalent :
\begin{enumerate}
    \item \textbf{Permutation} de deux lignes : $L_i \leftrightarrow L_j$ (avec $i \neq j$).
    \item \textbf{Multiplication} d'une ligne par un scalaire non nul $\alpha$ : $L_i \leftarrow \alpha L_i$ (avec $\alpha \neq 0$).
    \item \textbf{Ajout} à une ligne d'un multiple d'une autre ligne : $L_i \leftarrow L_i + \lambda L_j$ (avec $i \neq j$).
\end{enumerate}
\end{theorem}

\begin{example}[Résolution par pivot de Gauss]
\begin{enumerate}
    \item Soit le système :
    $$
    \begin{cases}
    x_1+2x_2+2x_3=2\\
    x_1+3x_2-2x_3=-1\\
    3x_1+5x_2+8x_3=8
    \end{cases}
    \xrightarrow[\text{L3} \leftarrow \text{L3}-3\text{L1}]{\text{L2} \leftarrow \text{L2}-\text{L1}}
    \begin{cases}
    x_1+2x_2+2x_3=2\\
    \qquad x_2-4x_3=-3\\
    \qquad -x_2+2x_3=2
    \end{cases}
    \xrightarrow{\text{L3} \leftarrow \text{L3}+\text{L2}}
    \begin{cases}
    x_1+2x_2+2x_3=2\\
    \qquad x_2-4x_3=-3\\
    \qquad \qquad -2x_3=-1
    \end{cases}
    $$
    Par remontée, on trouve $x_3 = 1/2$, puis $x_2 = -1$, et enfin $x_1=3$. D'où $S = \{(3, -1, 1/2)\}$.
    
    \item Soit le système :
    $$
    \begin{cases}
    x_1-x_2+x_3=1\\
    2x_1-x_2+x_3=3\\
    3x_1-2x_2+2x_3=4
    \end{cases}
    $$
    Après résolution, on trouve une infinité de solutions : $S = \{(2, 1+t, t) \mid t \in \mathbb{R}\}$.
\end{enumerate}
\end{example}

\section{Sous-espaces vectoriels de $\mathbb{R}^n$}

\begin{definition}[Structure de $\mathbb{R}^n$]
On munit l'ensemble $\mathbb{R}^n$ de deux lois :
\begin{itemize}
    \item L'addition vectorielle : $x+y = (x_1+y_1, \dots, x_n+y_n)$.
    \item La multiplication par un scalaire : $\lambda \cdot x = (\lambda x_1, \dots, \lambda x_n)$.
\end{itemize}
\end{definition}

\begin{definition}[Sous-espace vectoriel]
Une partie non vide $F \subseteq \mathbb{R}^n$ est un \textbf{sous-espace vectoriel} (s.e.v.) de $\mathbb{R}^n$ si elle est stable par combinaison linéaire :
$$ \forall (x, y) \in F^2, \forall \lambda \in \mathbb{R}, \quad x + \lambda y \in F $$
Cela équivaut à vérifier deux conditions :
\begin{enumerate}
    \item Le vecteur nul $\vec{0} = (0, \dots, 0)$ appartient à $F$.
    \item $F$ est stable par addition et multiplication par un scalaire.
\end{enumerate}
\end{definition}

\begin{definition}[S.E.V. engendré par une famille de vecteurs]
Soient $\vec{v_1}, \dots, \vec{v_p}$ des vecteurs de $\mathbb{R}^n$. L'ensemble de toutes leurs combinaisons linéaires est un s.e.v. de $\mathbb{R}^n$, appelé \textbf{sous-espace vectoriel engendré} par la famille $(\vec{v_1}, \dots, \vec{v_p})$. On le note :
$$ \text{Vect}(\vec{v_1}, \dots, \vec{v_p}) = \left\{ \sum_{i=1}^{p} \lambda_i \vec{v_i} \mid (\lambda_1, \dots, \lambda_p) \in \mathbb{R}^p \right\} $$
C'est le plus petit (au sens de l'inclusion) s.e.v. de $\mathbb{R}^n$ contenant tous les vecteurs $\vec{v_i}$.
\end{definition}

\begin{example}
Dans $\mathbb{R}^3$, considérons $\vec{v_1} = (1, 0, 0)$ et $\vec{v_2} = (0, 1, 0)$.
$$ \text{Vect}(\vec{v_1}, \vec{v_2}) = \{ \lambda_1(1,0,0) + \lambda_2(0,1,0) \mid \lambda_1, \lambda_2 \in \mathbb{R} \} = \{ (\lambda_1, \lambda_2, 0) \mid \lambda_1, \lambda_2 \in \mathbb{R} \} $$
C'est le plan horizontal d'équation $z=0$.
\begin{center}
\begin{tikzpicture}[scale=1.5]
    % Axes
    \draw[->, thick] (0,0,0) -- (2,0,0) node[anchor=north east]{$x$};
    \draw[->, thick] (0,0,0) -- (0,2,0) node[anchor=north west]{$y$};
    \draw[->, thick] (0,0,0) -- (0,0,2) node[anchor=south]{$z$};
    % Vecteurs
    \draw[->, ultra thick, red] (0,0,0) -- (1,0,0) node[anchor=north]{$\vec{v_1}$};
    \draw[->, ultra thick, blue] (0,0,0) -- (0,1,0) node[anchor=west]{$\vec{v_2}$};
    % Plan
    \fill[gray, opacity=0.2] (0,0,0) -- (1.5,0,0) -- (1.5,1.5,0) -- (0,1.5,0) -- cycle;
    \node at (1,1,0) [above] {Plan $z=0$};
\end{tikzpicture}
\end{center}
\end{example}

\begin{definition}[Famille libre et liée]
Une famille de vecteurs $(\vec{v_1}, \dots, \vec{v_p})$ est dite :
\begin{itemize}
    \item \textbf{libre} si la seule façon d'obtenir le vecteur nul comme combinaison linéaire est de choisir tous les coefficients nuls.
    $$ \sum_{i=1}^{p} \lambda_i \vec{v_i} = \vec{0} \implies \lambda_1 = \lambda_2 = \dots = \lambda_p = 0 $$
    \item \textbf{liée} si elle n'est pas libre. Cela signifie qu'au moins un des vecteurs peut s'exprimer comme combinaison linéaire des autres.
\end{itemize}
\end{definition}

\begin{definition}[Base et Dimension]
Soit $F$ un s.e.v. de $\mathbb{R}^n$.
\begin{itemize}
    \item Une famille $(\vec{v_1}, \dots, \vec{v_p})$ est une \textbf{famille génératrice} de $F$ si $F = \text{Vect}(\vec{v_1}, \dots, \vec{v_p})$.
    \item Une \textbf{base} de $F$ est une famille génératrice de $F$ qui est également libre.
    \item Toutes les bases d'un s.e.v. $F$ ont le même nombre de vecteurs. Ce nombre est appelé la \textbf{dimension} de $F$, notée $\dim(F)$.
\end{itemize}
\end{definition}

\section{Sous-espaces affines et structure des solutions}

\begin{definition}[Sous-espace affine]
Soit $F$ un s.e.v. de $\mathbb{R}^n$ et $\vec{a}$ un vecteur de $\mathbb{R}^n$. L'ensemble
$$ \mathcal{A} = \vec{a} + F = \{ \vec{a} + \vec{v} \mid \vec{v} \in F \} $$
est appelé un \textbf{sous-espace affine} passant par $\vec{a}$ et de direction $F$. Sa dimension est celle de sa direction : $\dim(\mathcal{A}) = \dim(F)$.
\end{definition}

\begin{proposition}[Non-unicité du point de passage]
\itshape
Le point de passage d'un sous-espace affine n'est pas unique. Si $\mathcal{A} = \vec{a} + F$, alors pour tout point $\vec{b} \in \mathcal{A}$, on a aussi $\mathcal{A} = \vec{b} + F$.
\end{proposition}

\begin{proof}
\quad
Soit $\vec{b} \in \mathcal{A}$. Par définition, il existe un vecteur $\vec{v_1} \in F$ tel que $\vec{b} = \vec{a} + \vec{v_1}$.
\begin{itemize}
    \item Montrons que $\mathcal{A} \subseteq \vec{b} + F$. Soit $\vec{x} \in \mathcal{A}$. Il existe $\vec{v_2} \in F$ tel que $\vec{x} = \vec{a} + \vec{v_2}$. Comme $\vec{a} = \vec{b} - \vec{v_1}$, on a $\vec{x} = (\vec{b} - \vec{v_1}) + \vec{v_2} = \vec{b} + (\vec{v_2} - \vec{v_1})$. Puisque $F$ est un s.e.v., $\vec{v_2} - \vec{v_1} \in F$, donc $\vec{x} \in \vec{b} + F$.
    \item Montrons que $\vec{b} + F \subseteq \mathcal{A}$. Soit $\vec{x} \in \vec{b} + F$. Il existe $\vec{v_3} \in F$ tel que $\vec{x} = \vec{b} + \vec{v_3}$. En remplaçant $\vec{b}$, on a $\vec{x} = (\vec{a} + \vec{v_1}) + \vec{v_3} = \vec{a} + (\vec{v_1} + \vec{v_3})$. Comme $\vec{v_1} + \vec{v_3} \in F$, on a $\vec{x} \in \mathcal{A}$.
\end{itemize}
Les deux inclusions prouvent l'égalité.
\end{proof}

\begin{theorem}[Unicité de la direction]
\quad
\itshape
\itshape
La direction d'un sous-espace affine est unique. Si $\mathcal{A} = \vec{a_1} + F_1 = \vec{a_2} + F_2$, alors $F_1 = F_2$.
\end{theorem}

\begin{proof}
\quad
On a $\vec{a_2} \in \mathcal{A} = \vec{a_1} + F_1$, donc il existe $\vec{v_1} \in F_1$ tel que $\vec{a_2} = \vec{a_1} + \vec{v_1}$, d'où $\vec{a_1} - \vec{a_2} = -\vec{v_1} \in F_1$. De même, $\vec{a_1} \in \mathcal{A} = \vec{a_2} + F_2$, donc $\vec{a_1} - \vec{a_2} \in F_2$.
\begin{itemize}
    \item Soit $\vec{x} \in F_2$. Alors $\vec{a_2} + \vec{x} \in \mathcal{A} = \vec{a_1} + F_1$. Il existe donc $\vec{v'} \in F_1$ tel que $\vec{a_2} + \vec{x} = \vec{a_1} + \vec{v'}$. On en déduit $\vec{x} = (\vec{a_1} - \vec{a_2}) + \vec{v'}$. Comme $\vec{a_1}-\vec{a_2} \in F_1$ et $\vec{v'} \in F_1$, leur somme $\vec{x}$ est dans $F_1$. Donc $F_2 \subseteq F_1$.
    \item L'argument est symétrique : en partant de $\vec{x} \in F_1$, on montre de la même manière que $\vec{x} \in F_2$, donc $F_1 \subseteq F_2$.
\end{itemize}
Par double inclusion, $F_1 = F_2$.
\end{proof}

\begin{example}
\quad
\begin{enumerate}
    \item Soit $\mathcal{A} = \{ (x,y,z) \in \mathbb{R}^3 \mid x+y+z=1 \}$.
    Un point $(x,y,z) \in \mathcal{A}$ vérifie $z = 1-x-y$. On peut donc l'écrire :
    $$ (x,y, 1-x-y) = (0,0,1) + (x,y,-x-y) = (0,0,1) + x(1,0,-1) + y(0,1,-1) $$
    Ainsi, $\mathcal{A} = \vec{a} + \text{Vect}(\vec{v_1}, \vec{v_2})$ avec $\vec{a}=(0,0,1)$, $\vec{v_1}=(1,0,-1)$ et $\vec{v_2}=(0,1,-1)$.
    $\mathcal{A}$ est un sous-espace affine de dimension 2 (un plan).
    
    \item Soit $D = \{ (x,y) \in \mathbb{R}^2 \mid x-2y=1 \}$.
    C'est la droite affine d'équation $x=1+2y$. On peut écrire un point de $D$ :
    $$ (1+2y, y) = (1,0) + y(2,1) $$
    Ainsi, $D = (1,0) + \text{Vect}((2,1))$. C'est une droite affine.
\end{enumerate}
\end{example}

\subsection{Structure de l'ensemble des solutions d'un système linéaire}

\begin{theorem}
\quad
Soit $(L)$ un système linéaire et $\mathcal{A}$ l'ensemble de ses solutions.
\begin{itemize}
    \item Soit $\mathcal{A}$ est vide ($\mathcal{A} = \emptyset$).
    \item Soit $\mathcal{A}$ est un sous-espace affine $\mathcal{A} = \vec{x_p} + F$, où :
    \begin{itemize}
        \item $\vec{x_p}$ est une \textbf{solution particulière} de $(L)$.
        \item $F$ est le s.e.v. des solutions du \textbf{système homogène associé} $(H)$, où tous les seconds membres sont nuls.
    \end{itemize}
\end{itemize}
\end{theorem}

\begin{proof}
\quad
Soit le système $(L): \sum_{j=1}^{p} a_{ij}x_j = b_i$ et le système homogène associé $(H): \sum_{j=1}^{p} a_{ij}x_j = 0$. Notons $F$ l'ensemble des solutions de $(H)$.
\begin{enumerate}
    \item \textbf{Montrons que $F$ est un s.e.v. de $\mathbb{R}^p$}.
    \begin{itemize}
        \item Le vecteur nul $\vec{0}$ est dans $F$ car $\sum a_{ij} \cdot 0 = 0$.
        \item Soient $\vec{y}, \vec{y'} \in F$ et $\lambda \in \mathbb{R}$.
        $\sum_{j=1}^{p} a_{ij}(y_j + \lambda y'_j) = \sum_{j=1}^{p} a_{ij}y_j + \lambda \sum_{j=1}^{p} a_{ij}y'_j = 0 + \lambda \cdot 0 = 0$. Donc $\vec{y}+\lambda\vec{y'} \in F$.
    \end{itemize}
    \item \textbf{Montrons que $\mathcal{A} = \vec{x_p} + F$}, en supposant que $\mathcal{A}$ n'est pas vide et que $\vec{x_p}$ est une solution particulière de $(L)$.
    Soit $\vec{x} \in \mathcal{A}$. On a donc $\sum a_{ij}x_j = b_i$ et $\sum a_{ij}(x_p)_j = b_i$.
    En soustrayant les deux équations, on obtient : $\sum a_{ij}(x_j - (x_p)_j) = 0$.
    Ceci signifie que le vecteur $\vec{x}-\vec{x_p}$ est une solution du système homogène, donc $\vec{x}-\vec{x_p} \in F$.
    On a donc $\vec{x} \in \vec{x_p} + F$. L'égalité est démontrée par équivalences successives.
\end{enumerate}
\end{proof}

\begin{example}
Résoudre le système $(L): \begin{cases} x+y+z+t = 4 \\ 2x+y+2z+t = 6 \\ 3x+2y+3z+2t=10 \end{cases}$.
\begin{itemize}
    \item \textbf{Solution particulière :} On remarque que $\vec{a}=(1,1,1,1)$ est une solution particulière de $(L)$.
    \item \textbf{Résolution du système homogène $(H)$ :}
    $$ (H): \begin{cases} x+y+z+t = 0 \\ 2x+y+2z+t = 0 \\ 3x+2y+3z+2t=0 \end{cases} \xrightarrow[\text{L3} \leftarrow \text{L3}-3\text{L1}]{\text{L2} \leftarrow \text{L2}-2\text{L1}} \begin{cases} x+y+z+t=0 \\ -y-t=0 \\ -y-t=0 \end{cases} \iff \begin{cases} y+t=0 \\ x+z=0 \end{cases} $$
    Les solutions sont de la forme $(x,y,-x,-y)$ avec $x,y \in \mathbb{R}$.
    L'ensemble des solutions de $(H)$ est $F = \{ x(1,0,-1,0) + y(0,1,0,-1) \mid x,y \in \mathbb{R} \} = \text{Vect}(\vec{v_1}, \vec{v_2})$ avec $\vec{v_1}=(1,0,-1,0)$ et $\vec{v_2}=(0,1,0,-1)$.
    \item \textbf{Solution générale de $(L)$ :}
    L'ensemble des solutions de $(L)$ est $\mathcal{A} = \vec{a} + F = \{ (1,1,1,1) + x(1,0,-1,0) + y(0,1,0,-1) \mid x,y \in \mathbb{R} \}$.
\end{itemize}
\end{example}

\subsection{Exercices d'application}

\begin{exercise}
\quad
Résoudre les systèmes suivants :
\begin{enumerate}
    \item Dans $\mathbb{R}^n$: $(L_1) \quad \left\{ \sum_{k=1}^{i} x_k = 2^{i+1}-2 \quad \forall i \in \{1, \dots, n\} \right.$.
    \item Dans $\mathbb{R}^{n+1}$: $(L_2) \quad \left\{ \sum_{k=0}^{n} C^i_k x_k = b_i \quad \forall i \in \{0, \dots, n\} \right.$.
\end{enumerate}
\end{exercise}

\begin{proof}[Solutions]
\quad
\begin{enumerate}
    \item Pour $(L_1)$, on procède par soustraction. Pour $i \ge 2$:
    $$ x_i = \left( \sum_{k=1}^{i} x_k \right) - \left( \sum_{k=1}^{i-1} x_k \right) = (2^{i+1}-2) - (2^i-2) = 2^{i+1}-2^i = 2^i $$
    Pour $i=1$, $x_1 = 2^{1+1}-2=2$. La formule $x_i=2^i$ fonctionne aussi.
    La solution unique est donc $S = \{ (2^1, 2^2, \dots, 2^n) \}$.
    
    \item Pour $(L_2)$, le système est triangulaire car $\binom{k}{i}=0$ si $k<i$. On peut l'écrire $\sum_{k=i}^{n} \binom{k}{i} x_k = b_i$.
    Le système homogène associé a pour unique solution le vecteur nul (par récurrence descendante).
    Il existe donc une unique solution au système $(L_2)$.
    L'idée est de poser un polynôme $P(X) = \sum_{k=0}^{n} a_k X^k$, où $(a_0, \dots, a_n)$ est la solution recherchée.
    On montre que $P(X+1) = \sum_{i=0}^{n} b_i X^i$. En posant $Q(X) = \sum b_i X^i$, on a $P(X)=Q(X-1)$.
    Par identification des coefficients de $P(X) = \sum_{j=0}^{n} b_j (X-1)^j$, on trouve la solution unique:
    $$ a_k = \sum_{j=k}^{n} (-1)^{j-k} \binom{j}{k} b_j \quad \forall k \in \{0, \dots, n\} $$
\end{enumerate}
\end{proof}

\end{document}