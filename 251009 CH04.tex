\documentclass[12pt, a4paper]{report}

\usepackage{elegant-cours}

\begin{document}

\MaPageDeGarde{
    Chapitre IV : Suites numériques
}{
    Rédigé par Samy Youssoufine
}{
    ./assets/logo.png
}{}

\tableofcontents
\clearpage

\chapter{Généralités sur les suites réelles}

\section{Définitions des suites usuelles}

\begin{definition}[Suite réelle]
Soit $n_0 \in \mathbb{N}$. On appelle \textbf{suite réelle} toute application $U$ de l'ensemble $\{k \in \mathbb{N} \mid k \ge n_0\}$ dans $\mathbb{R}$.
On note $U(n) = U_n$, et la suite est notée $(U_n)_{n \ge n_0}$. L'ensemble des suites réelles est noté $\mathbb{R}^{\mathbb{N}}$.
\end{definition}

\begin{remark}
    On note l'ensemble des applications d'un ensemble E vers un ensemble F par $F^E$. Ainsi, $\mathbb{R}^{\mathbb{N}}$ est l'ensemble des applications de $\mathbb{N}$ dans $\mathbb{R}$.
\end{remark}

\begin{remark}
\begin{elegantlist}
    \item Si $n_0=0$, on note plus simplement la suite $(U_n)_n$.
    \item $(U_n)_n$ désigne la suite en tant qu'objet (l'application).
    \item $U_n$ désigne le terme de rang $n$ de la suite (la valeur de l'application en $n$).
\end{elegantlist}
\end{remark}

\begin{example}\quad
\begin{itemize}
    \item $(e^{-n})_{n \in \mathbb{N}}$ est une suite réelle.
    \item $(\frac{1}{\sqrt{n-1}})_{n \ge 2}$ est une suite réelle.
\end{itemize}
\end{example}

\begin{definition}[Suite arithmétique]
Une suite $(U_n)_n$ est dite arithmétique s'il existe un réel $r$, appelé \textbf{raison}, tel que pour tout entier $n \in \mathbb{N}$ :
$$ U_{n+1} = U_n + r $$
\end{definition}

\begin{property}[Terme général d'une suite arithmétique]
Si $(U_n)_n$ est une suite arithmétique de raison $r$, alors pour tous entiers $n \ge p$ :
\begin{keyformula}
$$U_n = U_p + (n-p) \cdot r$$
\end{keyformula}
\end{property}

\begin{definition}[Suite géométrique]
Une suite $(U_n)_n$ est dite géométrique s'il existe un réel $q \in \mathbb{R}^*$, appelé \textbf{raison}, tel que pour tout entier $n \in \mathbb{N}$ :
$$ U_{n+1} = q \cdot U_n $$
\end{definition}

\begin{property}[Terme général d'une suite géométrique]
Si $(U_n)_n$ est une suite géométrique de raison $q$, alors pour tous entiers $n \ge p$ :
\begin{keyformula}
$$U_n = U_p \cdot q^{n-p}$$
\end{keyformula}
\end{property}

\begin{definition}[Suite arithmético-géométrique]
Une suite $(U_n)_n$ est dite arithmético-géométrique s'il existe deux réels $(a,b)$ avec $a \in \mathbb{R} \setminus \{0,1\}$ et $b \in \mathbb{R}$, tels que pour tout $n \in \mathbb{N}$ :
$$ U_{n+1} = a U_n + b $$
\end{definition}

\begin{property}[Terme général d'une suite arithmético-géométrique]
\begin{methodbox}
\textbf{Méthode :} On cherche le point fixe $l$ tel que $l=al+b$, soit $l = \frac{b}{1-a}$.
\\[0.3em]
La suite auxiliaire $(V_n)_n$ définie par $V_n = U_n - l$ est géométrique de raison $a$.
\end{methodbox}
On en déduit le terme général de $(U_n)_n$ :
\begin{keyformula}
$$\forall n \in \mathbb{N}, \quad U_n = l + a^n(U_0 - l)$$
\end{keyformula}
\end{property}

\begin{definition}[Suite récurrente linéaire d'ordre 2]
Une suite $(U_n)_n$ est dite récurrente linéaire d'ordre 2 s'il existe deux réels $(a,b)$ tels que pour tout $n \in \mathbb{N}$ :
$$ U_{n+2} = a U_{n+1} + b U_n $$
La suite est entièrement déterminée par la donnée de ses deux premiers termes $U_0$ et $U_1$.
\end{definition}

\begin{theorem}[Solution d'une suite récurrente linéaire d'ordre 2]
Pour trouver le terme général de la suite $(U_n)_n$, on résout son \textbf{équation caractéristique} :
$$ x^2 - ax - b = 0 \quad (*) $$
Soit $\Delta$ son discriminant. Trois cas se présentent :
\begin{itemize}
    \item \textbf{Cas 1 : $\Delta > 0$} \\
    L'équation $(*)$ admet deux solutions réelles distinctes $r_1$ et $r_2$. Le terme général est de la forme :
    \begin{keyformula}[colframe=green!70, colbacktitle=green!70]
    $$U_n = \alpha r_1^n + \beta r_2^n$$
    \end{keyformula}
    où $(\alpha, \beta) \in \mathbb{R}^2$ sont déterminés par les conditions initiales $U_0$ et $U_1$.

    \item \textbf{Cas 2 : $\Delta = 0$} \\
    L'équation $(*)$ admet une solution réelle double $r_0$. Le terme général est de la forme :
    \begin{keyformula}[colframe=orange!70, colbacktitle=orange!70]
    $$U_n = (\alpha n + \beta) r_0^n$$
    \end{keyformula}
    où $(\alpha, \beta) \in \mathbb{R}^2$ sont déterminés par les conditions initiales.

    \item \textbf{Cas 3 : $\Delta < 0$} \\
    L'équation $(*)$ admet deux solutions complexes conjuguées $z = re^{i\theta}$ et $\bar{z} = re^{-i\theta}$. Le terme général est de la forme :
    \begin{keyformula}[colframe=purple!70, colbacktitle=purple!70]
    $$U_n = r^n (\alpha \cos(n\theta) + \beta \sin(n\theta))$$
    \end{keyformula}
    où $(\alpha, \beta) \in \mathbb{R}^2$ sont déterminés par les conditions initiales.
\end{itemize}
\end{theorem}

\section{Suites bornées}

\begin{definition}[Suite bornée]\quad
Soit $(U_n)_n \in \mathbb{R}^{\mathbb{N}}$.
\begin{itemize}
    \item $(U_n)_n$ est dite \textbf{majorée} ssi $\exists M \in \mathbb{R}, \forall n \in \mathbb{N}, U_n \le M$. \\
    (i.e. l'ensemble $\{U_n, n \in \mathbb{N}\}$ est majoré).
    \item $(U_n)_n$ est dite \textbf{minorée} ssi $\exists m \in \mathbb{R}, \forall n \in \mathbb{N}, U_n \ge m$. \\
    (i.e. l'ensemble $\{U_n, n \in \mathbb{N}\}$ est minoré).
    \item $(U_n)_n$ est dite \textbf{bornée} ssi elle est majorée et minorée.
\end{itemize}
\end{definition}

\begin{remark}
On note l'ensemble des suites réelles bornées $\mathcal{B}(\mathbb{R})$.
\end{remark}

\begin{property}
$(U_n)_n \in \mathcal{B}(\mathbb{R}) \iff \exists \alpha > 0, \forall n \in \mathbb{N}, |U_n| \le \alpha$.
\end{property}

\begin{example}\quad
\begin{itemize}
    \item $(-n^2)_n$ est majorée (par 0) et non minorée.
    \item $(\sqrt{n})_n$ est minorée (par 0) et non majorée.
    \item $((-2)^n)_n$ n'est ni majorée, ni minorée.
    \item $(\frac{(-1)^n}{n+1})_n$, $(\cos(n))_n$ sont des suites bornées.
\end{itemize}
\end{example}

\begin{proposition}[Stabilité par opérations]
Soient $(U_n)_n, (V_n)_n \in \mathcal{B}(\mathbb{R})$ et $\lambda \in \mathbb{R}$.
\begin{enumerate}
    \item $(U_n)_n + \lambda(V_n)_n \in \mathcal{B}(\mathbb{R})$
    \item $(U_n)_n \cdot (V_n)_n \in \mathcal{B}(\mathbb{R})$
\end{enumerate}

\begin{proof}
Soit $\Pi, \Pi' \ge 0$ tels que $\forall n \in \mathbb{N}, |U_n| \le \Pi$ et $|V_n| \le \Pi'$. \\
Alors $\forall n \in \mathbb{N}, |U_n + \lambda V_n| \le |U_n| + |\lambda||V_n| \le \Pi + |\lambda|\Pi'$. \\
Et $|U_n \cdot V_n| = |U_n| \cdot |V_n| \le \Pi \cdot \Pi'$.
\end{proof}
\end{proposition}

\begin{proposition}
Si $(U_n)_n, (V_n)_n \in \mathcal{B}(\mathbb{R})$, alors $(\max(U_n, V_n))_n$ et $(\min(U_n, V_n))_n$ sont dans $\mathcal{B}(\mathbb{R})$.
\end{proposition}

\begin{consequence}
On peut exprimer le max et le min à l'aide de la valeur absolue :
\begin{keyformula}[colframe=teal!70, colbacktitle=teal!70]
\begin{align}
\max(U_n, V_n) &= \frac{1}{2}(U_n+V_n+|U_n-V_n|) \\
\min(U_n, V_n) &= \frac{1}{2}(U_n+V_n-|U_n-V_n|)
\end{align}
\end{keyformula}
\end{consequence}


\section{Monotonie d'une suite}

\begin{definition}[Monotonie]
Soit $(U_n)_n \in \mathbb{R}^{\mathbb{N}}$.
\begin{itemize}
    \item $(U_n)_n$ est dite \textbf{croissante} (resp. \textbf{strictement croissante}) ssi $\forall n \in \mathbb{N}, U_{n+1} \ge U_n$ (resp. $U_{n+1} > U_n$).
    \item $(U_n)_n$ est dite \textbf{décroissante} (resp. \textbf{strictement décroissante}) ssi $\forall n \in \mathbb{N}, U_{n+1} \le U_n$ (resp. $U_{n+1} < U_n$).
    \item $(U_n)_n$ est dite \textbf{monotone} ssi elle est croissante ou décroissante.
\end{itemize}
\end{definition}

\begin{remark}[Méthodes pour étudier la monotonie]
\begin{methodbox}
\textbf{Deux méthodes principales :}
\begin{elegantlist}
    \item Pour étudier la monotonie de $(U_n)_n$, on étudie le signe de $U_{n+1} - U_n$.
    \item Si $\forall n, U_n \neq 0$ et de signe constant, on peut étudier la position de $\frac{U_{n+1}}{U_n}$ par rapport à 1, ou le signe de $\frac{U_{n+1}}{U_n} -1$.
\end{elegantlist}
\end{methodbox}
\end{remark}

\begin{example}\quad
\begin{enumerate}
    \item $ (n^2)_n, (\sqrt{n})_n $ sont croissantes. $\doubleheadrightarrow$
    \item Soit la suite $(U_n)_{n \ge 1}$ définie par $U_n = \sum_{k=1}^{n} \frac{1}{n+k}$. Pour étudier sa monotonie, calculons $U_{n+1}-U_n$.
    \begin{align*}
        U_{n+1}-U_n &= \sum_{k=1}^{n+1} \frac{1}{n+1+k} - \sum_{k=1}^{n} \frac{1}{n+k} \\
        &= \sum_{k=2}^{n+2} \frac{1}{n+k} - \sum_{k=1}^{n} \frac{1}{n+k} \\
        &= \left( \sum_{k=2}^{n} \frac{1}{n+k} + \frac{1}{2n+1} + \frac{1}{2n+2} \right) - \left( \frac{1}{n+1} + \sum_{k=2}^{n} \frac{1}{n+k} \right) \\
        &= \frac{1}{2n+1} + \frac{1}{2(n+1)} - \frac{1}{n+1} \\
        &= \frac{1}{2n+1} - \frac{1}{2(n+1)} \\
        &= \frac{2(n+1) - (2n+1)}{2(n+1)(2n+1)} \\
        &= \frac{1}{2(n+1)(2n+1)} > 0
    \end{align*}
    Donc la suite $(U_n)_{n \ge 1}$ est (strictement) croissante.
\end{enumerate}
\end{example}

\section{Suites extraites}
\begin{definition}[Suite extraite]
    Soit $(U_n)_n \in \mathbb{R}^{\mathbb{N}}$ et $(\varphi(n))_n$ une suite strictement croissante d'entiers naturels. La suite $(V_n)_n$ définie par $V_n = U_{\varphi(n)}$ est appelée \textbf{suite extraite} (ou sous-suite) de $(U_n)_n$ par la suite d'extraction $(\varphi(n))_n$. $\varphi$ est appelée \textbf{fonction d'extraction} ou l'extractrice ($\varphi:\mathbb{N}\rightarrow\mathbb{N}$).
\end{definition}

\begin{property}\quad
    \begin{itemize}
        \item Si $(U_n)_n$ est convergente vers $L$, alors toute suite extraite $(V_n)_n$ de $(U_n)_n$ est convergente vers $L$.
        \item L'extractrice $(\varphi(n))_n$ vérifie l'inégalité suivante : $\forall n \in \mathbb{N}, \varphi(n) \ge n$.
        \item Si $(U_n)_n$ est croissante (resp. décroissante), alors toute suite extraite $(V_n)_n$ de $(U_n)_n$ est croissante (resp. décroissante).
        \item Si $(U_n)_n$ est majorée (resp. minorée, bornée), alors toute suite extraite $(V_n)_n$ de $(U_n)_n$ est majorée (resp. minorée, bornée).
    \end{itemize}
\end{property}

\section{Convergence d'une suite}
\begin{definition}[Convergence d'une suite]
Soit $(U_n)_n \in \mathbb{R}^{\mathbb{N}}$ et $L \in \mathbb{R}$. On dit que $(U_n)_n$ \textbf{converge} vers $L$ (ou que $L$ est la \textbf{limite} de $(U_n)_n$) ssi :
\begin{importantbox}[title=\textbf{Définition fondamentale de la convergence}]
$\forall \varepsilon > 0, \exists n_0 \in \mathbb{N}, \forall n \ge n_0, |U_n - L| < \varepsilon \quad (*)$
\end{importantbox}
On note $\lim_{n \to +\infty} U_n = L$ ou $U_n \to L$ quand $n \to +\infty$.
\end{definition}

\begin{remark}
    Le $n_0$ s'appelle le rang à partir duquel l'inégalité $(*)$ est vérifiée.
\end{remark}

\begin{property}
    Soit $(U_n)_n \in \mathbb{R}^{\mathbb{N}}$ et $L \in \mathbb{R}$.
    Si $(U_n)_n$ converge vers $L$, alors $L$ est unique.
    \begin{proof}
    Soit $L_1, L_2 \in \mathbb{R}$ tels que $U_n \to L_1$ et $U_n \to L_2$.\\
    Soit $\varepsilon > 0$. Par définition de la convergence, il existe $n_1, n_2 \in \mathbb{N}$ tels que :
    \begin{align*}
        \forall n \ge n_1, |U_n - L_1| < \varepsilon \\
        \forall n \ge n_2, |U_n - L_2| < \varepsilon
    \end{align*}
    Soit $n_0 = \max(n_1, n_2)$. Alors pour tout $n \ge n_0$ :
    \begin{align*}
        |L_1 - L_2| &= |L_1 - U_n + U_n - L_2| \\
        &\le |L_1 - U_n| + |U_n - L_2| \\
        &< \varepsilon + \varepsilon = 2\varepsilon
    \end{align*}
    Comme cette inégalité est vraie pour tout $\varepsilon > 0$, on en déduit que $L_1 = L_2$ (on peut se référer aux propriétés étudiées en Chapitre 3 sur la borne inférieure et la borne supérieure).
\end{proof}
\end{property}

\begin{definition}[Suite divergente]\quad
\begin{itemize}
    \item Si $(U_n)_n$ n'admet pas de limite finie, on dit que $(U_n)_n$ est \textbf{divergente}.
    \item On dit que $(U_n)_n$ \textbf{diverge vers $+\infty$} ssi :
    $ \forall A > 0, \exists n_0 \in \mathbb{N}, \forall n \ge n_0, U_n > A $ et on note $\lim_{n \to +\infty} U_n = +\infty$.
    \item On dit que $(U_n)_n$ \textbf{diverge vers $-\infty$} ssi :
    $ \forall A < 0, \exists n_0 \in \mathbb{N}, \forall n \ge n_0, U_n < A $ et on note $\lim_{n \to +\infty} U_n = -\infty$.
\end{itemize}
    
\end{definition}

\begin{property}
Soient $(U_n)_n \in \mathbb{R}^{\mathbb{N}}$ et $l\in\bar{\mathbb{R}}=\mathbb{R}\bigcup\{+\infty,-\infty\}$.
\begin{itemize}
    \item Si $(U_n)_n$ converge vers $l$, alors toute suite extraite $(V_n)_n$ de $(U_n)_n$ converge vers $l$. On écrit : $U_n \to l \implies \forall (U_{\varphi(n)})_n \text{ sous-suite de } (U_n)_n, (U_{\varphi(n)})_n \to l$. L'équivalence est valable.
    \item Un cas particulier intéressant : $\begin{cases}
        U_{2n} \to l \\
        U_{2n+1} \to l 
    \end{cases} \iff U_n \to l$.
    \item Preuve à refaire (ou à revoir...).
\end{itemize}
\end{property}

\begin{remark}
    \quad
    \begin{itemize}
        \item Soit $(U_n)_n \in \mathbb{R}^{\mathbb{N}}$. Si $\exists (U_{\varphi(n)})_n$ une suite extraite de $(U_n)_n$ qui n'admet pas de limite, alors $(U_n)_n$ n'admet pas de limite.
        \item Si $(U_n)_n$ admet deux suites extraites $(U_{\varphi(n)})_n$ et $(U_{\psi(n)})_n$ qui convergent vers deux limites distinctes, alors $(U_n)_n$ n'admet pas de limite. En d'autres termes, 
    \end{itemize}
\end{remark}

\begin{exercise}
Étudier la convergence des suites : $(\cos(n\theta))_n$ et $(\sin(n\theta))_n$ où $\theta \in \mathbb{R}$.
\end{exercise}

\begin{property}
Soient $(U_n)_n \in \mathbb{R}^{\mathbb{N}}$ et $l\in\mathbb{R}$.
\begin{itemize}
    \item $U_n \to 0 \iff |U_n| \to 0$.
    \item $U_n \to l \iff |U_n| \to |l|$.
    \item $(U_n)_n$ convergente $\implies (U_n)_n$ bornée.
    \item Si $U_n\to l$ avec $l\in]a,b[$, alors $\exists n_0 \in \mathbb{N}, \forall n \ge n_0, U_n \in ]a,b[$.
\end{itemize}

\begin{proof}
\textbf{Preuve de ($U_n \to 0 \iff |U_n| \to 0$)}

La définition de la convergence vers 0 est : $\forall \varepsilon > 0, \exists n_0 \in \mathbb{N}, \forall n \ge n_0, |X_n - 0| < \varepsilon$.
L'expression $|X_n - 0|$ se simplifie en $|X_n|$.
Ainsi, $U_n \to 0$ se traduit par $|U_n| < \varepsilon$ (à partir d'un certain rang), et $|U_n| \to 0$ se traduit par $||U_n|-0| < \varepsilon$, ce qui est aussi $|U_n| < \varepsilon$. Les deux affirmations sont donc identiques.

\vspace{1em}
\textbf{Preuve de ($U_n \to l \implies |U_n| \to |l|$)}

L'implication inverse est fausse, comme le montre le contre-exemple $U_n=(-1)^n$ où $|U_n| \to 1$ mais $(U_n)_n$ diverge. Pour l'implication directe, on utilise l'inégalité triangulaire inversée : $||U_n| - |l|| \le |U_n - l|$.
Soit $\varepsilon > 0$. Comme $U_n \to l$, il existe un rang $n_0$ tel que pour tout $n \ge n_0$, on a $|U_n - l| < \varepsilon$. En combinant les deux inégalités, on obtient que pour $n \ge n_0$, $||U_n| - |l|| < \varepsilon$. Ceci prouve bien que $|U_n| \to |l|$.

\vspace{1em}
\textbf{Preuve que toute suite convergente est bornée}

Soit $(U_n)_n$ une suite qui converge vers $l$. En appliquant la définition de la convergence pour $\varepsilon=1$, on sait qu'il existe un rang $n_0$ tel que pour tout $n \ge n_0$, on a $|U_n - l| < 1$. Ceci implique que $l-1 < U_n < l+1$ pour $n \ge n_0$.
Les termes de la suite sont donc bornés à partir de ce rang $n_0$.
L'ensemble des premiers termes $\{U_0, U_1, \dots, U_{n_0-1}\}$ est un ensemble fini de nombres réels, il est donc également borné.
Par conséquent, la suite entière est bornée.

\vspace{1em}
\textbf{Preuve de la stabilité dans un intervalle ouvert}

On suppose $U_n \to l$ avec $l \in ]a, b[$. Puisque $l$ est strictement entre $a$ et $b$, la distance de $l$ à chaque borne est strictement positive. Posons $\varepsilon = \min(l-a, b-l)$. On a bien $\varepsilon > 0$.
Par définition de la convergence, pour ce $\varepsilon$, il existe un rang $n_0$ tel que pour tout $n \ge n_0$, on a $|U_n - l| < \varepsilon$, soit $l-\varepsilon < U_n < l+\varepsilon$.
Par construction, $a \le l-\varepsilon$ et $l+\varepsilon \le b$.
On en déduit que pour tout $n \ge n_0$, on a $a < U_n < b$.
\end{proof}
\end{property}



\begin{property}
    Soient $(U_n)_n, (V_n)_n \in \mathbb{R}^{\mathbb{N}}$ et $l, l' \in \mathbb{R}$ et $l \in \mathbb{R}$. \\Si $V_n \to 0$ et $\exists p \in \mathbb{N}, \forall n \ge p: |U_n - l| \le |V_n|$, alors $U_n \to l$.
\begin{proof}
    On a $V_n \to 0 \iff \forall \varepsilon > 0, \exists n_0 \in \mathbb{N}, \forall n \ge n_0, |V_n| < \varepsilon$. \\Donc $\forall n \ge \max(n_0, p)$, on a $|U_n - l| \le |V_n| < \varepsilon$. D'où la conclusion.
\end{proof}

\end{property}




\begin{property}
    Soient $(U_n)_n, (V_n)_n \in \mathbb{R}^{\mathbb{N}}$ tels que $\exists p\in\mathbb{N},\forall n\ge p:U_n\le V_n$.
    \\Alors $\begin{cases}
        U_n \to +\infty \implies V_n \to +\infty \\
        V_n \to -\infty \implies U_n \to -\infty
    \end{cases}$
\begin{proof}
    Soit $A > 0$. Par hypothèse, il existe $n_0 \in \mathbb{N}$ tel que $\forall n \ge n_0, U_n > A$. Donc $\forall n \ge \max(n_0, p), V_n \ge U_n > A$. D'où la conclusion.
    On procède de même pour la deuxième implication (on utilise $-V_n$ et $-U_n$).
\end{proof}

\end{property}

\begin{property}[Règle de d'Alembert]
    Soit $(U_n)_n \in {(\mathbb{R}^*)}^{\mathbb{N}}$ tel que $|\frac{U_{n+1}}{U_n}|\to l \in [0;+\infty[$.
    \begin{methodbox}[title=\textbf{Critères de convergence}]
    \begin{elegantlist}
        \item Si $l < 1$, alors $U_n \to 0$.
        \item Si $l > 1$, alors $|U_n| \to +\infty$.
        \item Si $l = 1$, aucune conclusion n'est possible.
    \end{elegantlist}
    \end{methodbox}
    \begin{attention}
    Attention : Le cas $l = 1$ ne permet aucune conclusion ! Il faut alors utiliser d'autres méthodes.
    \end{attention}
\end{property}

\begin{example}
    On pose $U_n = \frac{n!}{n^n}$. Étudier la convergence de $(U_n)_n$.
    \\
    On a $\frac{U_{n+1}}{U_n} = \frac{(n+1)!}{(n+1)^{n+1}} \cdot \frac{n^n}{n!} = \frac{(n+1)n^n}{(n+1)^{n+1}} = \frac{n^n}{(n+1)^n} = \left(\frac{n}{n+1}\right)^n$. \\
    Or $\left(\frac{n}{n+1}\right)^n = \left(\frac{n+1}{n}\right)^{-n} = \left(1 + \frac{1}{n}\right)^{-n} \to e^{-1} < 1$. \\
    Donc, par la règle de d'Alembert, $U_n \to 0$
\end{example}

\begin{remark}
    On a : $\forall x\in\mathbb{R}: \left(1+\frac{x}{n}\right)^n\to e^x$, parce que : $\left(1+\frac{x}{n}\right)^n = e^{n\ln\left(1+\frac{x}{n}\right)} = \exp(x\frac{\ln(1+\frac{x}{n})}{\frac{x}{n}}) \to e^x$.
\end{remark}

\begin{property}
    Soient $(U_n)_n, (V_n)_n \in {\mathbb{R}}^{\mathbb{N}}$ tels que $\exists p\in\mathbb{N},\forall n\ge p:U_n\le V_n$ et $\begin{cases}
        U_n \to l \in \mathbb{R}\\
        V_n \to l' \in \mathbb{R}
    \end{cases}$
    \\Alors $l \le l'$.
\begin{proof}
    Soit $\varepsilon > 0, \exists n_1,n_2 \in \mathbb{N}$ tels que :\\$\forall n \ge n_1: -\frac{\varepsilon}{2} +l < U_n < l + \frac{\varepsilon}{2}$
    \\$\forall n \ge n_2: -\frac{\varepsilon}{2} +l' < V_n < l' + \frac{\varepsilon}{2}$
    \\Donc pour tout $n \ge \max(n_1,n_2)$, on a :\\
    $-\frac{\varepsilon}{2} +l < U_n < l + \frac{\varepsilon}{2} \le V_n < l' + \frac{\varepsilon}{2}$
    \\$\implies l-l'<\varepsilon$ ($\forall \varepsilon > 0$)\\
    $\implies l-l' \leq \inf(\mathbb{R}_+^*) = 0$
    D'où $l \le l'$.
\end{proof}

\end{property}



\begin{theorem}[Théorème des gendarmes]
    Soient $(U_n)_n, (V_n)_n, (W_n)_n \in {\mathbb{R}}^{\mathbb{N}}$ tels que $\exists p\in\mathbb{N},\forall n\ge p:U_n\le V_n\le W_n$ et $\begin{cases}
        U_n \to l \in \mathbb{R}\\
        W_n \to l' \in \mathbb{R}
    \end{cases}$ avec $l = l'$. Alors $V_n \to l$.
\begin{proof}
    Soit $\varepsilon > 0, \exists n_1,n_2 \in \mathbb{N}$ tels que :\\$\forall n \ge n_1: -\varepsilon +l < U_n < l + \varepsilon$
    \\$\forall n \ge n_2: -\varepsilon +l' < W_n < l' + \varepsilon$
    \\Donc pour tout $n \ge \max(n_1,n_2)$, on a :\\
    $-\varepsilon +l < U_n < l + \varepsilon \le V_n \le W_n < l' + \varepsilon = l + \varepsilon$
    \\$\implies |V_n - l| < \varepsilon$ ($\forall \varepsilon > 0$)
    \\D'où le théorème.
\end{proof}

\end{theorem}

\begin{example}
    On a : $\forall n \ge 1: U_n=\sum_{k=1}^{n} \frac{n}{k+n^2}$.\\Calculons $\lim_{n \to +\infty} U_n$.
    \\On a : $\forall n \ge 1, \forall k \in [1,n], 1+n^2 \le k+n^2 \le n+n^2$. Donc : $\frac{n}{n+n^2} \le \frac{n}{k+n^2} \le \frac{n}{1+n^2}$.
    \\Donc : $\frac{n^2}{n+n^2} \le U_n \le \frac{n^2}{1+n^2}$.
    \\Or : $\frac{n^2}{n+n^2} = \frac{1}{1+\frac{1}{n}} \to 1$ et $\frac{n^2}{1+n^2} = \frac{1}{1+\frac{1}{n^2}} \to 1$.
    \\Donc, par le théorème des gendarmes, $U_n \to 1$.
\end{example}

\begin{consequence}
    Soient $(U_n)_n, (V_n)_n \in {\mathbb{R}}^{\mathbb{N}}$ tels que $V_n \in \mathbb{B(R)}$ et $U_n \to 0$. Alors $U_n V_n \to 0$.
\end{consequence}

\begin{property}
    Soient $(U_n)_n, (V_n)_n \in {\mathbb{R}}^{\mathbb{N}}$ tels que $\begin{cases}
        U_n \to l \in \mathbb{R}\\
        V_n \to l' \in \mathbb{R}
    \end{cases}$
    \\Alors :
    \begin{itemize}
        \item $U_n + \alpha V_n \to l + \alpha l'$ pour tout $\alpha \in \mathbb{R}^*$
        \item $\lambda U_n \to \lambda l$ pour tout $\lambda \in \mathbb{R}$
        \item $U_n V_n \to l\cdot l'$
        \item Si $l' \neq 0$, alors $\exists n_0\in\mathbb{N},\forall n\ge n_0:V_n\neq 0$ et $\frac{1}{V_n} \to \frac{1}{l'}$
    \end{itemize}

    \begin{proof}
    \begin{itemize}
        \item Soit $\alpha \in \mathbb{R}^*$. Soit $\varepsilon > 0$. Par hypothèse, il existe $n_1, n_2 \in \mathbb{N}$ tels que :\\$\forall n \ge n_1: |U_n - l| < \frac{\varepsilon}{2}$ et $\forall n \ge n_2: |V_n - l'| < \frac{\varepsilon}{2|\alpha|}$. Donc pour tout $n \ge \max(n_1,n_2)$, on a :\\
        $|U_n + \alpha V_n - (l + \alpha l')| \le |U_n - l| + |\alpha||V_n - l'| < \frac{\varepsilon}{2} + \frac{\varepsilon}{2} = \varepsilon$.
        \item Soit $\lambda \in \mathbb{R}$. Si $\lambda = 0$, la conclusion est évidente. Sinon, on applique le point précédent avec $\alpha = \lambda$.
        % \item Soit $\varepsilon > 0$. Par hypothèse, il existe $n_1, n_2 \in \mathbb{N}$ tels que :\\$\forall n \ge n_1: |U_n - l| < \min(1, \frac{\varepsilon}{2(|l'|+1)})$ et $\forall n \ge n_2: |V_n - l'| < \min(1, \frac{\varepsilon}{2(|l|+1)})$. Donc pour tout $n \ge \max(n_1,n_2)$, on a :\\
        % $|U_n V_n - ll'| = |U_n V_n - l V_n + l V_n - ll'| \le |V_n||U_n - l| + |l||V_n - l'|$. \\
        % Or $|U_n - l| < 1 \implies |U_n| < |l| + 1$ et $|V_n - l'| < 1 \implies |V_n| < |l'| + 1$. Donc :\\
        % $|U_n V_n - ll'| < (|l'| + 1) \cdot \frac{\varepsilon}{2(|l'|+1)} + |l| \cdot \frac{\varepsilon}{2(|l|+1)} < \varepsilon$.
        % \item Si $l' \neq 0$, il existe $n_0 \in \mathbb{N}$ tel que $\forall n \ge n_0: |V_n - l'| < \frac{|l'|}{2} \implies |V_n| > \frac{|l'|}{2} > 0 \implies V_n \neq 0$. Soit $\varepsilon > 0$. Par hypothèse, il existe $n_1, n_2 \in \mathbb{N}$ tels que :\\$\forall n \ge n_1: |U_n - l| < \min(1, \frac{\varepsilon |l'|^2}{2})$ et $\forall n \ge n_2: |V_n - l'| < \min(1, \frac{\varepsilon |l'|^2}{2})$. Donc pour tout $n \ge \max(n_0,n_1,n_2)$, on a :\\
        % $|\frac{1}{V_n} - \frac{1}{l'}| = |\frac{l' - V_n}{l' V_n}| = \frac{|V_n - l'|}{|l'||V_n|} < \frac{\varepsilon |l'|^2 / 2}{|l'| \cdot |l'|/2} = \varepsilon$.
        \item Les autres preuves sont dans le cahier, manuscrites.
    \end{itemize}
\end{proof}
\end{property}

\begin{property}
    Soient $(U_n)_n, (V_n)_n \in {\mathbb{R}}^{\mathbb{N}}$.
    \\Alors :
    \begin{enumerate}
        \item $(U_n)_n$ est minorée et $V_n\to +\infty \implies U_n + V_n \to +\infty$
        \item $(V_n)_n$ est majorée et $U_n\to -\infty \implies U_n + V_n \to -\infty$
    \end{enumerate}
    \begin{proof}
    On a : $\exists m \in \mathbb{R},\forall n\in \mathbb{N},U_n\ge m$.\\
    Donc : $\forall n \in \mathbb{N}, U_n + V_n \ge m + V_n \to +\infty$.
    D'où $U_n+V_n\to +\infty$.
\end{proof}
\end{property}

\begin{theorem}
    Soit $(U_n)_n \in {\mathbb{R}}^{\mathbb{N}}$.
    \begin{enumerate}
        \item Si $(U_n)_n$ est croissante et majorée, alors $(U_n)_n$ converge vers $\sup(\{U_n, n \in \mathbb{N}\})$.
        \item Si $(U_n)_n$ est décroissante et minorée, alors $(U_n)_n$ converge vers $\inf(\{U_n, n \in \mathbb{N}\})$.
        \item Si $(U_n)_n$ est croissante non majorée, alors $U_n \to +\infty$.
        \item Si $(U_n)_n$ est décroissante non minorée, alors $U_n \to -\infty$.
        \item Toute suite monotone admet une limite dans $\bar{\mathbb{R}}$. Cela \textbf{ne veut pas dire que toute suite monotone converge}.
    \end{enumerate}
    \begin{proof}\quad
    \begin{enumerate}
        \item Soit $M = \sup(\{U_n, n \in \mathbb{N}\})$. Soit $\varepsilon > 0$. Par définition de la borne supérieure, il existe $n_0 \in \mathbb{N}$ tel que $M - \varepsilon < U_{n_0} \le M$. Donc pour tout $n \ge n_0$, on a : $M - \varepsilon < U_{n_0} \le U_n \le M$. D'où $|U_n - M| < \varepsilon$.
        \item On procède de même en utilisant la borne inférieure.
    \end{enumerate}
\end{proof}
\end{theorem}

\begin{exercise}
    Calculer la limite des suites suivantes (si elle existe) :
    \begin{itemize}
        \item $U_0 \in \mathbb{R},\forall n \in \mathbb{N}:U_{n+1}=U_n+e^{-U_n}$
        \item $U_0 \in \mathbb{R},\forall n \in \mathbb{N}:U_{n+1}=U_n-U_n^2$
    \end{itemize}
\end{exercise}

\section{Suites adjacentes}
\begin{definition}
    Soient $(U_n)_n, (V_n)_n \in {\mathbb{R}}^{\mathbb{N}}$. On dit que $(U_n)_n$ et $(V_n)_n$ sont \textbf{adjacentes} ssi :
    \begin{itemize}
        \item $(U_n)_n$ est croissante et $(V_n)_n$ est décroissante (ou l'inverse)
        \item $\lim_{n \to +\infty} (V_n - U_n) = 0$
    \end{itemize}
\end{definition}

\begin{theorem}
    Soient $(U_n)_n, (V_n)_n \in {\mathbb{R}}^{\mathbb{N}}$ deux suites adjacentes telles que $U_n \nearrow$ et $V_n \searrow$.
    \\Alors :
    \begin{itemize}
        \item Les deux suites $(U_n)_n$ et $(V_n)_n$ convergent vers une même limite $l \in \mathbb{R}$.
        \item $\forall n,p \in \mathbb{N}$, on a : $U_n \le V_p$.
        \item $\forall n \in \mathbb{N}$, on a : $U_n \le l \le V_n$.
    \end{itemize}
    \begin{proof}
    \begin{itemize}
        \item Supposons que $\exists n_0,p_0 \in \mathbb{N}$ tel que $U_{n_0} > V_{p_0}$.\\Alors, pour tout $n \ge \max(n_0,p_0)$, on a : $U_n \ge U_{n_0} > V_{p_0} \ge V_n$.\\Donc, $\forall n \ge \max(n_0,p_0)$, on a : $U_n - V_n \ge U_{n_0} - V_{p_0}$.\\Donc $lim_{n \to +\infty} (U_n - V_n) =0 \ge U_{n_0}-V_{p_0}\\ \quad\quad\implies U_{n_0} \le V_{p_0}$.
        \item On a $(U_n)_n$ croissante et majorée par $V_0$, donc elle converge vers $l_1\in\mathbb{R}$.\\On a $(V_n)_n$ décroissante et minorée par $U_0$, donc elle converge vers $l_2 \in \mathbb{R}$.\\Or, $V_n - U_n \to 0$, alors $l_2 - l_1 = 0 \implies l_1 = l_2 = l \in \mathbb{R}$.
        \item On a $\forall n,p \in \mathbb{N}, U_n \le V_p$. Donc $p\to +\infty \implies \forall n \in \mathbb{N} U_n \le \lim_{p\to +\infty} V_p = l$.\\De même, $\forall p \in \mathbb{N}, l \le V_p$.\\Donc : $\forall n \in \mathbb{N}, U_n \le l \le V_n$.
    \end{itemize}
\end{proof}
\end{theorem}

\begin{example}
    Soit $\begin{cases}
        U_n = \sum_{k=0}^{n} \frac{1}{k!} \\
        V_n = U_n + \frac{1}{n\cdot n!}
    \end{cases}$
\end{example}
\todo{Clair, voir cahier.}

\begin{exercise}
    On pose $\forall n \in \mathbb{N} : I_n = \int_{0}^{1}\frac{(1-t)^n}{n!}\cdot e^{-t}dt$. Montrer, à l'aide de $(I_n)_n$ que $\lim U_n = \lim V_n = e$. Puis, montrer que $e\notin\mathbb{Q}$.
\end{exercise}

\begin{theorem}[Théorème des segments emboîtés]
    Soit $(I_1=[a_n,b_n])_n$ une suite décroissante de segments (i.e. $\forall n \in \mathbb{N}, I_{n+1} \subset I_n$), avec $\delta(I_n)=b_n-a_n \to 0$. Alors, $\bigcap_{n=1}^{+\infty} I_n = \{l\}$ avec $l \in \mathbb{R}$.
    \begin{proof}
    On a : $\forall n \in \mathbb{N}, I_n = [a_n,b_n]$.
    \begin{itemize}
        \item On a : $\forall n \in \mathbb{N}: I_{n+1}\subset I_n \iff [a_{n+1},b_{n+1}] \subset [a_n,b_n] \\ \iff \forall n \in \mathbb{N}: a_n \le a_{n+1} \le b_{n+1} \le b_n$. Donc $(a_n)_n$ est croissante et $(b_n)_n$ est décroissante.
        \item On a : $\delta(I_n) = b_n-a_n \to 0$. Donc $(a_n)_n$ et $(b_n)_n$ sont adjacentes. Donc, par le théorème des suites adjacentes, $(a_n)_n$ et $(b_n)_n$ convergent vers une même limite $l \in \mathbb{R}$.
        \item On a : $\forall n \in \mathbb{N}, a_n \le l \le b_n$. Donc $\forall n \in \mathbb{N}, l \in [a_n,b_n] = I_n$. Donc ${l} \subset \bigcap_{n=1}^{+\infty} I_n$.
        \item Soit $x \in \bigcap_{n=1}^{+\infty} I_n$. Donc $\forall n \in \mathbb{N}, x \in I_n = [a_n,b_n]$. Donc $\forall n \in \mathbb{N}, a_n \le x \le b_n$. En passant à la limite, on a $l \le x \le l$. Donc $x = l$. Donc $\bigcap_{n=1}^{+\infty} I_n \subset {l}$.
        \item En combinant les deux inclusions, on obtient : $\bigcap_{n=1}^{+\infty} I_n = {l}$.
    \end{itemize}
\end{proof}
\end{theorem}

\begin{theorem}[Théorème de Bolzano-Weierstrass]
    Soit $(U_n)_n \in {\mathbb{R}}^{\mathbb{N}}$ une suite bornée. Alors, il existe une sous-suite $(U_{\varphi(n)})_n$ qui converge vers une limite $l \in \mathbb{R}$, i.e. $(U_n)_n \in \mathbb{B(R)} \implies \exists (U_{\varphi(n)})_n \text{ sous-suite de } (U_n)_n, (U_{\varphi(n)})_n \to l \in \mathbb{R}$.

\end{theorem}

\begin{proof}
    On a : $(U_n)_n\in \mathbb{B(R)} \iff \exists a_0,b_0 \in \mathbb{R}, \forall n \in \mathbb{N}, a_0 \le U_n \le b_0$.
    \\On pose $\varphi(0) = 0$, et $c_0$ le milieu de $[a_0,b_0]$.
    \begin{center}
        On pose : $$\begin{cases}
            A_1=\{k>0 / a_0 \le U_k \le c_0\} \\
            A_2=\{k>0 / c_0 \le U_k \le b_0\}
        \end{cases}$$\\
    \end{center}
    On a : $A_1 \cup A_2 = \{k>0 / a_0 \le U_k \le b_0\}=\mathbb{N}^*$.\\
    Donc $A_1$ ou $A_2$ est infini.
    \begin{itemize}
        \item Si $A_1$ est infini, on pose $a_1=a_0,b_1=c_0$. On choisit $\varphi(1) \in A_1$, donc $U_{\varphi(1)}\in[a_0,c_0]=[a_1,b_1]$, donc $b_1-a_1=\frac{b_0-a_0}{2}=\frac{b_0-a_0}{2}$, et $\varphi(1)>\varphi(0) = 0$.
        \item Si $A_2$ est infini, on pose $a_1=c_0,b_1=b_0$. On choisit $\varphi(1) \in A_2$, et on a $a_1\le U_{\varphi(1)} \le b_1$, donc $b_1-a_1=\frac{b_0-a_0}{2}=\frac{b_0-a_0}{2}$ et $\varphi(1)>\varphi(0)$.
    \end{itemize}
    \textbf{Hypothèse de récurrence :} Soit $n \in \mathbb{N}$. Supposons que $\exists a_1,\dots,a_n,b_1,\dots,b_n \in \mathbb{R}$ et $\varphi$ qui vérifient :
    $$
    \begin{cases}
        \forall k \in [0,n], a_k \le U_{\varphi(k)} \le b_k \\
        \forall k \in [0,n], b_k - a_k = \frac{b_0 - a_0}{2^k} \\
        \forall k \in [0,n], \varphi(k) > \varphi(k-1)\\
        \{k>\varphi(n)/a_n\le U_k\le b_n\}=E\text{ est infini}
    \end{cases}
    $$
    Soit $c_n$ le milieu de $[a_n,b_n]$. On pose $E_1=\{k>\varphi(n)/a_n\le U_k\le c_n\}$ et $E_2=\{k>\varphi(n)/c_n\le U_k\le b_n\}$.\\
    On a : $E_1 \cup E_2 = E$. Donc $E_1$ ou $E_2$ est infini.
    \begin{itemize}
        \item Si $E_1$ est infini, on pose $a_{n+1}=a_n,b_{n+1}=c_n$. On choisit $\varphi(n+1) \in E_1$, donc :
        $$
        \begin{cases}
            \varphi(n+1) > \varphi(n) > \dots > \varphi(0) \text{ d'après H.R.} \\
            a_n \le U_{\varphi(n+1)} \le c_n\\
            b_{n+1}-a_{n+1}=\frac{b_n-a_n}{2}=\frac{b_0-a_0}{2^{n+1}}
        \end{cases}
        $$
        \item Si $E_2$ est infini, on suit un raisonnement similaire.
    \end{itemize}
    On a construit deux suites $(a_n)_n$ et $(b_n)_n$ telles que $\varphi : \mathbb{N} \to \mathbb{N}$ est strictement croissante et :
    $$\begin{cases}
        \forall n \in \mathbb{N}, [a_{n+1},b_{n+1}] \subset [a_n,b_n]\\
        \forall n \in \mathbb{N}, a_n \le U_{\varphi(n)} \le b_n \\
        \forall n \in \mathbb{N}, b_n - a_n = \frac{b_0 - a_0}{2^n} 
    \end{cases}
    $$

    $(1)+(2) \implies (a_n)_n et (b_n)_n$ sont deux suites adjacentes.\\
    Donc, par le théorème des suites adjacentes, $(a_n)_n$ et $(b_n)_n$ convergent vers une même limite $l \in \mathbb{R}$.
    D'où $(3) \implies \forall n \in \mathbb{N} : U_{\varphi(n)} \to l \in \mathbb{R}$.\\
\end{proof}

\begin{remark}[Récurrence forte]
    \begin{itemize}
        \item En récurrence simple, on prouve $P(n)$ en supposant $P(n-1)$.
        \item En récurrence forte, on prouve $P(n)$ en supposant $\forall k \leq n, P(k)$.
        \item Exemple : Pour prouver que $(U_n)_n$ est croissante, on peut supposer que $\forall k < n, U_k \le U_{k+1}$.
    \end{itemize}
\end{remark}

\begin{remark}
    $I$ est dit intervalle de $\mathbb{R}$ ssi $\forall a<b \in I, [a,b] \subset I$.\\
    Un segment est un intervalle fermé et borné.
\end{remark}

\chapter{Suites récurrentes}

\begin{definition}
    Soient $D$ une partie non-vide de $\mathbb{R}$ et $f : D \to \mathbb{R}$. Soit $I$ un intervalle de $D$ stable par $f$, i.e. $f(I) \subset I$ (ce qui équivaut à dire que $\forall x \in I, f(x) \in I$). Soit la suite $(U_n)_n$ définie par $U_0 \in I$ et $\forall n \in \mathbb{N}, U_{n+1} = f(U_n)$. La suite $U_n$ est appelée la \textbf{suite récurrente} associée à $f$ (d'ordre 1).
\end{definition}

\begin{remark}
    La condition $f(I) \subset I$ est essentielle pour que la suite soit bien définie.

    \begin{proof}[Contre-exemple]
    On pose : $f:x\to \sqrt{x-1}$, $I=]1,+\infty[$ et $U_0=5$.\\
    On a : $U_1=f(U_0)=\sqrt{5-1}=2$, $U_2=f(U_1)=\sqrt{2-1}=1$.\\
    Or $1 \notin I$, donc $U_3$ n'existe pas et donc la suite n'est pas définie pour tout $n$. Le problème vient du fait que $f(I) = \mathbb{R}_+^* \not\subset I$.
    \end{proof}
\end{remark}

\begin{remark}
    Si $U_0 =a$ et $\forall n \in \mathbb{N}:U_{n+1} = f(U_n)$, alors la suite $(U_n)_n$ est entièrement déterminée par $U_0$, et on peut écrire:
    \begin{keyformula}
        $$\forall n \in \mathbb{N}: U_n = f^n(a)$$
    \end{keyformula}
    où $f^n$ est la composée de $f$ avec elle-même $n$ fois (avec la convention $f^0 = id_\mathbb{R}$).
\end{remark}

\section{Monotonie}

\begin{property}
    \begin{enumerate}
        \item $\forall x \in I: \begin{cases}
            f(x) \ge x \implies (U_n)_n \text{ est croissante} \nearrow \\
            f(x) \le x \implies (U_n)_n \text{ est décroissante} \searrow
        \end{cases}$
        \item Si $f$ est croissante sur $I$, si $U_0 \le U_1$, alors $(U_n)_n$ est croissante, et si $U_0 \ge U_1$, alors $(U_n)_n$ est décroissante. \textit{Ce résultat est démontré par récurrence}.
        \item Si $f$ est décroissante sur $I$ :\\ $\forall n \in \mathbb{N}: \begin{cases}
            U_{2n+2}=f(U_{2n+1})=f(f(U_{2n}))=f^2(U_{2n}) \\
            U_{2n+3}=f(U_{2n+2})=f(f(U_{2n+1}))=f^2(U_{2n+1})
            \end{cases}$\\Donc les suites $(U_{2n})_n$ et $(U_{2n+1})_n$ sont récurrentes d'ordre 1 associées à $f^2=f \circ f$, qui est croissante sur $I$. Donc $U_{2n}$ et $U_{2n+1}$ sont chacune monotones (voir point précédent).
    \end{enumerate}
\end{property}

\section{Convergence}

\begin{property}
    Supposons que $f$ est continue sur $\bar{I}\subset \mathbb{R}$ avec $\bar{I}=I \cup \{ \inf(I), \sup(I) \}$ (exemple : $I=]1, +\infty[$, alors $\bar{I}=[1,+\infty[$).
    \\Si $U_n$ converge vers $l \in \bar{I}$, alors $l=f(l)$, i.e. la limite éventuelle de $U_n$ est une solution de l'équation $f(x)=x$ sur $\bar{I}$.
\end{property}

\begin{example}
    Étudier la convergence de la suite définie par $U_0 \in \mathbb{R}$ et $\forall n \in \mathbb{N}: U_{n+1}= \frac{U_n^2 +3}{4}$.
    \\On a : $\forall n \in \mathbb{N}^*, U_n \in [0, +\infty[$.
    \\On a : $\forall x \in [0, +\infty[, f(x) = \frac{x^2 +3}{4} \ge x \\\iff x^2 -4x +3 \ge 0 \\\iff (x-1)(x-3) \ge 0$.  \\Donc $f(x) \ge x$ pour $x \in [0,1] \cup [3,+\infty[$ et $f(x) \le x$ pour $x \in [1,3]$.
    \\\textbf{Cas 1 :}  $U_0 \in [0,1[$. On a : $f([0,1[) \subset [0,1[$. \\Donc $(U_n)_n$ est croissante et majorée par 1. \\Donc $U_n \to l \in [0,1[$. \\Or $l=f(l) \implies l^2 -4l +3=0 \implies l \in \{1,3\}$. \\Donc $\boxed{U_n\to1}$.
    \\\textbf{Cas 2 :} $U_0 \in ]1,3[$. \\On a : $f([1,3]) \subset [1,3]$. \\Donc $(U_n)_n$ est décroissante et minorée par 1. \\Donc $U_n \to l \in [1,3]$. Or $l=f(l) \implies l^2 -4l +3=0 \implies l \in \{1,3\}$. \\Or $l\le U_0 < 3\implies l\not=3$. \\Donc $\boxed{U_n\to1}$.
    \\\textbf{Cas 3 :} $U_0 \in [3,+\infty[$. \\On a : $f([3,+\infty[) \subset [3,+\infty[$. \\Donc $(U_n)_n$ est croissante. Supposons que $U_n$ est majorée. \\Donc $U_n \to l \in [3,+\infty[$. Or $l=f(l) \implies l^2 -4l +3=0 \implies l \in \{1,3\}$. \\Or $l\ge 3\implies l=3$. \\Donc $\boxed{U_n\to3}$ si $(U_n)_n$ est majorée.\\Cela impliquerait que $3 \ge U_0$, or $U_0 > 3$ car $U_n \nearrow$. \\Donc, comme $(U_n)_n$ est croissante et non-majorée, on a : $\boxed{U_n \to +\infty}$.
    \\\textbf{Cas 4 :} $U_0 \in \{1,3\}$. \\On a : $\boxed{U_n = U_0\to U_0}$.
\end{example}

\chapter{Suites complexes}

\begin{definition}
    Soit $n\in\mathbb{N}$. On appelle suite complexe toute application $U : \begin{cases}
    \{k\in\mathbb{N}/k\ge n_0\} \to \mathbb{C}\\
    n\rightarrowtail U(n)=U_n
    \end{cases}$.
    \\On note l'ensemble des suites complexes par ${\mathbb{C}}^{\mathbb{N}}$.
\end{definition}

\begin{remark}
    Soit $(Z_n)_n \in \mathbb{C}^{\mathbb{N}}$. Alors : $\forall n \in \mathbb{N}, Z_n = \Re(Z_n)+i\Im(Z_n)$ avec $\Re(Z_n), \Im(Z_n) \in \mathbb{R}$. Donc, on peut définir deux suites réelles $(\Re(Z_n))_n$ et $(\Im(Z_n))_n$. On peut alors définir une suite complexe par la réunion de deux suites réelles.
\end{remark}

\begin{example}
    \begin{enumerate}
        \item $\forall n \in \mathbb{N}, Z_n = \frac{1-n}{1+n}+ie^{-n}$.
        \item $\forall n \in \mathbb{N}, Z_n = \left(\frac{1-i}{2}\right)^n$.
    \end{enumerate}
\end{example}

\begin{definition}[Suite bornée dans $\mathbb{C}$]
    Soit $(Z_n)_n \in {\mathbb{C}}^{\mathbb{N}}$. On dit que $(Z_n)_n$ est bornée ssi la suite des modules $(|Z_n|)_n$ est bornée dans $\mathbb{R}$, i.e. ssi $\exists M \in \mathbb{R}, \forall n \in \mathbb{N}, |Z_n| \le M$.\\On note $\mathbb{B(C)}$ l'ensemble des suites bornées dans $\mathbb{C}^{\mathbb{N}}$.
\end{definition}

\begin{example}
    $Z_n=\left(\frac{1-i}{2}\right)^n$ est bornée car : $\forall n \in \mathbb{N}, |Z_n| = \left|\frac{1-i}{2}\right|^n = \left(\frac{\sqrt{2}}{2}\right)^n \le 1$.
\end{example}

\begin{property}
    Soit $(Z_n)_n \in {\mathbb{C}}^{\mathbb{N}}$.\\On a : $(Z_n)_n \in \mathbb{B(C)} \iff (\Re(Z_n))_n \in \mathbb{B(R)} \text{ et } (\Im(Z_n))_n \in \mathbb{B(R)}$.
    \begin{proof}
    \begin{itemize}
        \item[$\implies$] $\begin{cases}
            \forall n \in \mathbb{N}, |\Re(Z_n)| \le |Z_n|\\
            \forall n \in \mathbb{N}, |\Im(Z_n)| \le |Z_n|
        \end{cases}$
        \item[$\impliedby$] Trivial.
    \end{itemize}
    \end{proof}
\end{property}

\begin{remark}
    \begin{enumerate}
        \item Les définitions et propriétés de convergence dans $\mathbb{R}$ s'étendent naturellement à $\mathbb{C}$ (elles sont analogues).
        \item Les opérations algébriques sur les limites des suites complexes sont analogues à celles des suites réelles.
        \item Une suite complexe est soit convergente soit elle n'admet pas de limite.
        \item La notion de suite monotone n'existe pas dans $\mathbb{C}^\mathbb{N}$ car $\mathbb{C}$ n'est pas ordonné. Cependant, on peut définir la notion de suite bornée (on vient de le faire $\uparrow\uparrow\uparrow$). La notion de suite adjacente n'existe pas non plus dans $\mathbb{C}^\mathbb{N}$.
    \end{enumerate}
\end{remark}

\begin{property}
    Soit $(Z_n)_n \in \mathbb{C}^\mathbb{N}$. On a : $Z_n \to l=a+ib\iff \begin{cases}
        \Re(Z_n) \to a \\
        \Im(Z_n) \to b
    \end{cases}$
    \begin{proof}
    \begin{itemize}
        \item[$\implies$] On a : $|\Re(Z_n)-a|=|\Re(Z_n-l)|\le|Z_n-l|\quad (\dots)$.
        \item[$\impliedby$] Trivial.
    \end{itemize}
    \end{proof}
\end{property}

\begin{exercise}
    Soit $(a_n)_n \in \mathbb{C}^\mathbb{N}$ tel que $a_n\to 0$. Montrer que $(1+\frac{a_n}{n})^n \to 1$.

    On a : $|(1+\frac{a_n}{n})^n-1|=|\sum_{k=0}^{n-1}C^k_n\frac{a_n^k}{n^k}|\le \sum_{k=1}^{n}C_n^k \frac{|a_n|^k}{n^k}=(1+\frac{|a_n|}{n})^n -1\to 0$. D'où $\dots$ CQFD.
\end{exercise}

\begin{theorem}[Théorème de Bolzano-Weierstrass dans $\mathbb{C}^\mathbb{N}$]
    Toute suite bornée dans $\mathbb{C}^\mathbb{N}$ admet une sous-suite convergente.
    \begin{proof}
    Soit $(Z_n)_n \in \mathbb{B(C)}$. Donc, par la propriété précédente, $(\Re(Z_n))_n$ et $(\Im(Z_n))_n$ sont dans $\mathbb{B(R)}$.\\
    On pose : $x_n = \Re(Z_n)$ et $y_n = \Im(Z_n)$.\\
    On a : $(x_n)_n \in \mathbb{B(R)}$. Donc, par le théorème de Bolzano-Weierstrass dans $\mathbb{R}^\mathbb{N}$, il existe une sous-suite $(x_{\varphi(n)})_n$ qui converge vers $l_1 \in \mathbb{R}$.\\
    On a : $(y_{\varphi(n)})_n \in \mathbb{B(R)}$. Donc, par le théorème de Bolzano-Weierstrass dans $\mathbb{R}^\mathbb{N}$, il existe une sous-suite $(y_{\varphi(\psi(n))})_n$ qui converge vers $l_2 \in \mathbb{R}$.\\
    Et on a : $x_{\varphi(\psi(n))} \to l_1$, parce que c'est une sous-suite de $x_{\varphi{n}}$.\\
    On pose : $\tilde{\varphi} = \varphi \circ \psi$. On a bien : $(x_{\tilde{\varphi}(n)}, y_{\tilde{\varphi}(n)})$ convergentes.\\
    Donc : $Z_{\tilde{\varphi}(n)} = x_{\tilde{\varphi}(n)} + i y_{\tilde{\varphi}(n)} \to l_1 + i l_2 \in \mathbb{C}$.\\
    Donc : $(Z_n)_n$ admet une sous-suite convergente. CQFD.
    \end{proof}
\end{theorem}

\begin{exercise}
    Soient $(a_n)_n,(b_n)_n,(c_n)_n,(d_n)_n \in \mathbb{B(R)}$. \\Montrer que $\exists f : \mathbb{N}\to\mathbb{N}$ telle que $f$ str. croissante et : $(a_{f(n)})_n,(b_{f(n)})_n,(c_{f(n)})_n,(d_{f(n)})_n$ convergent.

    \textbf{Solution : } $$\forall n \in \mathbb{N}: \begin{cases}
        U_n = a_n + i b_n \\
        V_n = c_n + i d_n
    \end{cases}$$
\end{exercise}

\begin{application}
    Soit $(U_n)_n \in \mathbb{C}^\mathbb{N}$.\\
    On pose : $\begin{cases}
        P_n=\sum_{k=0}^{n}|U_k|\\
        S_n=\sum_{k=0}^{n}U_k
    \end{cases}$
    \\Montrer que : $(P_n)_n$ converge $\implies (S_n)_n$ converge.
    \begin{proof}
    Preuve à refaire (recopiage complètement faux)
    \end{proof}
\end{application}

\chapter{Complément : Moyenne de Cesàro}

\begin{definition}[Convergence au sens de Cesàro]
    Soit $(U_n)_n \in \mathbb{C}^\mathbb{N}$. On dit que $(U_n)_n$ converge au sens de Cesàro ssi la suite des moyennes partielles $(V_n)_n$ définie par : $\forall n \in \mathbb{N}, V_n = \frac{1}{n}\sum_{k=0}^{n} U_k$ converge dans $\mathbb{C}$.
\end{definition}

\begin{property}
    Si $(U_n)_n \in \mathbb{C}^\mathbb{N}$ tel que : $U_n \to l \in \mathbb{C}$, alors $(U_n)_n$ converge au sens de Cesàro et $\lim_{n\to +\infty} V_n = l$.
    \begin{proof}
    Soit $\varepsilon > 0$.\\
    Comme $U_n \to l$, il existe $n_0 \in \mathbb{N}$ tel que : $\forall n \ge n_0, |U_n - l| < \frac{\varepsilon}{2}$.\\
    Donc, pour tout $n \ge n_0$, on a :
    $$|V_n - l| = \left|\frac{1}{n}(\sum_{k=0}^{n} U_k) - l\right| = \left|\frac{1}{n}\sum_{k=0}^{n} (U_k - l)\right| \le \frac{1}{n}\sum_{k=0}^{n} |U_k - l|$$\\
    $$\implies |V_n - l| \le \frac{1}{n}\left(\sum_{k=1}^{n_0 -1} |U_k - l| + \sum_{k=n_0+1}^{n} |U_k - l|\right)$$\\
    $$\implies |V_n - l| \le \frac{1}{n}\left(\sum_{k=1}^{n_0 -1} |U_k - l| + (n - n_0)\frac{\varepsilon}{2}\right).$$\\
    $$\implies |V_n - l| \le \frac{\alpha}{n} + (n - n_0)\frac{\varepsilon}{2}.$$
    avec $\alpha = \sum_{k=1}^{n_0 -1} |U_k - l|$ (constante).\\
    Et on a : $\frac{\alpha}{n} \to 0$.\\
    Donc il existe $n_1 \in \mathbb{N}$ tel que : $\forall n \ge n_1, \frac{\alpha}{n} < \frac{\varepsilon}{2}$.\\
    Donc, pour tout $n \ge \max(n_0,n_1)$, on a : $$|V_n - l| < \frac{\varepsilon}{2} + \frac{\varepsilon}{2} = \varepsilon.$$\\
    Donc $V_n \to l$. CQFD.\qed
    \end{proof}
\end{property}
    
\begin{property}
    Si $(U_n)_n \in \mathbb{R}^\mathbb{N}$ tel que $U_n\to +\infty$, alors $(V_n)_n \to + \infty$
    \begin{proof}
        On a : $\forall A > 0, \exists n_0 \in \mathbb{N}, \forall n \ge n_0, U_n > 2A$.\\
        Donc, pour tout $n \ge n_0$, on a :
        $$V_n = \frac{1}{n}\sum_{k=0}^{n} U_k \ge \frac{1}{n}$$
    \end{proof}
\end{property}

\begin{remark}
    Si $(U_n)_n \in \mathbb{R}^\mathbb{N}$ converge vers $l \in \bar{\mathbb{R}}$, alors $(V_n)_n$ converge vers $l$.\\
    \begin{attention}
        La réciproque est fausse : $(U_n)_n$ ne converge pas forcément si $(V_n)_n$ converge.
    \end{attention}
\end{remark}

\begin{theorem}[Cesàro généralisé]
    Soient $(U_n)_n \in \mathbb{C}^\mathbb{N}$ et $(\alpha_n)_n \in \mathbb{R}_+^{*\mathbb{N}}$ tel que : $\sum_{n=0}^{+\infty} \alpha_n = +\infty$.\\
    Si $lim_{n\to +\infty} U_n = a \in \mathbb{C}$, alors : $$V_n = \frac{1}{\sum_{k=0}^{n} \alpha_k} \sum_{k=0}^{n} \alpha_k U_k \to a$$
\end{theorem}

\todo{À DEMONTRER EN TANT QU'EXERCICE}

\end{document}